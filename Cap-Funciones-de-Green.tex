\chapter{Funciones de Green}

Hasta ahora, hemos visto maneras de resolver EDPs lineales y \emph{homogéneas}, es decir, que son igualables a cero, sin términos que no dependan de la función incógnita ni sus derivadas.

Sin embargo, muchas situaciones físicas no pueden ser descritas únicamente mediante ecuaciones homogéneas. ¿Cómo podemos resolverlas en este caso?

Una forma de hacerlo es mediante el \textbf{método de las funciones de Green}, gracias a las cuales podemos reducir una EDP lineal e inhomogénea a un problema abordable.

% En general, diremos que podemos hacer uso de las funciones de Green cuando tenemos un problema de la forma
% \begin{equation}
%     \mathcal{L}\Psi(\vec{x}) = f(\vec{x}) \ ,
% \end{equation}
% donde $f(\vec{x})$ corresponde a una función \emph{fuente} y $\mathcal{L}$ es un operador diferencial cualquiera. 
\begin{defi} \marginnote{Función de Green}
    Dado un operador diferencial $\mathcal{L}$ cualquiera, y una función \emph{fuente} $f(\x)$, tales que
    \begin{equation}
        \mathcal{L}\Psi(\vec{x}) = f(\vec{x}) \ .
    \end{equation}

    Una \textbf{función de Green} $G(\x, \x')$ para el operador $\mathcal{L}$ es aquella que satisface la ecuación
    \begin{equation}\label{eq:condicion_green}
        \mathcal{L}G(\vec{x}, \vec{x}') = \delta^{(n)}(\vec{x} - \vec{x}') \ ,
    \end{equation}
    donde $\delta^{(n)}$ es la delta de Dirac $n$-dimensional.
\end{defi}

\begin{propiedad} \marginnote{Condición de consistencia}
    \textbf{Condición de consistencia.} Integrando la definición de las funciones de Green \eqref{eq:condicion_green} sobre el volumen $V$ en el que se encuentre definida nuestro operador diferencial $\mathcal{L}$, la función de Green cumplirá que
    \begin{equation} \label{eq:condicion_consistencia}
        \int_V \mathcal{L} G(\x, \x') dV = \int_V \delta^{(n)}(\x - \x') dV = 1 \ , \quad \forall \ \x \in V \ .
    \end{equation}
\end{propiedad}
% En este caso, buscamos una \textbf{función de Green} $G(\vec{x}, \vec{x}')$ que satisfaga la ecuación
% \begin{equation}\label{eq:condicion_green}
%     \mathcal{L}G(\vec{x}, \vec{x}') = \delta^{(n)}(\vec{x} - \vec{x}') \ ,
% \end{equation}
% donde $\delta^{(n)}$ es la delta de Dirac $n$-dimensional.

\section{Motivación: Potencial electrostático}

En presencia de una carga (o distribución de cargas), el potencial electrostático satisface la \emph{ecuación de Poisson}
\begin{equation}
    \nabla^2 \phi = - \frac{\rho(x)}{\varepsilon_0} \ ,
\end{equation}
donde $\rho(x)$ es la distribución de cargas en la región considerada.

A partir de la ley de Coulomb, sabemos que podemos describir el potencial eléctrico de una distribución de cargas como
\begin{equation} \label{eq:potencial_coulomb}
    \phi(\vec{x}) = \frac{1}{4\pi \varepsilon_0} \int_V \frac{\rho(\vec{x}')}{|\vec{x} - \vec{x}'|} dV' \ ,
\end{equation}
donde hemos hecho la elección tradicional de que el potencial de referencia se anula en el infinito.

Observamos que, si definimos nuestra \emph{función de Green} como
\begin{equation}\label{eq:green_laplaciano} \marginnote{Función de Green para el Laplaciano}
    G(\vec{x}, \vec{x}') = - \frac{1}{4\pi} \frac{1}{|\vec{x} - \vec{x}'|} \ ,
\end{equation}
podemos reescribir \eqref{eq:potencial_coulomb} como
\begin{equation}
    \phi(x) = \int_V G(\vec{x}, \vec{x}') \left( - \frac{\rho(\vec{x}')}{\varepsilon_0} \right) dV' \ .
\end{equation}

Esta función de Green debe satisfacer la ecuación \eqref{eq:condicion_green}, que en este caso es dada por
\begin{equation}
    \nabla^2 G(\vec{x}, \vec{x}') = \delta^{(3)} (\vec{x} - \vec{x}') \ .
\end{equation}

De esta forma, observamos que
\begin{align}
    \nabla^2 \phi(\vec{x}) & = \nabla^2 \int_V G(\vec{x}, \vec{x}') \left( - \frac{\rho(\vec{x}')}{\varepsilon_0} \right) dV' \\
    & = \int_V [\nabla^2 G(\vec{x}, \vec{x}')] \left( - \frac{\rho(\vec{x}')}{\varepsilon_0} \right) dV' \\
    & = \int_V \delta^{(3)}(\vec{x} - \vec{x}') \left( - \frac{\rho(\vec{x}')}{\varepsilon_0} \right) dV'\\
    & = - \frac{\rho(\vec{x})}{\varepsilon_0} \ ,
\end{align}
recuperando la ecuación original.

\section{Encontrando soluciones mediante funciones de Green}

Consideremos una EDP \emph{lineal e inhomogénea} de la forma
\begin{equation}\label{eq:edp_general}
    \mathcal{L} \psi(\vec{x}) = \left(\nabla \cdot (p(\vec{x}) \nabla) + q(\vec{x})\right)\psi(\vec{x}) = f(\vec{x}) \ ,
\end{equation}
donde $p(\vec{x})$ y $q(\vec{x})$ son funciones conocidas.

Para resolver la EDP, encontramos primero la función de Green que satisface la expresión \eqref{eq:condicion_green}, junto a una solución \emph{particular} de la ecuación \eqref{eq:edp_general} en el punto $\vec{x}'$, podemos hallar una solución general del problema, la que será dada por
\begin{equation}\label{eq:solucion_green}
    \psi(\vec{x}) = \oint_{\partial V} p(\vec{x})[\psi(\vec{x}') \nabla' G(\vec{x}, \vec{x}') - G(\vec{x}, \vec{x}') \nabla' \psi(\vec{x}')] \cdot d\vec{S}' + \int_V G(\vec{x}, \vec{x}') f(\vec{x}') dV' \ ,
\end{equation}
donde $\nabla'$ indica que el operador actúa sobre las componentes de $\vec{x}'$. Aquí, $V$ denota al dominio en que la solución $\psi(\vec{x})$ es válida, y $\partial V$ es su frontera.

¿Cómo sabemos que esta ecuación es válida? Resolvamos la primera integral. Notemos que podemos utilizar el teorema de Gauss, de modo que
\begin{align}
    I & = \oint_{\partial V} p(\vec{x})[\psi(\vec{x}') \nabla' G(\vec{x}, \vec{x}') - G(\vec{x}, \vec{x}') \nabla' \psi(\vec{x}')] \cdot d\vec{S}' \label{eq:integral_green_superficie} \\
    & = \int_V \nabla' \cdot \left[ p(\vec{x}') \psi(\vec{x}') \nabla' G(\vec{x}, \vec{x}') - p(\vec{x}') G(\vec{x}, \vec{x}') \nabla' \psi(\vec{x}') \right] dV' \\
    & = \int_V \left[\psi(\vec{x}') \nabla' \cdot \left[ p(\vec{x}')  \nabla' G(\vec{x}, \vec{x}')\right] + \textcolor{blue}{\nabla \psi(\vec{x}') \cdot (p(\vec{x}') \nabla' G(\vec{x}, \vec{x}') )} \right. \nonumber \\
    & \qquad \quad \left. -  G(\vec{x}, \vec{x}') \nabla' \cdot \left[p(\vec{x}') \nabla' \psi(\vec{x}')\right] \textcolor{blue}{- \nabla \psi(\vec{x}') \cdot (p(\vec{x}') \nabla' G(\vec{x}, \vec{x}') )} \right] dV' \\
    & = \int_V \left[ \psi(\vec{x}') \left( \mathcal{L}' G(\vec{x}, \vec{x}') \right) - G(\vec{x}, \vec{x}') \left( \mathcal{L}' \psi(\vec{x}') \right) \right] dV' \\
    & = \int_V \left[ \psi(\vec{x}') \delta(\vec{x} - \vec{x}') - G(\vec{x}, \vec{x}') f(\vec{x}') \right] dV' \\
    & = \psi(\vec{x}) - \int_V G(\vec{x}, \vec{x}') f(\vec{x}') dV' \ ,
\end{align}
donde hemos definido un nuevo operador diferencial $\mathcal{L}'$ que actúa sobre las componentes primadas, tal que $\mathcal{L}' \psi(\vec{x}') = f(\vec{x}')$.

¿Cómo hallamos una solución en el punto $\vec{x}'$? Haciendo uso de las condiciones de borde del problema, que usualmente serán definidas en el contorno $\partial V$.

\subsection{Condiciones de borde de tipo Dirichlet}

Si las condiciones de borde son \emph{de tipo Dirichlet}, sabemos el valor de $\psi(\vec{x}')$ en el borde $\partial V$ de la región en que nos encontramos, por lo que podremos determinar el primer término de la integral sobre $\partial V$ en \eqref{eq:integral_green_superficie}, pero el segundo, que depende de la derivada de $\psi(\vec{x}')$, no estará determinado. Por ello, escogeremos \emph{de forma conveniente} una función de Green que se anule en los bordes, es decir,
\begin{equation}\label{eq:condicion_dirichlet_green}
    G(\vec{x}, \vec{x}') = 0 \ , \qquad \forall \ \vec{x}' \in \ \partial V \ .
\end{equation}

En este caso, la solución general tendrá la forma
\begin{equation}\label{eq:solucion_dirichlet}
    \psi(\vec{x}) = \int_V G(\vec{x}, \vec{x}') f(\vec{x}') dV + \oint_{\partial V} p(\vec{x}') \psi(\vec{x}') \frac{\partial G(\vec{x}, \vec{x'})}{\partial n'} dS' \ ,
\end{equation}
donde $\partial f/\partial n' = (\nabla' f) \cdot \hat{n}$ , con 
$\hat{n}$ un vector normal a la superficie.

Sin embargo, muchas veces puede ser difícil hallar directamente una función de Green que satisfaga a la vez \eqref{eq:condicion_dirichlet_green} y \eqref{eq:condicion_green}. Un ejemplo sencillo de esto es el operador laplaciano, cuando $\mathcal{L} = \nabla^2$. En este caso, es útil buscar una solución de la forma
\begin{equation}\label{eq:green_fundamental}
    G(\x, \x') = F(\x, \x') + H(\x, \x') \ ,
\end{equation}
donde $F(\x, \x')$ satisface la ecuación \eqref{eq:condicion_green} pero no necesariamente la condición de borde \eqref{eq:condicion_dirichlet_green}, mientras que $H(\x, \x')$ satisface la ecuación \emph{homogénea del problema} (es decir, donde no hay funciones fuente) en el interior de la región $V$, pero su valor en $\partial V$ es tal que, sumada a $F(\x, \x')$, resulta en que $G(\x,\x') = 0$ en el borde $\partial V$. Cuando seguimos este procedimiento, la función $F(\x,\x')$ es denominada \textbf{solución fundamental}. En efecto,
\begin{equation}
    \nabla^2 G(\x, \x') = \nabla^2 F(\x, \x') + \nabla^2 H(\x, \x') = \nabla^2 F(\x, \x') + 0 = \delta^{(3)}(\x - \x') \ .
\end{equation}

\begin{ejemplo}
    \textbf{(Riley, Sección 21.5.3)} Encuentre la solución fundamental a la ecuación de Poisson en tres dimensiones, sujeta a la condición de contorno $F(\x , \x') \to 0$ cuando $|\x| \to \infty$.

    \textbf{Solución.}  Dado que nuestro contorno se encuentra en infinito, el problema posee simetría esférica, y puede ser modelado como una esfera $S$ de radio $R$ \emph{muy grande} centrada en $\x'$. La condición de consistencia \eqref{eq:condicion_consistencia} establece
    \begin{equation*}
        \int_V \nabla^2 F(\x, \x') dV = \int_V \delta^{(3)}(\x-\x') dV = 1 \ ,
    \end{equation*}
    que a su vez, usando el teorema de Gauss, puede escribirse como
    \begin{equation}
        \int_S \nabla F(\x, \x') \cdot d\vec{S}' = 1 \ .
    \end{equation}

    Dada la simetría esférica del problema, podemos suponer que nuestra función dependa únicamente de la distancia entre el centro y el punto $\x$, de modo que $F(\x,\x') = F(|\x-\x') = F(r)$. Evaluando la integral de superficie, tenemos que
    \begin{equation*}
        4\pi r^2 \left. \frac{dF}{dr}\right|_{r=R} = 1 \ .
    \end{equation*}

    Integrando sobre $r$, podemos encontrar una expresión para $F$, de modo que
    \begin{equation*}
        F(r) = - \frac{1}{4\pi r} + C \ ,
    \end{equation*}
    que sumado a la condición de contorno del problema, deducimos que $C = 0$. Luego, la solución fundamental en tres dimensiones es dada por
    \begin{equation*}
        F(\x, \x') = - \frac{1}{4\pi |\x-\x'|} \ .
    \end{equation*}
\end{ejemplo}

\subsection{Condiciones de borde de tipo Neumann}


Si las condiciones de borde son \textbf{de tipo Neumann}, conocemos el valor de la derivada $\frac{\partial \psi}{\partial n}(\vec{x}')$ en el borde $\partial V$ de la región en que nos encontramos, por lo que podemos determinar el segundo término de la integral \eqref{eq:integral_green_superficie}, mas no así el primero. 

Por analogía al caso de las condiciones de Dirichlet, podríamos pensar en escoger una función de Green que satisfaga
\begin{equation} \label{eq:condicion_neumann_green}
    \frac{\partial G(\vec{x}, \vec{x}')}{\partial n'} = 0 \ , \qquad \forall \ \vec{x}' \in \ \partial V \ ,
\end{equation}
con lo que la solución general tendría la forma
% En este caso, la solución general tendrá la forma
\begin{equation}\label{eq:solucion_neumann}
    \psi(\x) = \int_V G(\x, \x') f(\x') dV - \oint_{\partial V} p(\x ') G(\x, \x') \frac{\partial \psi(\x')}{\partial n'} dS' \ .
\end{equation}

Lamentablemente, en general \emph{no es posible} encontrar una función de Green que satisfaga la condición \eqref{eq:condicion_neumann_green}, siendo nuevamente un ejemplo de esto el operador laplaciano $\nabla^2$, ya que las funciones de Green deben satisfacer la condición de consistencia \eqref{eq:condicion_consistencia}. Por ello, ya que descartamos la solución más sencilla posible ($\frac{\partial G(\vec{x}, \vec{x}')}{\partial n'} = 0$), escogemos la segunda más sencilla, igualar la derivada a una constante,
\begin{equation}
    \frac{\partial G(\vec{x}, \vec{x}')}{\partial n'} = C \ , \qquad \forall \vec{x}' \in \ \partial V \ .
\end{equation}
Para hallar el valor de esta constante, hacemos uso de la \emph{condición de consistencia}, de modo que el caso más sencillo será suponer que $C = 1/\text{Área}(\partial V)$.

De ser posible esta elección, la solución tendrá la forma
\begin{equation}\label{eq:solucion_neumann}
    \psi(\x) = \int_V G(\x, \x') f(\x') dV - \oint_{\partial V} p(\x ') G(\x, \x') \frac{\partial \psi(\x')}{\partial n'} dS' + \underbrace{\frac{1}{A} \oint_{\partial V} p(\x) \psi(\x') \ dS'}_{\langle \psi(\x') \rangle_{\partial V}} \ .
\end{equation}
Aquí, la notación $\langle \psi(\x') \rangle_{\partial V}$ representa una especie de \emph{valor promedio ponderado} sobre el contorno $\partial V$. Este promedio actuará como una constante que puede ser determinada libremente. Es más, si el contorno $\partial V$ de la región es el infinito, este término puede ser fijado a cero, pues no requeriremos de este término para que la solución tenga sentido físico.

% \begin{ejemplo}
%     \textbf{(Riley, Sección 21.5.4)} Resuelva la ecuación de Laplace en la región bidimensional $|\x| \leq a$ sujeta a la condición de contorno $\partial \Psi/\partial n = f(\phi)$ en $|\x| = a$. Considere que la condición de consistencia implica que $\int_0^{2\pi} f(\phi) d\phi = 0$.

%     \textbf{Solución.} 
% \end{ejemplo}

\begin{obs}{Observación}
    Es importante recordar que una misma EDP puede dar origen a diferentes funciones de Green, pues estas dependen también de las condiciones de contorno.
\end{obs}


\section{Simetría de las funciones de Green}

Cuando tenemos el caso particular en que $\mathcal{L} = \nabla^2$, la función de Green \eqref{eq:green_laplaciano} es simétrica bajo el intercambio de argumentos, es decir, $G(\x, \x') = G(\x', \x)$. Podemos buscar cuál es la condición general que permite que esto ocurra, escogiendo $\psi(\x) = \psi(\x'', \x)$, de modo que, según la expresión \eqref{eq:solucion_green},
\begin{align}
    \nonumber G(\x'', \x) & = \int_V G(\x, \x') \delta(\x'' - \x') dV' \\ 
    & \qquad + \oint_{\partial V} p(\x')\left[ G(\x'', \x') \nabla' G(\x, \x') - G(\x, \x') \nabla'G(\x'', \x') \right] \cdot d\vec{S}' \\
    & = G(\x, \x'') + \oint_{\partial V} p(\x')\left[ G(\x'', \x') \nabla' G(\x, \x') - G(\x, \x') \nabla'G(\x'', \x') \right] \cdot d\vec{S}' \ ,
\end{align}
con lo que vemos que la función de Green será simétrica siempre y cuando la integral de superficie se anule en la frontera, 
\begin{equation}\label{eq:condicion_simetria}
    \oint_{\partial V} p(\x')\left[ G(\x'', \x') \nabla' G(\x, \x') - G(\x, \x') \nabla'G(\x'', \x') \right] \cdot d\vec{S}' = 0 \ .
\end{equation}
% Esto ocurre, típicamente, para condiciones de tipo Dirichlet.
A continuación, listamos algunas situaciones en que una función de Green será simétrica.

\begin{propiedad}
    \textbf{Simetría de las funciones de Green.} Una función de Green para un operador diferencial $\mathcal{L}$ será simétrica, esto es, $G(\x, \x') = G(\x', \x)$, si 
    \begin{enumerate}
        \item El problema posee condiciones de tipo Dirichlet homogéneas.
        \item El problema posee condiciones de tipo Neumann homogéneas.
        \item El operador diferencial es \emph{hermítico}, esto es, $\mathcal{L} = \mathcal{L}^\dagger$.
        \item El operador diferencial es \emph{real}, ya que no poseerá parte imaginaria por conjugar.
        \item Cualquier otra condición que satisfaga la expresión \eqref{eq:condicion_simetria}.
    \end{enumerate}
\end{propiedad}

\section{Método de las imágenes}

Anteriormente mencionamos que podemos hallar la función de Green para un problema con condiciones de Dirichlet hallando la solución fundamental y una solución que produzca que $G(\x,\x') = 0$ en $\partial V$, como establecimos en \eqref{eq:green_fundamental}. Para ello, podemos hacer uso del \textbf{método de las imágenes}, en el cual hemos de considerar \emph{copias} de nuestra solución fundamental que representen \emph{fuentes} ubicadas en el exterior de nuestra región $V$. Este método es de gran utilidad en problemas de electrostática que presenten geometrías altamente simétricas, y será aplicado en este contexto en el curso de Electrodinámica I.
\begin{propo} \marginnote{Método de las imágenes}
    \textbf{Método de las imágenes.} Es posible hallar una función de Green para una ecuación inhomogénea al sumar a la solución fundamental $F(\x, \x')$ diferentes copias (o \emph{imágenes}) de la ella misma, ubicadas \emph{fuera} de la región $V$ en la que deseamos resolver la ecuación. Para ello, seguiremos el siguiente procedimiento,
    \begin{enumerate}
        \item Para una fuente singular $\delta(\x - \x')$ dentro de la región $V$, añadiremos fuentes imágenes \emph{fuera} de $V$, donde las posiciones $\x_n$ e intensidades $q_n$ de estas fuentes las podremos determinar a futuro, de modo que las fuentes externas se pueden expresar como
        \begin{equation}
            \sum_{n=1}^N q_n \delta(\x - \x_n) \ , \qquad \x_n \notin \ V \ .
        \end{equation}
    
        \item Ya que todas las imágenes se encuentran en el exterior de $V$, la solución fundamental que corresponde a cada una de las fuentes deberá satisfacer la ecuación de Laplace \emph{dentro} de $V$. Por ello, podemos suponer que cada imagen tiene asociada la misma \emph{solución fundamental} que la fuente en el interior de $V$, con lo que la función de Green tomará la forma
        \begin{equation}
            G(\x, \x') = F(\x, \x') + \sum_{n=1}^N q_n F(\x,\x_n) \ .
        \end{equation}
    
        \item Ajustamos las posiciones $\x_n$ e intensidades $q_n$ de las imágenes de modo que las condiciones de contorno se satisfagan en $S$. En el caso de las condiciones de Dirichlet, esto es exigir que $G(\x, \x') = 0$, para cualquier $\x' \in \partial V$.
    
        \item Por último, podemos usar la función de Green hallada para encontrar la solución, sujeta a las condiciones de Dirichlet, que satisfaga \eqref{eq:solucion_dirichlet}.
    \end{enumerate}
\end{propo}

En general, no es tarea sencilla encontrar las posiciones e intensidades correctas para cualquier problema, pero sí lo es para ciertos problemas con geometrías simples. Particularmente, este método es útil para problemas en los que los bordes sean \emph{líneas rectas} (problema bidimensional) o \emph{planos} (problema tridimensional), pues en esos casos simplemente supondremos que estos actúan como espejos, de modo que la fuente \emph{verdadera} se refleja simétricamente en este espejo.

\begin{ejemplo}
    \textbf{(Riley, Sección 21.5.3)} Resuelva la ecuación de Laplace en la región bidimensional $|\x| \leq a$ sujeta a la condición de contorno $u = h(\phi)$ en $|\x| = a$, dada una carga fuente ubicada en $\x_0$.

    \textbf{Solución.} Nuestro problema corresponde a un disco de radio $a$ sujeto a condiciones de Dirichlet. Como queremos que nuestra función de Green satisfaga que $G(\x, \x_0) = 0$ para $|\x| = a$, podemos suponer que nuestra \emph{carga imagen} se encuentra fuera del disco en una posición $\x_1 = (a^2/|\x_0|^2)\x_0$, de modo que, suponiendo que la intensidad de la carga imagen es $-1$, nuestra función de Green toma la forma (véase la siguiente sección para la deducción detallada de la función de Green para el Laplaciano en 2D)
    \begin{equation*}
        G(\x, \x_0) = \frac{1}{2\pi} \ln(|\x - \x_0|) - \frac{1}{2\pi} \ln(|\x - \x_1|) + C \ .
    \end{equation*}

    Del hecho que $G(\x, \x_0) = 0$ para $|\x| = a$, tenemos que
    \begin{align*}
        2\pi C & = \ln(|\x - \x_1|) - \ln(|\x - \x_0|) \\
        & = \ln\left( \frac{|\x - \x_1|}{|\x - \x_0|} \right) \\
        & = \ln \left( \frac{\sqrt{a^4/x_0^2 - 2 a^3 \cos(\phi - \phi_0)/x_0 + a^2}}{|\x - \x_0|} \right) \\
        & = \ln \left( \frac{\sqrt{(a^2/x_0^2)}\cancel{|\x - \x_0|}}{\cancel{|\x - \x_0|}} \right) \\
        & = - \ln \left( \frac{x_0}{a} \right)
        % \\
        % & = \ln\left( \sqrt{\frac{a^2}{x_0^2}} \right) \ ,
    \end{align*}
    donde $x_0 = |\x_0|$. Así,
    \begin{equation*}
        C = - \frac{1}{2\pi} \ln \frac{x_0}{a} \ ,
    \end{equation*}
    y nuestra función de Green toma la forma 
    \begin{equation*}
        G(\x, \x_0) = \frac{1}{2\pi} \left[ \ln|\x - \x_0| - \ln\left| \x - \frac{a^2}{x_0^2} \x_0 \right| - \ln \frac{x_0}{a} \right] \ .
    \end{equation*}

    Conociendo nuestra función de Green, podemos utilizar la ecuación \eqref{eq:solucion_dirichlet} para hallar una solución a nuestro problema. Como trabajamos con la ecuación de Laplace, $f(\x_0) = 0$, de modo que solo sobrevive la integral sobre el contorno de nuestra región en \eqref{eq:solucion_dirichlet}, con lo cual
    \begin{equation*}
        \psi(\x) = \oint_C \psi(\x') \frac{\partial G(\x, \x')}{\partial n'} \ d\ell' = \int_0^{2\pi} \psi(\x') \left. \frac{\partial G}{\partial \rho'}\right|_{\rho' = a} a \ d\phi' \ .
    \end{equation*}

    Para nuestra función de Green,
    \begin{align*}
        \frac{\partial G}{\partial \rho'} & = \hat{x}' \cdot \nabla' G(\x,\x') \\
        & = \hat{x}' \cdot \nabla' G(\x',\x) \\
        & = \frac{\x'}{2\pi x'} \cdot \left( \frac{\x' - \x}{|\x' - \x|^2} - \frac{\x' - (a^2/x^2)\x}{|\x' - (a^2/x^2)\x|^2} \right) \ .
    \end{align*}
    que en $\rho' = a$ satisface que
    \begin{align*}
        \left. \frac{\partial G}{\partial \rho'} \right|_{\rho' = a} & = \frac{\x'}{2\pi a} \cdot \left( \frac{\x' - \x}{|\x' - \x|^2} - \frac{\x' - (a^2/x^2)\x}{(a^2/x^2)|\x' - \x|^2} \right) \\
        & = \frac{\x'}{2\pi a} \cdot \frac{(a^2/x^2)\x' - (a^2/x^2) \x - \x' + (a^2/x^2)\x }{(a^2/x^2)|\x' - \x|^2} \\
        & = \frac{1}{2\pi a} \frac{a^2 - x^2}{|\x' - \x|^2} \\
        & = \frac{1}{2\pi a} \frac{a^2 - \rho^2}{a^2 + \rho^2 - 2a\rho \cos(\phi' - \phi)} \ .
    \end{align*}
    
    De esta manera, concluimos que una solución a nuestro problema puede ser hallada como
    \begin{equation*}
        u(\rho, \phi) = \frac{1}{2\pi} \int_0^{2\pi}  \frac{(a^2 - \rho^2) f(\phi') \ d\phi'}{a^2 + \rho^2 - 2a\rho \cos(\phi' - \phi)} \ .
    \end{equation*}
\end{ejemplo}

También es útil mencionar que es válido utilizar el método de las imágenes al trabajar con condiciones de Neumann, donde el procedimiento descrito anteriormente sigue siendo el mismo, pero utilizando las condiciones de Neumann para establecer nuestras imágenes.

\begin{ejemplo}
    \textbf{(Riley, Sección 21.5.4)} Resuelva la ecuación de Laplace en la región bidimensional $|\x| \leq a$ sujeta a la condición de contorno $\partial u/\partial n = f(\phi)$ en $|\x| = a$, con $\int_0^{2\pi} f(\phi) d\phi = 0$, como requiere la condición de consistencia.

    \textbf{Solución.} La geometría del problema es la misma que la del ejemplo anterior. Por ello, podemos suponer que la carga imagen se encuentra en la misma posición $\x_1 = (a^2/x_0^2) \x_0$, donde nuevamente $\x_0$ corresponde a la carga fuente dentro del círculo. Sin embargo, ahora no asumiremos la intensidad de la carga imagen, sino que la dejaremos como un parámetro $q$. De esta forma, la función de Green será de la forma
    \begin{equation*}
        G(\x, \x_0) = \frac{1}{2\pi} \left( \ln |\x - \x_0| + q \ln |\x - \x_1| + C \right) \ .
    \end{equation*}

    En este caso, la derivada radial (normal) de esta función es dada por
    \begin{equation*}
        \frac{\partial G}{\partial \rho} = \frac{\x}{|\x|} \cdot \nabla G(\x, \x_0) = \frac{\x}{2\pi |\x|} \cdot \left[ \frac{\x - \x_0}{|\x - \x_0|^2} + q \frac{\x - \x_1}{|\x - \x_1|^2} \right] \ ,
    \end{equation*}
    que evaluada en $\rho = a$, tenemos que
    \begin{equation*}
        \left. \frac{\partial G}{\partial \rho} \right|_{\rho = a} = \frac{1}{2\pi a} \frac{1}{|\x - \x_0|^2} \left[ |\x|^2 + qx_0^2 - (1+q) \x \cdot \x_0 \right] \ .
    \end{equation*} 

    Como se discutió anteriormente, en problemas con condiciones de tipo Neumann es conveniente escoger que esta derivada se iguale a alguna constante, típicamente el inverso del valor del ``área'' del contorno de nuestra región. Dado que estamos trabajando en una región bidimensional, esta ``área'' corresponde al perímetro del círculo, de modo que
    \begin{equation*}
        \left. \frac{\partial G}{\partial \rho} \right|_{\rho = a} = \frac{1}{2\pi a} \ ,
    \end{equation*}
    lo que se cumple escogiendo $q = 1$. Así, observamos que para un problema con la misma geometría, las condiciones de contorno afectarán al valor de la intensidad $q$ de la carga imagen.

    Podemos encontrar la solución a nuestro problema a partir de la ecuación \eqref{eq:solucion_neumann}, donde al tratarse de la ecuación de Laplace, $f(x) = 0$, con lo que la solución será de la forma
    \begin{equation*}
        \psi(\x) = \langle \psi(\x) \rangle_{C} - \oint_{\partial V} G(\x', \x) f(\x') \ d\ell' 
    \end{equation*}

    Así, la función de Green adecuada, al ser evaluada en $\rho' = a$, tomará el valor
    \begin{align*}
        G(\x',\x)|_{\rho' = a} & = \frac{1}{2\pi} \left[ \ln |\x' - \x| + \ln |\x' - (a^2/x^2)\x| + C \right] \\
        & = \frac{1}{2\pi} \left[ \ln\left( |\x' - \x| |\x - (a^2/x^2)\x| \right) + C \right] \\
        & = \frac{1}{2\pi} \left[ \ln\left( |\x' - \x|^2 \right) + \ln\frac{a}{x} C \right] \\
        & = \frac{1}{2\pi} \left[ \ln(a^2 + \rho^2 - 2 a \rho \cos(\phi' - \phi)) + \ln \frac{a}{\rho} + C \right] \ ,
    \end{align*}
    donde la constante $C$, es una constante arbitraria. Entonces,
    \begin{align*}
        \psi(\rho, \phi) & = \langle \psi(\x) \rangle_C - \frac{a}{2\pi} \int_0^{2\pi} f(\phi') \ln[a^2 + \rho^2 - 2a\rho \cos(\phi' - \phi)] \ d\phi' \\
        & \qquad - \frac{a}{2\pi} \left(\ln \frac{a}{\rho} + C \right) \int_0^{2\pi} f(\phi') \ d\phi' \\
        & = \langle \psi(\x) \rangle_C - \frac{a}{2\pi} \int_0^{2\pi} f(\phi') \ln[a^2 + \rho^2 - 2a\rho \cos(\phi' - \phi)] \ d\phi' \ ,
    \end{align*}
    donde hemos usado la condición dada en el enunciado, $\int_0^{2\pi} f(\phi) d\phi = 0$. 

\end{ejemplo}


\section{Algunas funciones de Green comunes}

Como mencionamos anteriormente, las funciones de Green no son únicas, sino que pueden construirse siempre a partir de una solución fundamental $F(\x,\x')$ que también es una función de Green, sumada a una función $H(\x,\x')$ que es solución al problema homogéneo asociado al operador $\mathcal{L}$,
\begin{equation}
    \mathcal{L} H(\x,\x') = 0 \ .
\end{equation} 

Por ello, listamos a continuación las soluciones fundamentales para el operador Laplaciano y para el Operador de Helmholtz, a las que podemos sumar soluciones a la ecuación homogénea de modo que la suma satisfaga las condiciones de borde del problema específico que podamos estar desarrollando.

\subsection{Ecuación de Laplace}

En este caso, nuestro operador diferencial corresponde a $p(x) = 1$ y $q(x) = 0$ en la definición \eqref{eq:edp_general}. Dado que este curso se centra en métodos útiles en problemas físicos, consideraremos siempre la función de Green que respete la \emph{homogeneidad e isotropía del espacio}, es decir, que la función de Green dependa únicamente de la \emph{diferencia} entre $\x$ y $\x'$ (homogeneidad), y en particular, dependa de la \emph{distancia} $|\x - \x'|$ (isotropía, o invariancia bajo rotaciones). Por ello, la función de Green en este caso corresponde a una función de una variable, que podemos llamar $r = |\x - \x'|$.

\subsubsection{Caso tridimensional}

En un sistema de coordenadas esféricas centradas en $\x'$, nuestra función de Green satisface
\begin{equation}
    \nabla^2 G(r) = \frac{1}{r^2} \frac{d}{dr} \left( r^2 \frac{dG}{dr} \right) = \delta^{(3)}(r) \ ,
\end{equation}
de modo que, si $r \neq 0$, 
\begin{equation}
    \frac{1}{r^2} \frac{d}{dr} \left( r^2 \frac{dG}{dr} \right) = 0 \ .
\end{equation}
Integrando esta expresión, hallamos que
\begin{equation}
    G(r) = \alpha + \frac{\beta}{r} \ ,
\end{equation}
donde $\alpha$ y $\beta$ son coeficientes a determinar. Podemos hacerlo utilizando el teorema de Gauss en la condición de consistencia \eqref{eq:condicion_consistencia}, de modo que
\begin{equation}
    \int_{\partial V} \nabla G \cdot d\vec{S} = \int_{\partial V} \frac{dG}{dr} r^2 d\Omega = 1 \ ,
\end{equation}
donde $\Omega$ corresponde a un ángulo sólido. Integrando, tenemos que
\begin{equation}
    \beta = - \frac{1}{4\pi} \ .
\end{equation}

Por otro lado, en la mayoría de los problemas es conveniente escoger una función de Green que se anule a grandes distancias ($\lim_{r\to \infty} G(r) = 0)$. Por ello, es común escoger $\alpha = 0$. Es más, podemos argumentar que una solución constante $H(r) = - \alpha$ corresponde a una solución al problema homogéneo, que podemos sumar a nuestra función de Green anterior, obteniendo así la solución fundamental para la ecuación de Laplace,
\begin{equation} \marginnote{Solución fundamental de la ecuación de Laplace en 3D}
    \boxed{G(\x - \x') = - \frac{1}{4\pi} \frac{1}{|\x - \x'|} \ .}
\end{equation}

\subsubsection{Caso bidimensional}

En un sistema de coordenadas polares centradas en $\vec{x}'$, nuestra función de Green satisface
\begin{equation}
    \nabla^2 G = \frac{1}{\rho} \frac{d}{d\rho} \left( \rho \frac{dG}{d\rho} \right) = \delta^{(2)}(\rho) \ .
\end{equation}
Integrando para $\rho \neq 0$, tenemos que
\begin{equation}
    G(\rho) = \alpha + \beta \ln \rho \ .
\end{equation}
Utilizando el teorema de Gauss (en dos dimensiones) y la condición de consistencia \eqref{eq:condicion_consistencia}, 
\begin{equation}
    \int_{\partial S} \nabla G \cdot d\vec{S} = \oint \frac{dG}{d\rho} \rho d\phi = 2\pi \beta = 1 \ , 
\end{equation}
por lo que nuevamente suponiendo $\alpha = 0$, tenemos que 
\begin{equation} \marginnote{Solución fundamental de la ecuación de Laplace en 2D}
    \boxed{G(|\x - \x'|) = \frac{1}{2\pi} \ln |\x - \x'| \ .}
\end{equation}

% \subsubsection{Caso unidimensional}

% En un sistema unidimensional con origen en $\x'$, nuestra función de Green satisface
% \begin{equation}
%     \nabla^2 G = \frac{d^2G}{dx^2} = \delta(x) \ .
% \end{equation}
% Integrando para $x \neq 0$, tenemos que
% \begin{equation}
%     G(x) = \alpha + \beta x \ ,
% \end{equation}
% de modo que aplicando la condición de consistencia, tenemos que
% \begin{equation}
%     \int \frac{dG}{dx} dx
% \end{equation}

\subsection{Ecuación de Helmholtz}

En este caso, nuestro operador diferencial corresponde a $p(x) = 1$ y $q(x) = k^2$ en \eqref{eq:edp_general}, de modo que $\mathcal{L} = \nabla^2 + k^2$. Nuevamente, supondremos homogenenidad e isotropía, de modo que buscaremos funciones de la forma $G(\x, \x') = G(|\x - \x'|)$.

\subsubsection{Caso tridimensional}

En coordenadas esféricas centradas en $\x'$, nuestra función de Green satisface
\begin{equation}
    (\nabla^2 + k^2) G(r) = \frac{1}{r^2} \frac{d}{dr}\left( r^2 \frac{dG}{dr} \right) + k^2 G(r) = \delta^{(3)}(r) \ ,
\end{equation}
de modo que al considerar $r \neq 0$, obtenemos la ecuación
\begin{equation}
    \frac{1}{r^2} \frac{d}{dr}\left( r^2 \frac{dG}{dr} \right) + k^2 G(r) = 0 \ .
\end{equation}

Bajo el cambio de variable $u(r) = r G(r)$, podemos reescribir la ecuación anterior como
\begin{equation}
    \frac{d^2 u}{dr^2} + k^2 u = 0 \ ,
\end{equation}
que corresponde a la ecuación de un oscilador armónico simple. Luego, las soluciones son de la forma
\begin{equation}\label{eq:solucion_green_Helmholtz}
    G(r) = \alpha \frac{e^{ikr}}{r} + \beta \frac{e^{-ikr}}{r} \ ,
\end{equation}
donde nuevamente, $\alpha$ y $\beta$ son coeficientes a determinar. Mediante el teorema de Gauss aplicado a la condición de consistencia \eqref{eq:condicion_consistencia}, sobre una esfera de rario $R$ centrada en $r=0$, de modo que
\begin{equation}
    \oint_{\partial V}  \nabla G \cdot \ d\vec{S} + k^2 \int_V G \ dV = \oint_{\partial V} \frac{dG}{dr} r^2 \ d\Omega + k^2 \int_V G r^2 \ dr \ d\Omega = 1 \ .
\end{equation}

La primera integral es directa, puesto que el integrando no depende del ángulo sólido $\Omega$. La segunda integral, por otro lado, debe ser resuelta con más cuidado. Tenemos que
\begin{align*}
    1 & = 4\pi R^2 \left. \frac{dG}{dr}\right|_R + 4\pi k^2 \int_0^R G r^2 \ dr \nonumber \\
    & = 4\pi R^2 \left[\frac{1}{R} (\alpha ik e^{ikR} - \beta ik e^{-ikR}) - \frac{1}{R^2} (\alpha e^{ikR} + \beta e^{-ikR}) \right] + 4\pi k^2 \int_0^R \left[ \alpha e^{ikr} + \beta e^{-ikr} \right] r \ dr \nonumber \\
    & = 4\pi \left[ \alpha (ikR-1)e^{ikR} - \beta (1+ikR)e^{-ikR} + \alpha (1-ikR)e^{ikR} + \beta (1+ikR) e^{-ikR} - (\alpha + \beta) \right] \nonumber \\
    & = -4\pi (\alpha + \beta) \ ,
\end{align*}
de modo que nuestros coeficientes deben satisfacer
\begin{equation}
    \alpha + \beta = - \frac{1}{4\pi} \ .
\end{equation}
Luego, podemos escribir la solución \eqref{eq:solucion_green_Helmholtz} en términos del coeficiente $\beta$ como
\begin{equation}
    G(r) = - \frac{1}{4\pi} \frac{e^{ikr}}{r} - 2i\beta \frac{\sin(kr)}{r} \ ,
\end{equation}
donde el segundo término corresponde a una solución de la ecuación de Helmholtz homogénea, pues es proporcional a la función esférica de Bessel $j_0(kr)$. Luego, considerando $\beta = 0$ por este motivo, es posible escoger la solución fundamental de la ecuación de Helmholtz como
\begin{equation} \marginnote{Solución fundamental de la ecuación de Helmholtz en 3D}
    \boxed{G(\x - \x') = - \frac{1}{4\pi} \frac{e^{ik|\x - \x'|}}{|\x - \x'|} \ .}
\end{equation}

\begin{obs}{Observación}
    Una elección igual de válida habría sido escribir la ecuación \eqref{eq:solucion_green_Helmholtz} en términos de $\alpha$, donde de igual forma obtendríamos un término proporcional a la función esférica de Bessel $j_0(kr)$. Luego, escogiendo $\alpha = 0$, obtendríamos la solución
    \begin{equation}
        \boxed{G_2(\x - \x') = - \frac{1}{4\pi} \frac{e^{-ik|\x - \x'|}}{|\x - \x'|} \ .}
    \end{equation}
    
    Dependerá de la fuente que estén consultando cuál de las dos soluciones se prefiere.
\end{obs}

\subsubsection{Caso bidimensional}

En un sistema de coordenadas polares centrado en $\x'$, nuestra función de Green satisface
\begin{equation}
    (\nabla^2 + k^2)G(\rho) = \frac{1}{\rho} \frac{d}{d\rho} \left( \rho \frac{dG}{d\rho} \right) + k^2 G(\rho) = \delta^{(2)}(\rho) \ ,
\end{equation}
que para $\rho \neq 0$ corresponde a la ecuación
\begin{equation}
    \frac{1}{\rho} \frac{d}{d\rho} \left( \rho \frac{dG}{d\rho} \right) + k^2 G = 0 \ .
\end{equation}

Bajo el cambio de variable $x = k\rho$, esta se reduce a la ecuación de Bessel de orden 0,
\begin{equation}
    x \frac{d}{dx}\left( x \frac{dG}{dx} \right)  + x^2 G(x) = 0 \ ,
\end{equation}
con soluciones de la forma
\begin{equation}\label{eq:solucion_green_Helmholtz_2d}
    G(\rho) = \alpha J_0(k\rho) + \beta Y_0(k\rho) = \tilde{\alpha} H_0^{(1)}(k\rho) + \tilde{\beta} H_0^{(2)}(k\rho) \ .
\end{equation}
Preferiremos utilizar la solución en términos de funciones de Hankel, pues es la elección habitual.

Integrando la ecuación original sobre un círculo $S$ de radio $R$ centrado en $\rho = 0$, obtenemos
\begin{equation}
    \oint_{\partial S} \nabla G \cdot d\vec{S} + k^2 \int_S G \ dS = \oint \left. \frac{dG}{d\rho}\right|_R R \ d\phi + 2\pi \int_0^R G \rho \ d\rho = 1 \ .
\end{equation}

Nuevamente, la primera integral se calcula directamente, ya que no existe dependencia de $\phi$, mientras que la segunda necesita un tratamiento más delicado. Recordemos que las funciones de Hankel, al ser funciones de Bessel, satisfacen las identidades $H_0^{\prime \, (1)}(x) = - H_1^{(1)}(x)$ y $x H_0^{(1)}(x) = [xH_1^{(1)}(x)]'$. Entonces, tenemos que
\begin{align*}
    1 & = 2\pi \left[ kR (\tilde{\alpha} H_0^{\prime \, (1)}(kR) + \tilde{\beta} H_0^{\prime \, (2)}(kR))  + \int_0^{kR} \left(\tilde{\alpha} H_0^{(1)}(x) + \tilde{\beta} H_0^{(2)}(x) \right) x \ dx \right] \\
    & = 2\pi \left[ - kR (\tilde{\alpha} H_1^{(1)}(kR) + \tilde{\beta} H_1^{(2)}(kR)) + \int_0^{kR} \left(\tilde{\alpha} \frac{d}{dx} \left( x H_1^{(1)}(x) \right) + \tilde{\beta} \frac{d}{dx} \left( x H_1^{(2)}(x) \right) \right) \ dx \right] \\
    & = 2\pi \left[ - kR (\tilde{\alpha} H_1^{(1)}(kR) + \tilde{\beta} H_1^{(2)}(kR)) + kR \left(\tilde{\alpha} H_1^{(1)}(kR) + \tilde{\beta} H_1^{(2)}(kR) \right) \right. \\
    & \left. \qquad - \lim_{x \to 0} \left( \tilde{\alpha} x H_1^{(1)}(x) - \tilde{\beta} x H_1^{(2)}(x) \right) \right] \\
    & = 2\pi \left[ \tilde{\alpha} \frac{2i}{\pi} - \tilde{\beta} \frac{2i}{\pi} \right] \\
    & = 4i (\tilde{\alpha} - \tilde{\beta}) \ ,
\end{align*}
de modo que nuestros coeficientes deberán satisfacer la condición
\begin{equation}
    \tilde{\alpha} - \tilde{\beta} = - \frac{i}{4} \ .
\end{equation}
Luego, la solución \eqref{eq:solucion_green_Helmholtz_2d} se puede escribir, en términos de $\tilde{\beta}$, como
\begin{equation}
    - \frac{i}{4} H_0^{(1)}(k\rho) + \tilde{\beta} \left( H_0^{(1)}(k\rho) + H_0^{(2)}(k\rho) \right) = - \frac{i}{4} H_0^{(1)}(k\rho) + 2 \tilde{\beta} J_0(k\rho) \ .
\end{equation}
Al igual que en el caso tridimensional, la solución proporcional a la función de Bessel de orden 0 corresponde a una solución del problema homogéneo, de modo que si $\tilde{\beta = 0}$, la solución fundamental de la ecuación de Helmholtz en 2D corresponde a
\begin{equation} \marginnote{Solución fundamental de la ecuación de Helmholtz en 2D}
    \boxed{G(\x - \x') = - \frac{i}{4} H_0^{(1)}(k|\x - \x'|) \ .}
\end{equation}
\begin{obs}{Observación}
    Al igual que en el caso anterior, la elección de $\tilde{\alpha} = 0$ conduce a una segunda solución posible, la que corresponde a 
    \begin{equation}
        \boxed{G_2(\x - \x') = \frac{i}{4} H_0^{(2)}(k|\x - \x'|) \ .}
    \end{equation}
\end{obs}