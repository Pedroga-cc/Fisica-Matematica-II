\chapter{El Método de Separación de Variables}

Ya observamos que podemos hacer uso del método de la transformada de Fourier para reducir una ecuación diferencial en dos variables diferentes a una EDO en una sola variable. Sin embargo, no siempre será cómodo calcular la transformada de Fourier de una función, por lo que sería agradable tener una forma más general de hacer funcionar esta idea.

Para ello, hacemos uso del \textbf{método de separación de variables}, gracias al cual podemos reducir una EDP lineal de $n$ variables en un conjunto de $n$ EDOs para $n$ funciones auxiliares, cada una asociada a una variable independiente de la EDP. Esto lo podemos hacer al proponer que nuestro sistema puede ser descrito mediante una \emph{solución separable}, que consiste en el producto de todas las funciones auxiliares que encontremos mediante la solución de las EDOs.

Para realizar la separación de las variables, haremos uso de $(n-1)$ \emph{constantes de separación}, las que son escogidas, en primera instancia, de forma arbitraria para luego determinarlas gracias a las condiciones de contorno del sistema.

Finalmente, una vez hallamos las soluciones de cada una de las EDOs y planteamos la solución separable del sistema, consideraremos que la solución más general consiste en la \emph{superposición} o \emph{combinación lineal} de todas las posibles soluciones separables del sistema.

Como ejemplo bastante ilustrativo, durante el capítulo resolveremos la ecuación de Helmholtz en diferentes sistemas coordenados, dando origen a diferentes \emph{funciones especiales}, que serán discutidas en mayor profundidad en los capítulos siguientes.

\section{Resolviendo la ecuación de Helmholtz}

Recordemos que la ecuación de Helmholtz es dada por la expresión
\begin{equation}\label{eq:Helmholtz}
    \nabla^2 \psi + k^2 \psi = 0 \ ,
\end{equation}
donde $k$ es una constante asociada al sistema.

\subsection{Coordenadas cartesianas}

En un sistema cartesiano, el operador laplaciano se define simplemente como
\begin{equation}
    \nabla^2 = \frac{\partial^2}{\partial x^2} + \frac{\partial^2}{\partial y^2} + \frac{\partial^2}{\partial z^2} \ .
\end{equation}

De esta forma, dado que nuestra ecuación posee tres variables independientes, podremos reducirla a un sistema de 3 EDOs en las variables $x$, $y$ y $z$. Antes de realizar este procedimiento, plantearemos una solución de la forma
\begin{equation}\label{eq:ansatz}
    \psi(x,y,z) = X(x)Y(y)Z(z) \ .
\end{equation}

Una duda totalmente razonable es por qué considerar una solución de este estilo. La verdad no hay una respuesta directa a esto, más allá de ``\emph{esperemos que funcione}''. Si no fuera el caso, y nuestro sistema comienza a complicarse, puede que sea una mejor idea utilizar algún método alternativo. Sin embargo, cabe mencionar que si los operadores diferenciales (derivadas $n$-ésimas) son aditivos, es decir, no tenemos combinaciones de las variables, una solución de este estilo suele funcionar.

Evaluando la expresión \eqref{eq:ansatz} en la ecuación \eqref{eq:Helmholtz}, podemos escribirla como
\begin{equation}
    YZ \frac{d^2X}{dx^2} + XZ \frac{d^2 Y}{dy^2} + XY \frac{d^2Z}{dz^2} + k^2 XYZ = 0 \ ,
\end{equation}
donde ahora utilizamos derivadas totales en lugar de parciales, puesto que cada una de las funciones depende únicamente de una variable.

Dividimos ahora por la solución $XYZ$, donde hemos asumido que $\psi(x,y,z) \neq 0$, de modo que, luego de reordenar los términos, la ecuación resulta en
\begin{equation} \label{eq:sep_1}
    \frac{1}{X} \frac{d^2 X}{dx^2} = -k^2 - \frac{1}{Y} \frac{d^2 Y}{dy^2} - \frac{1}{Z} \frac{d^2 Z}{dz^2} \ .
\end{equation}

Hemos llegado al paso en donde se aprecia esta \emph{separación de variables}. Notemos que el lado izquierdo de la ecuación \eqref{eq:sep_1} contiene únicamente términos asociados a la variable $x$, mientras que el lado derecho aún tiene dependencia en $y$ y en $z$. La única posibilidad de que ambos lados sean iguales, dado que dependen de variables distintas, es que ambos son a su vez iguales a una \emph{constante de separación}, que en este caso asumiremos real y llamaremos $\lambda_1$,
\begin{align}
    \frac{1}{X} \frac{d^2 X}{dx^2} = \lambda_1 \ , \\
    -k^2 - \frac{1}{Y} \frac{d^2Y}{dy^2} - \frac{1}{Z} \frac{d^2Z}{dz^2} = \lambda_1 \ . \label{eq:EDO_de_y_z}
\end{align}

Notemos que, reordenando términos en la ecuación \eqref{eq:EDO_de_y_z}, podemos nuevamente separar las variables mediante una constante $\lambda_2$, de modo que hemos \emph{dividido la EDP original, dependiente de tres variables, en un sistema de tres EDOs},
\begin{align}
    \frac{d^2X}{dx^2} - \lambda_1 X & = 0 \ , \label{eq:EDO_de_x}  \\
    \frac{d^2Y}{dy^2} - \lambda_2 Y & = 0 \ , \label{eq:EDO_de_y}  \\
    \frac{d^2Z}{dz^2} + (k^2 + \lambda_1 + \lambda_2) Z & = 0 \ . \label{eq:EDO_de_z} 
\end{align}

Cada una de estas EDOs son resolubles mediante los métodos vistos en su primer curso de ecuaciones diferenciales, y las soluciones dependerán del valor y del signo de las constantes de separación. Analicemos las posibles soluciones de la ecuación \eqref{eq:EDO_de_x}:

\begin{equation}
    X_{\lambda_1}(x) = \left\{
    \begin{array}{cc} 
            c_1 \sinh(\sqrt{\lambda_1} x) + c_2 \cosh(-\sqrt{\lambda_1}x) \ , & \text{si } \lambda_1 > 0 \ , \\
            c_1 + c_2 x \ , & \text{si } \lambda_1 = 0 \ , \\
            c_1 \cos(\sqrt{-\lambda_1}x) + c_2 \sin(\sqrt{-\lambda_1}x) \ , & \text{si } \lambda_1 < 0 \ .
    \end{array}
    \right.
\end{equation}

Aquí es importante no olvidar que tenemos una motivación física para realizar estos cálculos, lo que nos ayudará a determinar el signo de la constante de separación. Para la mayoría de los problemas físicos, la solución para $X(x)$ que tiene sentido es aquella en que $\lambda_1$ es negativo, de modo que la solución es una oscilación en la coordenada $x$. Así mismo, podemos hacer un análisis análogo para cada una de las otras ecuaciones, obteniendo soluciones denotadas por $Y_{\lambda_2}(y)$ y $Z_{\lambda_1 \lambda_2}(z)$, donde es importante indicar el subíndice, ya que esta solución es válida para un valor particular de las constantes de separación.

De esta forma, una solución particular a nuestra EDP es dada por
\begin{equation} \label{eq:sol_particular_cartesianas}
    \psi_{\lambda_1 \lambda_2}(x,y,z) = X_{\lambda_1}(x) Y_{\lambda_2}(y) Z_{\lambda_1 \lambda_2}(z) \ ,
\end{equation}
y la solución general corresponderá a una combinación lineal de la solución \eqref{eq_sol_particular_cartesianas}, correspondiendo a una suma sobre todos los valores posibles de $\lambda_1$ y $\lambda_2$, es decir,
\begin{equation}
    \psi(x,y,z) = \sum_{\lambda_1 \lambda_2} C_{\lambda_1 \lambda_2} \psi_{\lambda_1 \lambda_2}(x,y,z) \ ,
\end{equation}
donde los coeficientes $C_{\lambda_1 \lambda_2}$ serán obtenidos al imponer las condiciones de contorno del problema, que por lo general nos llevará a un conjunto finito de valores para $\lambda_1$ y $\lambda_2$.

\newpage

\subsection{Coordenadas cilíndricas}

En este caso, el laplaciano debe definirse de una forma diferente, dada por
\begin{equation}
    \nabla^2 = \frac{1}{\rho} \frac{\partial}{\partial \rho} \left( \rho \frac{\partial}{\partial \rho} \right) + \frac{1}{\rho^2} \frac{\partial^2}{\partial \phi^2} + \frac{\partial^2}{\partial z^2} \ , 
\end{equation}
de modo que la ecuación de Helmholtz \eqref{eq:Helmholtz} se podrá escribir como
\begin{equation} \label{eq:Helmholtz_cilindrica}
    \frac{1}{\rho} \frac{\partial}{\partial \rho} \left( \rho \frac{\partial \psi}{\partial \rho} \right) + \frac{1}{\rho^2} \frac{\partial^2 \psi}{\partial \phi^2} + \frac{\partial^2 \psi}{\partial z^2} + k^2 \psi = 0 \ .
\end{equation}

Nuevamente plantearemos una solución que sea producto de tres funciones, cada una dependiente de una de las variables del problema, es decir
\begin{equation}
    \psi(\rho, \phi, z) = P(\rho) \Phi(\phi) Z(z) \ ,
\end{equation}
que al sustituirlas en la expresión \eqref{eq:Helmholtz_cilindrica} resultará en
\begin{equation}
    \frac{\Phi Z}{\rho} \frac{d}{d\rho}\left( \rho \frac{dP}{d\rho} \right) + \frac{PZ}{\rho^2} \frac{d^2 \Phi}{d\phi^2} + P\Phi \frac{d^2Z}{dz^2} + k^2 P\Phi Z = 0 \ ,
\end{equation}
y dividiendo por $P\Phi Z$, junto a un reordenamiento de los términos, llegamos a la expresión
\begin{equation}
    \frac{1}{\rho P} \frac{d}{d\rho} \left( \rho \frac{dP}{d\rho} \right) + \frac{1}{\rho^2 \Phi} \frac{d^2\Phi}{d\phi^2} + k^2 = - \frac{1}{Z} \frac{d^2Z}{dz^2} \ .
\end{equation}
