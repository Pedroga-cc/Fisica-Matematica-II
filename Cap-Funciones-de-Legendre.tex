\chapter{Funciones de Legendre}

Como vimos anteriormente en el capítulo \ref{chap:MSV}, al resolver la ecuación de Helmholtz en coordenadas esféricas, podemos obtener dos ecucaciones, según el valor de la constante de separación $m^2$. 

\section{Ecuación de Legendre}

Analizaremos en primer lugar el caso en que $m=0$, correspondiente a resolver la ecuación de Laplace, donde obtuvimos la EDO
\begin{equation}
    \frac{1}{\sin\theta} \frac{d}{d\theta}\left( \sin\theta \frac{d\Theta}{d\theta} \right) + \lambda \Theta = 0 \ ,
\end{equation}
que al considerar el cambio de variables $x = \cos\theta$, y llamar a $\Theta(\theta) = \Theta(\arccos(x)) = y(x)$, de modo que
\begin{equation}
    \frac{d}{d\theta} = \frac{dx}{d\theta} \frac{d}{dx} = -\sin\theta \frac{d}{dx} \ ,
\end{equation}
con lo que podremos escribir la ecuación como
\begin{equation}
    \frac{1}{\sin\theta}(-\sin\theta)\frac{d}{dx}\left( \sin\theta (-\sin\theta) \frac{dy}{dx} \right) + \lambda y = \frac{d}{dx}\left( (1-x^2) \frac{dy}{dx} \right) + \lambda y = 0 \ ,
\end{equation}
expresión que nombraremos como \textbf{ecuación de Legendre}.

Observamos que hemos escrito la ecuación de Legendre en la forma de una ecuación de Sturm-Liouville, con $p(x) = 1-x^2$ y $q(x) = r(x) = 1$. Observamos que, dado que $x=\cos\theta$, nuestra ecuación está definida en el intervalo $[-1,1]$. Nos interesa que estos casos extremos posean soluciones finitas, lo que nos permitirá encontrar el valor de $\lambda$ para esta ecuación.

\subsection{Resolviendo la ecuación de Legendre}

En primer lugar, escribimos la ecuación de forma más explícita, es decir,
\begin{equation}\label{eq:legendre_explicit}
    (1-x^2)y''(x) - 2x y'(x) + \lambda y(x) = 0 \ .
\end{equation}

Utilizando el método de series, planteamos soluciones de la forma
\begin{equation}
    \begin{dcases}
        y(x) & = \sum_{k=0}^{\infty} a_k x^k \\
        y'(x) & = \sum_{k=1}^{\infty} k a_k x^{k-1} \\
        y''(x) & = \sum_{k=2}^{\infty} (k-1)k a_k x^{k-2} \\
        & = \sum_{k=0}^{\infty} (k+1) (k+2) a_{k-2} x^k 
    \end{dcases} \ ,
\end{equation}
de modo que sustituyendo en la expresión \eqref{eq:legendre_explicit} tenemos 
\begin{align}
    \sum_{k=0}^{\infty} (k+1)(k+2) a_{k+2} x^k - \sum_{k=0}^{\infty} (k-1) k a_k x^k - 2 \sum_{k=1}^{\infty} k a_k x^k + \lambda \sum_{k=0}^{\infty} a_k x^k & = 0 \ , \\
    \sum_{k=0}^{\infty} \left[ (k+1)(k+2) a_{k+2} - k(k-1)a_k  - 2k a_k + \lambda a_k \right]x^k & = 0 \ ,
\end{align}
donde hemos usado que $\sum_{k=1}^{\infty} k = \sum_{k=0}^{\infty} k$. Dado que $x^k = 0$ corresponde a la solución trivial, analizamos la relación entre los coeficientes de nuestra ecuación,
\begin{equation}
    (k+1)(k+2) a_{k+2} - [(k-1) k + 2k - \lambda]a_k = 0 \ ,
\end{equation}
de donde encontramos la siguiente relación de recurrencia,
\begin{equation}\label{eq:recurrencia_legendre}
    a_{k+2} = \frac{k(k+1) - \lambda}{(k+1)(k+2)}a_k \ ,
\end{equation}
de modo que dados unos valores para $a_0$ y $a_1$ podremos determinar todos los coeficientes de nuestra expresión. Luego, si desarrollamos los términos pares e impares por separado, podemos observar la siguiente recurrencia,
\begin{align*}
    a_2 & = \frac{-\lambda}{2} a_0 \\
    a_4 & = \frac{2(2+1) - \lambda}{3 \cdot 4} a_2 = \frac{-\lambda(2(2+1)-\lambda)}{4!}a_0 \\
    a_6 & = \frac{4(4+1) - \lambda}{5 \cdot 6} a_4 = \frac{-\lambda (2(2+1)-\lambda)(4(4+1)-\lambda)}{6!} a_0 \\
    & \qquad \vdots \\
    a_n & = \frac{a_0}{n!}\prod_{\ell=0}^{n-2}[\ell (\ell+1) - \lambda] \ ,
\end{align*}
mientras que para el caso de coeficientes impares,
\begin{align*}
    a_3 & = \frac{1(1+1) - \lambda}{2 \cdot 3}a_1 \\
    a_5 & = \frac{(3(3+1)-\lambda)}{4 \cdot 5}a_3 = \frac{(1(1+1) - \lambda)(3(3+1)-\lambda)}{5!} a_1 \\
    a_7 & = \frac{5(5+1) - \lambda}{6 \cdot 7} = \frac{(1(1+1) - \lambda)(3(3+1)-\lambda)(5(5+1) - \lambda)}{7!} a_1 \\
    & \qquad \vdots \\
    a_n & = \frac{a_1}{n!}\prod_{\ell=1}^{n-2}[\ell (\ell+1) - \lambda] \ ,
\end{align*}
de modo que la solución general para la ecuación de Legendre será la combinación lineal de la solución para los valores pares y la solución para los valores impares,
\begin{align}
    y(x) & = a_0 y_{1}(x) + a_1 y_{2}(x) \\
    & = a_0 \left( \sum_{m=0}^{\infty} b_{m} x^{2m} \right) + a_1 \left( \sum_{m=0}^{\infty} c_{m} x^{2m+1} \right) \\
    & = \sum_{m=0}^{\infty} \frac{a_0}{(2m)!}\left( \prod_{\ell=0}^{2m-2}[\ell (\ell+1) - \lambda] \right)x^{2m} + \sum_{m=0}^{\infty} \frac{a_1}{(2m+1)!}\left( \prod_{\ell=1}^{2m-1}[\ell (\ell+1) - \lambda] \right)x^{2m+1} \ . \label{eq:Legendre_series_solution}
\end{align}

Con esto, podemos darnos cuenta de que la solución general de la ecuación de Legendre es una serie infinita. ¿Cómo imponemos la condición de finitud a estas expresiones? Podemos hacerlo al estudiar su convergencia para el intervalo en que se encuentran definidas.

Según el criterio de la razón (véase el capítulo 7 del apunte de Física Matemática I), la serie converge para $|x|<1$, independientemente del valor de $\lambda$. Nos queda analizar los casos extremos, es decir, $x=\pm1$.

Para $\lambda \neq n(n+1)$, la serie diverge, puesto que ella se convierte en una serie numérica sin ningún tipo de restricción, que crece indefinidamente. Volveremos a este caso más tarde.

Para $\lambda = n (n+1)$, existirá algún término $a_n=0$ para la solución par o la solución impar, y siguiendo la relación de recurrencia \eqref{eq:recurrencia_legendre}, todo término superior a este se anulará. Luego, basta imponer que el primer coeficiente de la solución de paridad opuesta se anule para así obtener una solución convergente. Es decir, si $n$ es par, $a_1=0$, mientras que si $n$ es impar, escogemos $a_0 = 0$.

De esta forma, nos gustaría poder escribir una expresión general para las soluciones que son válidas en todo el dominio, es decir, cuando $\lambda = n(n+1)$. Notamos que $\ell(\ell+1)-n(n+1) = (\ell - n)(\ell + n + 1)$, podemos encontrar que, para $n$ par,
\begin{align} \label{eq:y1n_legendre}
    y_{1,n}(x) & = \sum_{m=0}^{n/2} b_m x^{2m} = \sum_{m=0}^{n/2} \left(\frac{1}{(2m)!} \prod_{\ell=0}^{2m-2} (\ell-n)(\ell+n+1) \right) x^{2m} \\
    & = d_{n,1} P_n(x) = \frac{(-1)^{n/2} 2^n \left[ \left(\frac{n}{2}\right)! \right]^2}{n!}P_n(x) \ , \label{eq:y2n_legendre}
\end{align}
mientras que para $n$ impar,
\begin{align}
    y_{2,n}(x) & = \sum_{m=0}^{(n-1)/2} c_m x^{2m+1} = \sum_{m=0}^{n/2} \left(\frac{1}{(2m)!} \prod_{\ell=0}^{2m-2} (\ell-n)(\ell+n+1) \right) x^{2m+1} \\
    & = d_{n,2} P_n(x) = \frac{(-1)^{(n+1)/2} 2^{n-1} \left[ \left(\frac{n-1}{2}\right)! \right]^2}{n!}P_n(x) \ ,
\end{align}
donde en ambos casos las funciones $P_n(x)$ corresponden a los \textbf{polinomios de Legendre}, definidos como
\begin{equation}
    P_n(x) = \sum_{k=0}^{n/2} \frac{(-1)^k (2n-2k)}{2^n k! (n-k)! (n-2k)!}x^{n-2k} \ .
\end{equation}

Otra forma de definirlos es a partir de la \emph{fórmula de Rodrigues}, según la cual pueden expresarse como
\begin{equation}
    P_n(x) = \frac{1}{2^n n!} \frac{d^n}{dx^n} (x^2-1)^n \ .
\end{equation}

\subsection{Función Generatriz}

\begin{defi}
    Para cualquier conjunto de \emph{polinomios ortogonales} $\{P_n(x)\}_{n\in \mathbb{N}}$, se denomina \textbf{función generatriz} (o generadora) de dicho conjunto a la función $G(x,t)$, tal que cuando se desarrolla en una serie de Taylor para $t$, los coeficientes de dicha expansión son los polinomios $P_n(x)$:
\begin{equation}
    G(x,t) = \sum_{n=0}^{\infty} P_n(x)t^n \ .
\end{equation}
\end{defi}

En particular, la función generatriz para los polinomios de Legendre es dada por la expresión
\begin{equation}
    G(x,t) = \frac{1}{\sqrt{1-2xt + t^2}} \ .
\end{equation}
Si observamos esta expresión en detalle, y hacemos $t=r/a$ y $x = \cos\theta$, podemos escribir la función generatriz como
\begin{equation}
    G(\theta, r) = \frac{1}{\sqrt{1-2r \cos(\theta)/a + r^2/a^2}} = \frac{a}{\sqrt{a^2-2ar\cos\theta + r^2}} = \frac{a}{|\vec{x}-\vec{a}|} \ , 
\end{equation} 
donde $|\vec{x}-\vec{a}|$ es la distancia entre los puntos $\vec{x}$ y $\vec{a}$. Así, obtenemos la relación entre la distancia entre dos puntos del espacio con los Polinomios de Legendre,
\begin{equation}
    \frac{1}{|\vec{x} - \vec{a}|} = \frac{1}{a}\sum_{\ell=0}^\infty \left( \frac{r}{a} \right)^\ell P_\ell(\cos\theta) \ .
\end{equation} 
En esta expresión, hemos asumido que $r>a$. En caso contrario, deben invertirse las relaciones, ubicando $r$ en el lugar de $a$, y viceversa.

Podemos utilizar este resultado, por ejemplo, para escribir el potencial electrostático en un punto $\vec{r}$ producido por una carga $q$ en el punto $\vec{a}$,
\begin{equation}
    \phi(\vec{r}) = \frac{q}{4\pi \epsilon_0 r} \sum_{\ell=0}^\infty \left( \frac{a}{r} \right)^\ell P_\ell(\cos\theta) \ ,
\end{equation}
lo que se asocia a la llamada \emph{expansión multipolar}, que estudiarán en su curso de electrodinámica.

Por otro lado, podemos encontrar explícitamente los polinomios de Legendre a partir de su función generadora mediante
\begin{equation}
    P_n(x) = \frac{1}{n!} \left[ \frac{\partial ^n G(t,x)}{\partial t^n} \right] \left. \right|_{t=0} \ .
\end{equation}

\subsection{Propiedades}

\begin{enumerate}
    \item \textbf{Normalización.} Los polinomios de Legendre son funciones normalizadas, de modo que $P_n(1) = 1$, para cualquier valor de $n$.
    \item \textbf{Paridad.} Podemos encontrar que los polinomios de Legendre pueden ser funciones pares o impares según el valor de $n$, ya que
    \item \begin{equation}
        P_n(-x) = (1-)^n P_n(x) \ . 
    \end{equation}

    \item \textbf{Valor en el origen.} A partir de la función generatriz, podemos mostrar que
    % propiedad de paridad, podemos observar que, como los polinomios son \emph{impares} para $n$ impar, deberá cumplirse que $P_n(0) = 0$, para $n$ impar. Por otro lado, utilizando la función generatriz, podemos obtener que si $n$ es par, entonces
    \begin{equation}
        P_n(0) = \begin{dcases}
            0 \ , \qquad \text{si } n \text{ es impar.} \\
            (-1)^{n/2} \frac{n!}{2^n ((n/2)!)^2} \ , \qquad \text{si } n \text{ es par.}
        \end{dcases}
    \end{equation}

    \item \textbf{Ortogonalidad.} Los polinomios de Legendre satisfacen la relación de ortogonalidad
    \begin{equation}
        \int_{-1}^1 P_n(x) P_m(x) dx = \frac{2}{2n+1} \delta_{n,m} \ .
    \end{equation}

    \item \textbf{Completitud.} Los polinomios de Legendre forman un \emph{conjunto completo} de funciones definidas en $[-1,1]$, por lo que forma una base para dichas funciones, lo que se expresa como
    \begin{equation}
        \sum_{n=0}^\infty \frac{2n+1}{2} P_n(x) P_n(x') = \delta(x-x') \ .
    \end{equation}

    \item \textbf{Serie de Fourier-Legendre.} Dado que los polinomios de Legendre forman un abase en el intervalo $[-1,1]$, podemos expandir cualquier función en una Serie de Fourie-Legendre, tal que

    \begin{equation}
        f(x) = \sum_{n=0}^\infty a_n P_n(x) \ ,
    \end{equation}
    donde 
    \begin{equation}
        a_n = \frac{2n+1}{2} \int_{-1}^1 f(x) P_n(x) dx \ .
    \end{equation}
     
    \item \textbf{Relaciones de recurrencia.} Los polinomios de Legendre satisfacen las siguientes relaciones de recurrencia,
    \begin{align}
        n P_{n-1}(x) + (n+1) P_{n+1}(x) & = (2n+1) P_n(x) \ , \\
        (2n+1) P_n(x) & = P'_{n+1}(x) - P'_{n-1}(x) \ .
    \end{align}    
\end{enumerate}

\subsection{Funciones de Legendre de segunda especie}

Como discutimos anteriormente, al escoger $\lambda = n(n+1)$ para un número entero $n$, debemos truncar una de las series que aparecen en la ecuación \eqref{eq:Legendre_series_solution}, de modo que la solución se reduce a los polinomios de Legendre. El motivo de hacer esto es hallar una solución que converja en $x = \pm 1$. Sin embargo, si estamos estudiando un problema que no requiere convergencia en los contornos $x = \pm 1$, ¿es necesario descartar la segunda serie infinita?

La respuesta es no, y de hecho ciertos problemas físicos requieren de estas soluciones. Por ello, las llamaremos \textbf{funciones de Legendre de segunda especie}, denotadas por $Q_n(x)$, y que definimos como
\begin{align}
    Q_n(x) & = \begin{dcases}
        d_{1,n} y_{2,n}(x) \ , \qquad \text{ si } n \text{ es par} \ , \\
        d_{2,n} y_{1,n}(x) \ , \qquad \text{ si } n \text{ es impar} \ ,
    \end{dcases} \\
    & = \begin{dcases}
        (-1)^{n/2} \frac{[(n/2)!]^2}{n!} 2^n y_{2,n}(x) \ , \qquad \text{ si } n \text{ es par} \ , \\
        (-1)^{(n+1)/2} \frac{[\left(\frac{n-1}{2}\right)!]^2}{n!} 2^{n-1} y_{1,n}(x) \ , \qquad \text{ si } n \text{ es impar} \ ,
    \end{dcases}
\end{align}
donde $y_{1,n}$ e $y_{2,n}$ son las funciones definidas en \eqref{eq:y1n_legendre} y \eqref{eq:y2n_legendre}, respectivamente. Nótese que \emph{a la solución para} $n$ \emph{par, hemos multiplicado los coeficientes de la solución impar}, y viceversa.

Las funciones asociadas de Legendre satisfacen las mismas relaciones de recurrencia que las funciones de Legendre, de modo que
\begin{align}
    nQ_{n-1}(x) + (n+1)Q_{n+1}(x) & = (2n+1) x Q_n(x) \ , \\
    (2n+1)Q_n(x) & = Q'_{n+1}(x) - Q'_{n-1}(x) \ . 
\end{align}

Pueden encontrar la derivación explícita de estas relaciones en el apunte del profesor Rubilar \cite{Rubilar}.

Otras propiedades relebvantes que estas satisfacen son
\begin{enumerate}
    \item \textbf{Paridad.} Se tiene que
    \begin{equation}
        Q_n(-x) = (-1)^{n+1} Q_n(x) \ .
    \end{equation}
    \item \textbf{Valor en el origen.} Dada su paridad, las relaciones se invierten respecto a los polinomios de Legendre, de modo que
    \begin{equation}
        Q_n(0) = \begin{dcases}
            0 \ , \qquad \text{si } n \text{ es par} \ , \\
            (-1)^{(n+1)/2} \frac{\left[ \left( \frac{n-1}{2} \right)! \right]^2}{n!} 2^{n-1} \ , \qquad \text{si } n \text{ es par} \ .  
        \end{dcases}
    \end{equation}
    \item \textbf{Valor en el extremo del intervalo.} Los extremos del intervalo son puntos singulares para estas funciones, de modo que
    \begin{equation}
        \lim_{x \to 1} Q_n(x) = +\infty \ .
    \end{equation}
\end{enumerate} 

En resumen, la solución general de la EDO de Legendre para $\lambda = n(n+1)$, es dada por
\begin{equation}
    y(x) = C_1 P_n(x) + C_2 Q_n(x) \ ,
\end{equation}
donde los polinomios de Legendre $P_n(x)$ convergen en todo el intervalo $[-1,1]$, incluídos los extremos, mientras que las funciones de segunda especie $Q_n(x)$ convergen en el interior del intervalo $(-1,1)$, mas no en los extremos.

\section{Ecuación asociada de Legendre}

Hasta ahora, analizamos el caso en que $m=0$ en la EDO axial \eqref{eq:Helmholtz_esferica} que se obtiene de la ecuación de Helmholtz. En lo que resta del capítulo, analizaremos el caso en que $m \neq 0$.

En este caso, la ecuación se puede escribir como
\begin{equation}
    \frac{1}{\sin\theta} \frac{d}{d\theta}\left( \sin\theta \frac{d\Theta}{d\theta} \right) + \left(\lambda - \frac{m^2}{\sin^2\theta} \right) \Theta = 0 \ .
\end{equation}

Realizando nuestro cambio de variable $x = \cos\theta$, y desarrollando de forma más explícita la ecuación, esta toma la forma de la \textbf{ecuación asociada de Legendre},
\begin{equation}
    (1-x^2) y''(x) - 2xy'(x) + \left( \lambda - \frac{m^2}{1-x^2} \right) y(x) = 0 \ ,
\end{equation}
cuyas soluciones se encuentran definidas en el intervalo $[-1,1]$.

\subsection{Resolviendo la ecuación asociada de Legendre}

Para hacer más fácil el proceso de utilizar el método de series, realizamos la sustitución $y(x) = (1-x^2)^{m/2} u(x)$, y resolvemos para $u(x)$. Esto resulta en la EDO
\begin{equation}
    (1-x^2)u''(x) - 2x(m+1)u'(x) + (\lambda - m(m+1))u(x) = 0 \ ,
\end{equation}
a partir de la cual podemos plantear una solución de la forma
\begin{equation}
    u(x) = \sum_{k=0}^{+\infty} a_k x^k 
\end{equation} 
y encontrar la siguiente relación de recurrencia
\begin{equation}\label{eq:recurrencia_asociadas}
    a_{k+2} = \frac{k^2 + (2m+1)k - \lambda + m(m+1)}{(k+1)(k+2)} a_k \ ,
\end{equation}
que al igual que para la ecuación de Legendre, resultará en series que convergen para $|x|<1$ y, en general, divergen para $|x|=1$, salvo para ciertos valores de $\lambda$. Para encontrar dichos valores, necesitamos que el numerador de la ecuación \eqref{eq:recurrencia_asociadas} sea cero, es decir,
\begin{equation}
    \lambda = m(m+1) + k(k+1) + 2mk = (m+k)(m+k+1) \ ,
\end{equation}
con lo que, definiendo $n = m+k \geq m$, observamos que nuevamente requeriremos que $\lambda = n(n+1)$, de modo que la serie terminará luego del $(n-m)$-ésimo término.

Podemos hallar de forma explícita las soluciones a esta EDO diferenciando repetidamente la ecuación de Legendre \eqref{eq:legendre_explicit}, al hacer uso de la regla de Leibniz, es decir,
\begin{equation}
    \frac{d^m}{dx^m}\left( f_1(x) \cdot f_2(x) \right) = \sum_{s=0}^{m} \binom{m}{s} \frac{d^{m-s}}{dx^{m-s}}f_1(x) \frac{d^s f_2(x)}{dx^s} \ ,
\end{equation}
de donde obtenemos que
\begin{align*}
    \frac{d^m}{dx^m}\left[ (1-x^2)y''(x) \right] & = \binom{m}{m} (1-x^2) \frac{d^m }{dx^m}y'' + \binom{m}{m-1} (-2x) \frac{d^{m-1}}{dx^{m-1}} y'' + \binom{m}{m-2} (-2) \frac{d^{m-2}}{dx^{m-2}}y'' \\
    & = (1-x^2) \left(\frac{d^m y}{dx^m}\right)'' - 2mx \left(\frac{d^m y}{dx^m}\right)' - m(m-1)\left(\frac{d^m y}{dx^m}\right)  \\
    \frac{d^m}{dx^m}\left[ 2x y'(x) \right] & = \binom{m}{m} 2x \frac{d^m}{dx^m}y' + \binom{m}{m-1} 2 \frac{d^{m-1}}{dx^{m-1}}y' \\
    & = 2x \left(\frac{d^2 y}{dx^2}\right)' + 2m \left(\frac{d^2 y}{dx^2}\right) \\
    \frac{d^m}{dx^m}\left[ n(n+1)y'(x) \right] & = n(n+1) \frac{d^m}{dx^m}y(x) \ ,
\end{align*}
donde hemos usando que $\binom{m}{m} = 1$, $\binom{m}{m-1} = m$, y $\binom{m}{m-2} = \frac{m(m-1)}{2}$. Así, la EDO resulta en
\begin{equation}
    (1-x^2)\left(\frac{d^m y}{dx^m}\right)'' - (2mx + 2x) \left(\frac{d^m y}{dx^m}\right)' + [m(m-1) + 2m + n(n+1)]\left(\frac{d^m y}{dx^m}\right) = 0 \ ,
\end{equation}
donde al hacer la sustitución $f(x) = y^{(m)}(x)$, obtenemos la EDO para $u(x)$ cuando $\lambda = n(n+1)$. De esta forma, concluímos que las funciones $u(x)$ son proporcionales a $\frac{d^m }{dx^m} P_n(x)$.

De forma convencional (por motivos de normalización), las soluciones $u(x)$ corresponden a los \textbf{polinomios asociadas de Legendre}, expresados como
\begin{equation} \label{eq:asociadas_legendre}
    P_n^m(x) = (1-x^2)^{m/2} \frac{d^m}{dx^m}P_n(x) \ .
\end{equation}

Observamos que cuando $m=0$, recuperamos los polinomios de Legendre, de modo que $P^0_n(x) \equiv P_n(x)$. Sustituyendo la fórmula de Rodrigues para $P_n(x)$ en \eqref{eq:asociadas_legendre}, obtenemos la \emph{fórmula de Rodrigues para los polinomios asociados de Legendre},
\begin{equation} \label{eq:rodrigues-asociados}
    P_n^m(x) = \frac{1}{2^n n!}(1-x^2)^{m/2} \frac{d^{n+m}}{dx^{n+m}}(x^2-1)^n \ ,
\end{equation}
que también es válida para $m<0$, siempre y cuando $|m|\leq n$.

\subsection{Función generatriz}

En este caso, podemos hacer uso de la relación entre los polinomios asociados de Legendre y los polinomios de Legendre para obtener la función generatriz de los primeros. En efecto, observamos que derivando $m$ veces la función generatriz de los polinomios de Legendre respecto a $x$, tenemos
\begin{equation}
    \frac{d^m G}{dx^m} = \frac{d^m}{dx^m}(1-2xt+t^2)^{-1/2} = \sum_{n=0}^\infty \frac{d^m}{dx^m}P_n(x) t^n \ ,
\end{equation}
y multiplicando ambos lados por $(1-x^2)^{m/2}$, tenemos que
\begin{equation}
    \frac{d^m G}{dx^m} = (1-x^2)^{m/2} \frac{d^m}{dx^m}(1-2xt+t^2)^{-1/2} = \sum_{n=0}^\infty P^m_n(x) t^n \ .
\end{equation}

Derivando el lado izquierdo de la ecuación, obtenemos
\begin{equation}
    \frac{1 \cdot 3 \cdot 5 \dots (2m-1) (1-x^2)^{m/2} t^m}{(1-2t+t^2)^{m+1/2}} = \sum_{n=0}^\infty P_n^m t^n \ .
\end{equation}
Dividimos la expresión por $t^m$, y definimos $r=n-m$, y además notamos que
\begin{equation}
    1 \cdot 3 \cdot 5 \dots (2r-1) = \frac{1 \cdot 2 \cdot 3 \dots 2r}{2 \cdot 4 \cdot 6 \dots 2r} = \frac{(2r)!}{2^r r!} \ .
\end{equation}

De este modo, podemos hallar que
\begin{equation}
    G_m(x,t) = \frac{(2m)! (1-x^2)^{m/2}}{2^m m! (1-2xt+t^2)^{m+1/2}} = \sum_{r=0}^\infty P_{r+m}^m(x) t^r \ .
\end{equation}

\subsection{Propiedades}

A partir de la expresión \eqref{eq:asociadas_legendre}, es posible mostrar, aplicando la fórmula de Leibniz al producto $(n+1)^n (x-1)^n$, que
\begin{enumerate}[series=asociadas]
    \item \textbf{Simetría.} Los polinomios $P^{-m}_n$ se pueden obtener a partir de los $P^{m}_n$ siguiendo la relación
    \begin{equation}
        P^{-m}_n(x) = (-1)m \frac{(n-m)!}{(n+m)!}P^{m}_n(x)
    \end{equation}
\end{enumerate}

A partir de la fórmula de Rodrigues \eqref{eq:rodrigues-asociados}, se pueden demostrar las siguientes propiedades,
\begin{enumerate}[resume=asociadas]
    \item Evaluados en los extremos del intervalo $x= \pm 1$, los polinomios asociados de Legendre son nulos, salvo cuando $m=0$.
    \begin{equation}
        P_n^m(\pm 1) = \begin{dcases}
            (\pm 1)^n \ , & \qquad m = 0 \\
            0 \ , & \qquad m \neq 0
        \end{dcases} \ .
    \end{equation}

    \item \textbf{Paridad.} A partir de la paridad de los polinomios de Legendre $P_n(x)$, podemos encontrar que los polinomios asociados pueden ser tanto funciones pares como impares, dependiendo del valor de $m$ y $n$,
    \begin{equation}
        P_n^m(-x) = (-1)^{n+m} P_n^m(x) \ .
    \end{equation}

    \item \textbf{Ortogonalidad.} Los polinomios asociados de Legendre satisfacen las siguientes relaciones de ortogonalidad,
    \begin{equation}
        \int_{-1}^1 P_\ell^m(x) P_k^m(x) dx = \frac{2}{2\ell + 1} \frac{(\ell + m)!}{(\ell - m)!} \delta_{\ell, k} \ .
    \end{equation}

    \item \textbf{Relaciones de recurrencia.} Los polinomios asociados de Legendre satisfacen las siguientes relaciones de recurrencia,
    \begin{align}
        (2n+1)x P_n^m(x) & = (n+m) P^m_{n-1}(x) + (n-m+1) P^m_{n+1}(x) \\
        (2n+1) \sqrt{1-x^2} P_n^m(x) & = P^{m+1}_{m+1}(x) - P^{m+1}_{n-1}(x) \\
        & = (n+m)(n+m-1)P^{m-1}_{n-1} - (n-m+1)(n-m+2)P^{m-1}_{n+1} 
    \end{align}

    \item En general, la EDO asociada de Legendre también posee un segundo conjunto de soluciones, no analíticas en $x = \pm 1$, que corresponden a las \emph{funciones asociadas de Legendre de segunda especie}, que se obtienen a partir la ecuación \eqref{eq:asociadas_legendre}, utilizando las funciones de Legendre de segunda especie $Q_n(X)$ como derivando. Por ello, la solución general será dada por
    \begin{equation}
        y(x) = C_1 P_n^m(x) + C_2 Q_n^m(x) \ .
    \end{equation}

    \item La mayoría de los problemas físicos descartan esta segunda solución, ya que en general esperaremos que nuestra solución sea finita en $x = \pm 1$, es decir, en $\theta = 0$ y en $\theta = \pi$, con lo que hacemos $C_2 = 0$.
\end{enumerate}



\section{Armónicos Esféricos}

Como último tema de este capítulo, analizaremos qué ocurre al resolver la ecuación de Laplace, es decir, el caso en que $k=0$ en la ecuación de Helmholtz. Mediante el método de separación de variables, llegamos a las siguientes ecuaciones
\begin{align}
    \frac{d^2 \Phi}{d\phi^2} + m^2 \Phi & = 0 \label{eq:edo_laplace_phi} \\
    \frac{1}{\sin \theta} \frac{d}{d\theta}\left( \sin\theta \frac{d\Theta}{d\theta} \right) + \left( \ell(\ell+1) - \frac{m^2}{\sin^2\theta} \right)\Theta & = 0 \label{eq:edo_laplace_theta} \\
    \frac{d}{dr}\left( r^2 \frac{dR}{dr} \right) - \ell(\ell+1) R & = 0 \ , \label{eq:edo_laplace_r}
\end{align}
donde la elección de la constante de separación $\ell(\ell+1)$ proviene de exigir una solución analítica cuando $\cos\theta = \pm 1$. Luego, la solución de la ecuación \eqref{eq:edo_laplace_phi} es oscilante, la solución a la ecuación \eqref{eq:edo_laplace_theta} son los polinomios asociados de Legendre, y la solución para \eqref{eq:edo_laplace_r} es $R_\ell(r) = D_\ell r^\ell + E_\ell r^{-(\ell+1)}$.

Una solución para la ecuación de Laplace en coordenadas esféricas será, por ende,
\begin{align}
    \psi_{m\ell}(r,\theta,\phi) & = R_\ell(r) \Theta_{m\ell}(\theta) \Phi_m (\phi) \nonumber \\
    & = (D_\ell r^\ell + E_\ell r^{-(\ell+1)}) C_{m\ell} P_\ell^m(\cos\theta)(A_m \cos(m\phi) + B_m \sin(m\phi)) \ ,
\end{align}
o bien,
\begin{equation}
    \psi_{m\ell}(r,\theta,\phi) = R_\ell(r) P_\ell^m(\cos\theta) C_\ell e^{im\phi} \ ,
\end{equation}
donde se cumplirá que $- \ell \leq m \leq \ell$.

Comúnmente, nos podemos referir a la parte angular de la solución general como \textbf{funciones armónicas esféricas}, o simplemente, \textbf{Armónicos Esféricos}, los que se definen con una normalización conveniente como
\begin{equation}
    Y_\ell^m(\theta, \phi) = (-1)^m \sqrt{\frac{2\ell + 1}{4\pi} \frac{(\ell-m)!}{(\ell+m)!} } P_\ell^m (\cos\theta) e^{im\phi} \ .
\end{equation}

El factor radical tiene su origen en la normalización de los armónicos esféricos, mientras que el factor $(-1)^m$ es convencional al trabajar en mecánica cuántica en el contexto del momento angular. Esta fase es conocida como \emph{fase de Condon-Shortley}, y a veces es introducida en la definición de los polinomios asociados de Legendre.

\subsection{Propiedades}

\begin{enumerate}
    \item \textbf{Simetría.} Los armónicos esféricos $Y_\ell^{-m}$ se pueden obtener según la relación
    \begin{equation}
        Y_\ell^{-m}(\theta, \phi) = (-1)^m (Y_\ell^m (\theta, \phi))^\ast \ .
    \end{equation}

    \item \textbf{Paridad.} Los armónicos esféricos pueden ser pares o impares, según el valor de $\ell$,
    \begin{equation}
        Y_\ell^m(\pi - \theta, \pi + \phi) = (-1)^\ell Y_\ell^m(\theta, \phi) \ .
    \end{equation}

    \item \textbf{Ortonormalidad.} Los armónicos esféricos forman un conjunto ortonormal, es decir,
    \begin{equation}
        \int_0^{2\pi} \int_0^\pi Y_j^k(\theta, \phi) Y_\ell^m (\theta, \phi) \sin\theta d\theta d\phi = \delta_{j, \ell} \delta_{k, m} \ .
    \end{equation}

    \item \textbf{Completitud.} EL conjunto de funciones $\{ Y_\ell^m(\theta, \phi) \}_{\ell = 0, m = -\ell}^{\ell = \infty, m = \ell}$ es un \emph{conjunto completo de funciones}, con lo que \emph{forma una base en el espacio de funciones}, lo que se expresa como
    \begin{equation}
        \sum_{\ell = 0}^{\infty} \sum_{m = -\ell}^{\ell} (Y_\ell^m (\theta, \phi))^\ast Y_{\ell}^m (\theta', \phi') = \delta(\phi - \phi') \delta(\cos\theta - \cos\theta') \ .
    \end{equation}
    
    \item \textbf{Expansión en términos de Armónicos Esféricos.} Dada la completitud de los armónicos esféricos, cualquier función $f(\theta, \phi)$ cuadrado integrable, es decir, que satisface
    \begin{equation}
        \int | f(\theta, \phi) |^2 d\Omega < \infty \ ,
    \end{equation}
    puede ser desarrollada en términos de funciones armónicas esféricas,
    \begin{equation}
        f(\theta, \phi) = \sum_{\ell = 0}^{\infty} \sum_{m = -\ell}^{\ell} a_{\ell m} Y_\ell^m (\theta, \phi) \ ,
    \end{equation}
    donde los coeficientes $a_{\ell m}$ son dados por
    \begin{equation}
        a_{\ell m} = \int_{0}^{2\pi} \int_{0}^{\pi} f(\theta, \phi) (Y_\ell^m)^\ast (\theta, \phi) \sin\theta d\theta d\phi \ .
    \end{equation}

    \item \textbf{Teorema de adición de armónicos esféricos.} Dados dos vectores unitarios $\hat{r}_1$ y $\hat{r}_2$, con direcciones $(\theta_1, \phi_1)$ y $(\theta_2, \phi_2)$, podemos obtener el ángulo $\gamma$ formado entre ambos vectores como
    \begin{equation}
        \hat{r}_1 \cdot \hat{r}_2 = \cos\gamma = \sin\theta_1 \sin\theta_2 \cos(\phi_1 - \phi_2) + \cos\phi_1 \cos\phi_2 \ ,
    \end{equation}
    y podremos hallar los polinomios de Legendre de este ángulo como
    \begin{equation}
        P_\ell(\cos\gamma) = \frac{4\pi}{2\ell + 1} \sum_{m = -\ell}^\ell Y_\ell^m(\theta_1, \phi_1) (Y_\ell^m(\theta_2, \phi_2))^\ast \ .
    \end{equation}

    En particular, cuando $\gamma=0$, y por ende $\theta_1 = \theta_2 = \theta$ y $\phi_1 = \phi_2 = \phi$, obtenemos que
    \begin{equation}
        P_\ell(1) = 1 = \frac{4\pi}{2\ell + 1} \sum_{m = -\ell}^\ell |Y_\ell^m(\theta, \phi)|^2 \ ,
    \end{equation}
    y así,
    \begin{equation}
        \sum_{m = -\ell}^\ell |Y_\ell^m(\theta, \phi)|^2 = \frac{2\ell + 1}{4\pi} \ .
    \end{equation}

\end{enumerate}