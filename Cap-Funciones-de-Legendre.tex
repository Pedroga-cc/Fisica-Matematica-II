\chapter{Funciones de Legendre}

Como vimos anteriormente en el capítulo \ref{chap:MSV}, al resolver la ecuación de Helmholtz en coordenadas esféricas, podemos obtener dos ecucaciones, según el valor de la constante de separación $m^2$. 

Analizaremos en primer lugar el caso en que $m=0$, donde obtenemos la ecuación
\begin{equation}
    \frac{1}{\sin\theta} \frac{d}{d\theta}\left( \sin\theta \frac{d\Theta}{d\theta} \right) + \lambda \Theta = 0 \ .
\end{equation}

Por comodidad, consideremos el cambio de variables $x = \cos\theta$, y llamaremos a $\Theta(\theta) = \Theta(\arccos(x)) = Y(x)$, de modo que
\begin{equation}
    \frac{d}{d\theta} = \frac{dx}{d\theta} \frac{d}{dx} = -\sin\theta \frac{d}{dx} \ ,
\end{equation}
con lo que podremos escribir la ecuación como
\begin{equation}
    \frac{1}{\sin\theta}(-\sin\theta)\frac{d}{dx}\left( \sin\theta (-\sin\theta) \frac{dY}{dx} \right) + \lambda Y = \frac{d}{dx}\left( (1-x^2) \frac{dY}{dx} \right) + \lambda Y = 0 \ ,
\end{equation}
expresión que nombraremos como \textbf{ecuación de Legendre}.