\documentclass[letterpaper,12pt]{report}
\usepackage[spanish]{babel}
\usepackage[utf8]{inputenc}
\usepackage{graphicx}
\usepackage{amsfonts,amsmath,color,amssymb,float, amsthm,mathrsfs}  
\usepackage[right=3cm,left=2cm,top=2cm,bottom=2cm,headsep=0.7cm,footskip=0.5cm]{geometry}
\usepackage{enumerate}
\usepackage{wrapfig} 
\usepackage[rflt]{floatflt} 
\usepackage{framed}
\usepackage[most]{tcolorbox}
\usepackage{xcolor} 
\colorlet{shadecolor}{green!20}
\setlength\FrameSep{0.5ex}
\usepackage{thmtools}
\usepackage{esint}
\usepackage{cancel}
\usepackage{listings} 
\usepackage{pstricks, caption}
\usepackage[colorlinks]{hyperref}
\usepackage{csquotes}
\usepackage{fullpage}
\usepackage{enumitem}
\usepackage{etoolbox}
\usepackage{tcolorbox}
\usepackage{tikz}
\usepackage{multicol}
\usetikzlibrary{arrows,babel}
\usepackage[font=small]{caption}

\decimalpoint
\newcommand{\grad}{^\circ}
\newlength{\drop}
\DeclareMathOperator{\sign}{sgn}
\DeclareMathOperator{\Log}{Log}
\providecommand{\norm}[1]{\lVert#1\rVert}
\DeclareMathOperator{\real}{Re}
\DeclareMathOperator{\im}{Im}

%%%%%%%%%Entornos: Teoremas, Defs,etc%%%%%%%
\theoremstyle{definition}

\newtheorem{defis}{Definiciones}[section]
\newtheorem{corolario}{Corolario}[section] 

\newtheorem{protoexample}{Ejemplo}[section]
\newenvironment{ejemplo}
   {\colorlet{shadecolor}{orange!15}\begin{shaded}\begin{protoexample}}
   {\end{protoexample}\end{shaded}}

\newtheorem*{protoproof}{Demostración}
\newenvironment{demo}
   {\colorlet{shadecolor}{blue!10}\begin{shaded}\begin{protoproof}}
   {\end{protoproof} \end{shaded}}

\newenvironment{Rdbleftbar}{%
  \def\FrameCommand{\textcolor{red}{\vrule width0.6pt}\hspace{0.15em}\textcolor{red}{\vrule width0.6pt} \hspace{0.5em}}%
  \MakeFramed {\advance\hsize-\width \FrameRestore}}%
 {\endMakeFramed}

 \newenvironment{Bdbleftbar}{%
  \def\FrameCommand{\vrule width0.6pt\hspace{0.15em}\vrule width0.6pt \hspace{0.5em}}%
  \MakeFramed {\advance\hsize-\width \FrameRestore}}%
 {\endMakeFramed}

\newenvironment{Gdbleftbar}{%
  \def\FrameCommand{\textcolor{green}{\vrule width0.6pt}\hspace{0.15em}\textcolor{green}{\vrule width0.6pt} \hspace{0.5em}}%
  \MakeFramed {\advance\hsize-\width \FrameRestore}}%
 {\endMakeFramed}
 
% With parskip

\usepackage[indent=0.6cm]{parskip}

\declaretheorem[%
name            =Teorema,%
numberwithin=chapter,
postheadhook    =\vspace*{\parskip} \begin{Rdbleftbar}\vspace*{-5pt},%
prefoothook     =\vspace*{-1pt}\end{Rdbleftbar}%
]{teorema}

\declaretheorem[%
name            =Definición,%
numberwithin=chapter,
postheadhook    =\vspace*{\parskip}\begin{Bdbleftbar}\vspace*{-5pt},%
prefoothook     =\vspace*{-1pt}\end{Bdbleftbar}%
]{defi}

\declaretheorem[%
name            = Proposición,%
numberwithin=chapter,
postheadhook    =\vspace*{\parskip}\begin{Gdbleftbar}\vspace*{-5pt},%
prefoothook     =\vspace*{-1pt}\end{Gdbleftbar}%
]{propo}  

\declaretheorem[%
name            = Lema,%
numberwithin=chapter,
postheadhook    =\vspace*{\parskip}\begin{Gdbleftbar}\vspace*{-5pt},%
prefoothook     =\vspace*{-1pt}\end{Gdbleftbar}%
]{lema}   
%%%%%%%%%%%%%%%%%%%%%%%%%%%%%%%%%%%%%%%%%%%%%%

%%%%%%%%%%%%Fancy Chapters%%%%%%%%%%%%%%%%%
%Options: Sonny, Lenny, Glenn, Conny, Rejne, Bjarne, Bjornstrup
\usepackage[Lenny]{fncychap}
%%%%%%%%%%%%%%%%%%%%%%%%%%%%%%%%%%%%%%%%%%%

%Referencias
\usepackage[backend=biber]{biblatex}
\addbibresource{Referencias.bib}

\begin{document}

\begin{titlepage}
 \drop=0.1\textheight
    \centering
    \vspace*{\baselineskip}
    \rule{\textwidth}{1.6pt}\vspace*{-\baselineskip}\vspace*{2pt}
    \rule{\textwidth}{0.4pt}\\[\baselineskip]
    {\scshape\Huge Física Matemática II} \\[0.2\baselineskip]
    \rule{\textwidth}{0.4pt}\vspace*{-\baselineskip}\vspace{3.2pt}
    \rule{\textwidth}{1.6pt}\\[\baselineskip]
    {\Large Autor: \par}
{\large Pedro A. Contreras-Corral \par}
\vfill
{\large Enero 2023 \par}
{\large v1.0 \par}
\end{titlepage}

\chapter*{Prefacio}

Este documento ha sido preparado por Pedro Contreras Corral como material de apoyo para el curso Física Matemática II. Se han utilizado como base las notas de clase de las ocasiones en que el curso fue dictado por Guillermo Rubilar, Ariana Muñoz e Ivana Sebestova, junto al apunte preparado por Alejandro Saavedra.

\tableofcontents

\chapter{Un breve repaso: Espacios vectoriales y espacio de funciones}

\section{Definiciones}

\begin{defi}
    Un \emph{espacio vectorial} sobre un cuerpo $\mathbb{K}$ es una terna $(V, +, \cdot)$ formada por un \textbf{conjunto no vacío} $V$, una \textbf{operación de suma} $+: V \times V \to V$ y una \textbf{operación de producto escalar} $\cdot: \mathbb{K} \times V \to V$, que satisface ocho propiedades: 

    \begin{multicols}{2}   
        \begin{itemize}
            \item[\textbf{EV1}] (Asociatividad) $\vec{x} + (\vec{y} + \vec{z}) = (\vec{x} + \vec{y}) + \vec{z}$. 
            \item[\textbf{EV2}] (Elemento neutro) $\vec{x} + \vec{0} = \vec{0} + \vec{x} = \vec{x}$.
            \item[\textbf{EV3}] (Elemento opuesto) $\vec{x} + (-\vec{x}) = \vec{0}$.
            \item[\textbf{EV4}] (Conmutatividad) $\vec{x} + \vec{y} = \vec{y} + \vec{x}$. 
            \item[\textbf{EV5}] $a(b\vec{x}) = (ab) \vec{x}$.
            \item[\textbf{EV6}] (Distributividad) $a(\vec{x} + \vec{y}) = a \vec{x} + a\vec{y}$.
            \item[\textbf{EV7}] (Distributividad) $(a+b) \vec{x} = a\vec{x} + b\vec{x}$.
            \item[\textbf{EV8}] $1\vec{x} = \vec{x}$.    
        \end{itemize}
    \end{multicols}
\end{defi}

% \begin{defi}
Denotemos por $\mathscr{C}_0 [a,b]$ al conjunto de funciones complejas continuas de una variable real $t \in [a,b]$.
% \end{defi}

Notemos que claramente se cumple:
\begin{shaded}
$$f,g \in \mathscr{C}_0[a,b] ~\Rightarrow~ f + g \in \mathscr{C}_0 [a,b]$$
\end{shaded}

y 
\begin{shaded}
$$f \in \mathscr{C}_0[a,b] ~\mbox{y}~ \lambda \in \mathbb{C} ~\Rightarrow~ \lambda f \in  \mathscr{C}_0[a,b].$$ 
\end{shaded}

\begin{defi}
Una función $f: [a,b] \longrightarrow \mathbb{C}$ es \textbf{seccionalmente continua} si $[a,b]$ tiene una partición finita $a = t_0 < t_1 < \cdots < t_n = b$ tal que $f$ es continua y acotada en cada intervalo abierto $(t_i, t_{i+1}), i = 0, \dots, n-1$.

Denotaremos por $\mathscr{C}[a,b]$ al conjunto de las funciones complejas seccionalmente continuas.
\end{defi}

Geométricamente, $f: [a,b] \longrightarrow \mathbb{C}$ es una curva en el plano complejo y la condición de seccionalmente continua se puede apreciar en la figura \ref{fig:FunciónSC}. 

\begin{figure}
    \centering
    \includegraphics[scale=0.45]{Figuras/FuncionSC.pdf}
    \caption{En (a) una función de la forma $f: [a,b] \rightarrow \mathbb{R}$ y en (b) una función de la forma $f: [a,b] \rightarrow \mathbb{C}$, ambas seccionalmente continuas.}
    \label{fig:FunciónSC}
\end{figure}

Podemos afirmar que los conjuntos $\mathscr{C}_0 [a,b]$ y $\mathscr{C}[a,b]$ forman espacios vectoriales sobre el cuerpo de los complejos. Además, $\mathscr{C}_0 [a,b] \subset \mathscr{C} [a,b].$

\begin{defi}
Consideremos dos funciones $f,g \in \mathscr{C}[a,b]$. Definimos su \textbf{producto escalar}\footnote{Aquí utilizaremos la notación comúnmente utilizada en matemática para el producto escalar, donde el segundo término del producto es el que se conjuga, en este caso, $g$. Comúnmente en física, al usar notación de Dirac, se utiliza que el elemento conjugado es el primero, en este caso, $f$.} como
$$\boxed{\langle f , g \rangle = \int_a^b f(t) g^\ast(t) \,dt }$$
\end{defi}

\begin{propo}[Propiedades del producto escalar] \label{ProductoEscalar}
 Sean $f,g,h \in \mathscr{C}[a,b]$ y  $\lambda \in \mathbb{C}$.
 
 \begin{itemize}
     \item $\langle f , g \rangle = \langle g, f \rangle^*$
     \item $\langle f , g + h \rangle = \langle f , g \rangle + \langle f , h \rangle$
     \item $\langle f + g , h \rangle = \langle f , h \rangle + \langle g , h \rangle$
     \item $\langle \lambda f , g \rangle = \lambda \langle f , g \rangle$
     \item $\langle  f , \lambda g \rangle = \lambda^*\langle f , g \rangle$
     \item Si $f \not\equiv 0$, entonces $\langle f , f \rangle > 0$
 \end{itemize}
\end{propo}


%% Considero que lo comentado es solo una formalidad matemática, que no es relevante al nivel de este curso

% \textbf{Observación:} Con respecto a la última propiedad, se tiene que 
% $$\langle f , f \rangle = \int_a^b |f(t)|^2 \,dt = 0 \color{red}{\nRightarrow} f = 0.$$

% Por esto se define la relación de equivalencia \footnote{$m(E) = 0$ denota que $E$ es un conjunto de medida cero.}
% $$f = g ~ c.t.p ~\Leftrightarrow~ f(t) = g(t), \forall t \in [a,b]-E, \mbox{con}~ m(E) = 0.$$

% Esta relación se expresa diciendo que $f = g $ casi en todas partes.

% \textbf{Notación:} $f \equiv g ~\Leftrightarrow~ f = g ~c.t.p$

% Esencialmente una relación de equivalencia es una relación de igualdad, bajo la cual dos funciones equivalentes se consideran ``iguales". Así, toda función $f = 0 ~c.t.p$, o simplemente $f\equiv 0$, se considera como la función nula.

% \begin{defi}
% Un espacio vectorial complejo dotado de un producto escalar con las propiedades de la proposición \ref{ProductoEscalar}, se conoce como \textbf{espacio pre-Hilbert}.
% \end{defi} 

\begin{defi}
Sea $f \in \mathscr{C}[a,b]$. Definimos su \textbf{norma} como
$$\norm{f} = \sqrt{\langle f,f \rangle} \in \mathbb{R}.$$
\end{defi}

\begin{propo}
Sean $f,g \in \mathscr{C}[a,b]$ y $\lambda \in \mathbb{C}$.

\begin{itemize}
    \item $\norm{f} \geq 0$
    \item $\norm{\lambda f} = |\lambda| \norm{f}$
    \item $|\langle f| g \rangle | \leq \norm{f} \cdot \norm{g}$ (Desigualdad de Cauchy-Schwarz)
    \item $\norm{f \pm g} \leq \norm{f} + \norm{g}$ (Desigualdad triangular)
    \item Si $f \not\equiv 0$, entonces $\norm{f} > 0$.
\end{itemize}
\end{propo}

\begin{demo}
Demostraremos solo la desigualdad de Cauchy-Schwarz y la triangular.

\begin{itemize}
    \item \textbf{Desigualdad de Cauchy-Schwarz}:
    
    Sea $\lambda \in \mathbb{C}$ arbitrario,
\begin{align*}
0 \leq \norm{\lambda f + g}^2 = \langle \lambda f + g , \lambda f + g\rangle &= \langle \lambda f , \lambda f \rangle + \langle \lambda f , g \rangle + \langle g, \lambda f \rangle + \langle g , g \rangle \\
&= \lambda \lambda^* \norm{f}^2 + \lambda \langle f,g\rangle + \lambda^* \langle g,f\rangle + \norm{g}^2.
\end{align*}

Siendo $\lambda$ arbitrario, consideremos entonces
\begin{equation*}
    \lambda = - \frac{\langle g,f \rangle}{\norm{f}^2} ~\Rightarrow~ \lambda^* = - \frac{\langle f,g\rangle}{\norm{f}^2}, \quad \norm{f} \neq 0.
\end{equation*}

Luego, 
$$0 \leq \frac{|\langle f, g \rangle|^2}{\norm{f}^4} \norm{f}^2 - 2 \frac{|\langle f,g \rangle|^2}{\norm{f}^2} + \norm{g}^2 = -\frac{|\langle f,g \rangle|^2}{\norm{f}^2} + \norm{g}^2. $$

Lo que implica 
$$|\langle f,g \rangle|^2 \leq \norm{f}^2 \cdot \norm{g}^2 ~\Rightarrow~ \boxed{|\langle f , g \rangle| \leq \norm{f} \cdot \norm{g}}$$

Si suponemos que $\norm{f} = 0$, $f \equiv 0$ y la desigualdad se demuestra trivialmente.

 \item \textbf{Desigualdad triangular}: De la definición de norma
 \begin{align*}
     \norm{f\pm g}^2 = \langle  f \pm g , f \pm g \rangle &= \langle f \pm g , f \rangle \pm \langle f \pm g , g\rangle \\
     &= \langle f,f \rangle \pm \langle f , g \rangle^* \pm \langle f,g \rangle +  \langle g,g \rangle \\
     &= \norm{f}^2 \pm 2 \real(\langle f,g \rangle) + \norm{g}^2.
 \end{align*}
 
 Como $\pm \real(z) \leq |z|$ para todo $z \in \mathbb{C}$, obtenemos que 
 $$\norm{f\pm g}^2 \leq \norm{f}^2+ 2 |\langle f,g \rangle| + \norm{g}^2.$$
 
 Por la desigualdad de  Cauchy-Shwarz:
 $$\norm{f\pm g}^2 \leq \norm{f}^2 + 2 \norm{f} \cdot \norm{g} + \norm{g}^2 = (\norm{f} + \norm{g})^2 ~\Rightarrow~ \boxed{\norm{f \pm g} \leq \norm{f} + \norm{g}}$$
\end{itemize}


\end{demo}

% \section{Sucesiones y series de funciones}

% \begin{defi}[Sucesión de funciones]
% Sea $\{f_n\}_{n \in \mathbb{N}}$ una sucesión de funciones 
% $$f_n: D \subseteq \mathbb{R} \longrightarrow \mathbb{C}.$$

% y considere $f: D \subseteq \mathbb{R} \longrightarrow \mathbb{C}$.

% \begin{enumerate}
%     \item Diremos que $\{f_n\}_{n \in \mathbb{N}}$ \textbf{converge puntualmente} a $f$ si dado $t \in D$ se tiene que la sucesión de números complejos $\{f_n(t)\}_{n \in \mathbb{N}}$ converge a $f(t)$ donde $f$ se llama la \textbf{función límite} de $\{f_n\}_{n \in \mathbb{N}}$, matemáticamente:
%     $$(\forall t \in D)(\forall \varepsilon > 0)(\exists N(t,\varepsilon) \in \mathbb{N})(n \geq N ~\Rightarrow~ |f_n(t) - f(t)| < \varepsilon).$$
    
%     \textbf{Notación:} $\lim\limits_{n \to + \infty} f_n(t) = f(t)$.
    
%     \item  Diremos que $\{f_n\}_{n \in \mathbb{N}}$ \textbf{converge uniformemente} a $f$ si 
%      $$(\forall \varepsilon > 0)(\exists N(\varepsilon) \in \mathbb{N})(n \geq N ~\wedge~ \forall t \in D ~\Rightarrow~ |f_n(t) - f(t)| < \varepsilon).$$
     
%      \textbf{Notación:} $\lim\limits_{n \to + \infty} f_n(t) = f(t) ~[uniforme]$.
% \end{enumerate}

% \end{defi}

% \textbf{Observación:} Es fácil de ver que si $\{f_n\}_{n\in \mathbb{N}}$ converge uniformemente a $f$, entonces $\{f_n\}_{n\in \mathbb{N}}$ converge puntualmente a $f$.

% \begin{defi}[Serie de funciones]
% Sea $\{f_n\}_{n \in \mathbb{N}}$ una sucesión de funciones
% $$f_n: D \subseteq \mathbb{R} \longrightarrow \mathbb{C}.$$

% Sea $F_n = \sum\limits_{k=1}^n f_k$. Se llama \textbf{serie de funciones} a la sucesión de sumas parciales $\{F_n\}_{n\in\mathbb{N}}$ y se denota por $\sum\limits_{n=1}^{\infty} f_n$.

% \begin{enumerate}
%     \item La serie $\sum\limits_{n=1}^{\infty} f_n$ converge (puntualmente) a $F$ si y solamente si $\{F_n\}_{n \in \mathbb{N}}$ converge puntualmente a $F$ sobre $D$.
    
%     \textbf{Notación:} 
%     $$\sum_{n=1}^{\infty} f_n = F = \lim_{n\to + \infty} \sum_{k=1}^n f_k.$$
    
%     \item La serie $\sum\limits_{n=1}^{\infty} f_n$ converge uniformemente a $F$  sobre $D$ si y solamente si $\{F_n\}_{n \in \mathbb{N}}$ converge uniformemente a $F$ sobre $D$.
    
%     \textbf{Notación:} 
%     $$\sum_{n=1}^{\infty} f_n = F ~[uniforme].$$
    
%     \item La serie $\sum\limits_{n=1}^{\infty} f_n$ converge absolutamente a $F$  sobre $D$ si y solamente si la serie $\sum\limits_{n=1}^{\infty} |f_n|$ converge puntualmente a $F$ sobre $D$.
    
% \end{enumerate}
% \end{defi}

% \begin{defi}
% La \textbf{distancia} entre dos funciones $f,g \in \mathscr{C}[a,b]$ se define por 
% $$\boxed{\norm{f-g} = \sqrt{\int_a^b [f(t)-g(t)][f(t)-g(t)]^* dt}}$$
% \end{defi}

% A partir de la definición y las propiedades de la norma, se tiene que
% $$\norm{f-g} = 0 ~\Leftrightarrow~ f \equiv g.$$

% \begin{defi}
% Un \textbf{espacio métrico} es un conjunto $X$ provisto de una \textbf{distancia} (o \textbf{métrica}) $d: X \times X \rightarrow \mathbb{R}$ que verifica:

% \begin{enumerate}
%     \item[(i)] $\forall x,y \in X: d(x,y) = 0 \Leftrightarrow x = y$.
    
%     \item[(ii)] $\forall x,y \in X: d(x,y) = d(y,x)$.
    
%     \item[(iii)] $\forall x,y,z \in X: d(x,y) \leq d(x,z) + d(z,y)$. (Desigualdad triangular)
% \end{enumerate}
% \end{defi}

% \textbf{Observación:} Note que de las condiciones para una métrica, se desprende la no negatividad de la función $d$. En efecto, para todo $x,y \in X$, se tiene que
% $$d(x,x) = 0 \leq d(x,y) + d(y,x) = d(x,y) + d(x,y) = 2  d(x,y) \Rightarrow d(x,y) \geq 0.$$

% El par $(\mathscr{C}[a,b], \norm{\cdot})$ es un espacio métrico y como tal se introducen los conceptos de convergencia de sucesiones y series en el sentido de la distancia dada en este espacio. La convergencia en esta métrica se llama \textbf{convergencia en media} o \textbf{convergencia cuadrática}.

% \begin{defi}
% Sea $\{f_n\}_{n\in \mathbb{N}}$ una sucesión de elementos de $\mathscr{C}[a,b]$. Se dice que $\{f_n\}_{n\in \mathbb{N}}$ \textbf{converge en media} a $f \in \mathscr{C}[a,b]$ si 
% \begin{equation*}
%     \lim_{n \to + \infty} \norm{f_n - f} = 0.
% \end{equation*}

% Se escribe, $\lim\limits_{n \to + \infty} f_n = f ~[en ~media]$ en $[a,b]$.
% \end{defi}

% \textbf{Observación:}
% \begin{shaded}
% $$ \lim_{n \to + \infty} \norm{f_n - f} = 0  ~\Leftrightarrow~ \lim_{n \to + \infty} \int_a^b |f_n(t) - f(t)|^2 dt = 0.$$    
% \end{shaded}

% \begin{propo}
% Considere $f,f_n \in \mathscr{C}[a,b]$, $n\in \mathbb{N}$. Si $\{f_n\}_{n \in \mathbb{N}}$ converge uniformemente a $f$, entonces $\{f_n\}_{n \in \mathbb{N}}$ converge en media a $f$.
% \end{propo}

% \begin{demo}
% Por hipótesis tenemos que dado $\varepsilon > 0$, existe $N = N(\varepsilon) \in \mathbb{N}$ tal que
% \begin{align*}
%     n \geq N ~\wedge~ \forall t \in [a,b] &\Rightarrow |f_n(t) - f(t)| < \sqrt{\frac{\varepsilon}{b-a}} \\
%     &\Rightarrow |f_n(t) - f(t)|^2 < \frac{\varepsilon}{b-a} \\
%     &\Rightarrow \int_a^b |f_n(x) - f(x)|^2 \,dt < \varepsilon. \qquad \mbox{(Propiedad de Monotonía)}
% \end{align*}

% Por lo tanto, 
% $$\forall\varepsilon > 0, \exists N \in \mathbb{N}: ~ n \geq N ~\Rightarrow~ \int_a^b |f_n(t) - f(t)|^2 \,dt < \varepsilon,$$

% lo que muestra que $\{f_n\}_{n \in \mathbb{N}}$ converge en media a $f$

% \end{demo}

% \textbf{Observación:} No hay relación entre la convergencia en media y la convergencia puntual.

% \begin{ejemplo}
% Sea la sucesión de polinomios definidos por $p_n(x) = x^n, n \in \mathbb{N}$ para $x \in [-1,1]$.

% \begin{figure}[H]
%     \centering
%     \includegraphics[scale = 0.57]{Figuras/SucesionPolinomios.pdf}
%     \caption{Sucesión de polinomios $p_n(x) = x^n, x \in [-1,1]$ para $n = 1, \dots, 6$.}
% \end{figure}

% Ésta converge en media a $f \equiv 0$. En efecto, 
% \begin{align*}
% \lim_{n \to + \infty} \int_{-1}^1 |x^n - 0|^2 dx  = \lim_{n \to + \infty }  \int_{-1}^1 x^{2n} \,dx &= \lim_{n \to + \infty} \left. \frac{x^{2n+1}}{2n+1} \right|_{-1}^1 \\
% &=  \lim_{n \to + \infty} \frac{2}{2n+1} = 0.
% \end{align*}

% Sin embargo, no converge puntualmente a $f$ sobre $[-1,1]$, pues 
% $$\lim_{n \to + \infty} x^n = \left\{ \begin{array}{cl}
%    0 ,& \mbox{si}~ -1 < x < 1 \\
%    1  ,&  \mbox{si}~ x = 1 \\
%    \mbox{diverge},&  \mbox{si}~ x = -1
% \end{array}  \right. .$$

% Luego, tampoco uniformemente a $f \equiv 0$ sobre $[-1,1]$.
% \end{ejemplo}

% \begin{defi}
% Considere $f, f_n \in \mathscr{C}[a,b], n \in \mathbb{N}$ y $F_n = \sum\limits_{k=1}^n f_k$. Diremos que la serie $\sum\limits_{n=1}^{\infty} f_n$ \textbf{converge en media} a $f$ si la sucesión de sumas parciales $\{F_n\}_{n\in \mathbb{N}}$ converge en media a $f$. 
% \\

% \textbf{Notación:} 
% $$\sum_{n=1}^{\infty} f_n(t) \sim f(t), \quad t \in [a,b]$$

% o 
% $$\sum_{n=1}^{\infty} f_n(t) = f(t) ~ [en ~media], \quad t \in [a,b].$$
% \end{defi}

% \textbf{Observación:}  
% \begin{shaded}
% $$\sum_{n=1}^{\infty} f_n(t) \sim f(t), \quad t \in [a,b] ~\Leftrightarrow~ \lim_{n \to + \infty} \int_a^b \left[ \sum_{k=1}^n f_k(t) - f(t)\right]^2 \, dt = 0.$$ 
% \end{shaded}

% \begin{defi}
% Una sucesión $\{f_n\}_{n \in \mathbb{N}}$ en $\mathscr{C}[a,b]$ se dice \textbf{sucesión de Cauchy} si dado $\varepsilon > 0$, existe un $N \in \mathbb{N}$ tal que 
% $$\forall n,m \geq N ~\Rightarrow~ \norm{f_n-f_m} < \varepsilon.$$
% \end{defi}

% La definición anterior se puede generalizar a espacios de funciones sin norma, pero con una métrica definida.

% Es inmediato verificar que $\{f_n\}_{n \in \mathbb{N}}$ converge a una función $f$ en media, entonces es de Cauchy, pues 
% $$\norm{f_n - f_m} \leq \norm{f_n - f} + \norm{f_m - f},$$

% y ambos términos en el lado derecho se pueden acotar por un $\varepsilon > 0$ arbitrario para todo $n,m \geq N$. El inverso, sin embargo, es falso, como se puede apreciar en el siguiente ejemplo:

% \begin{ejemplo}
% Consideremos el conjunto de funciones reales 
%  $\mathscr{C}_0[0,1]$, con el producto escalar definido como 
% $$\langle f,g\rangle = \int_0^1 f(x) g(x) \,dx.$$

% Sea  
% \begin{equation*}
%     f_n(x) = \left\{ \begin{array}{cl}
%        1,  & \mbox{si} ~ 0 \leq x \leq \frac{1}{2}\\
%     1 - \left(x - \frac{1}{2} \right)n,     & \mbox{si}  ~ \frac{1}{2} < x < \frac{1}{2} + \frac{1}{n} \\
%     0, & \mbox{si} ~ \frac{1}{2} + \frac{1}{n} \leq x \leq 1
%     \end{array} \right., n \geq 2.
% \end{equation*}

% \begin{figure}[H]
%     \centering
%     \includegraphics[scale = 0.5]{Figuras/EjemploCauchy.pdf}
%     \caption{Ejemplo de sucesión de Cauchy que no converge en media a $C_0[0,1]$.}
% \end{figure}

% Para $m \geq n$, 
% \begin{align*}
%     \norm{f_n - f_m}^2 = \int_0^1 |f_n(x) - f_m(x)|^2 \,dx &= \int_0^{\frac{1}{2}} [1-1] \,dx + \int_{\frac{1}{2}}^{\frac{1}{2} + \frac{1}{n}} |f_n(x) - f_m(x)|^2 \,dx + \int_{\frac{1}{2} + \frac{1}{n}}^1 0 \, dx \\
%     &=  \int_{\frac{1}{2}}^{\frac{1}{2} + \frac{1}{n}} |f_n(x) - f_m(x)|^2 \,dx.
% \end{align*}

% Ahora, para todo $x \in \left[\frac{1}{2}, \frac{1}{2}  + \frac{1}{n} \right]$, tenemos

% $$|f_n(x) - f_m(x)| \leq |f_n(x)| + |f_m(x)| \leq 1 + 1 = 2. $$

% Luego, 
% \begin{equation*}
%    \norm{f_n - f_m}^2  = \int_{\frac{1}{2}}^{\frac{1}{2} + \frac{1}{n}} |f_n(x) - f_m(x)|^2 \,dx \leq \int_{\frac{1}{2}}^{\frac{1}{2} + \frac{1}{n}} 4 \,dx = \frac{4}{n} ~\Rightarrow~ \norm{f_n - f_m} \leq \frac{2}{\sqrt{n}}.
% \end{equation*}

% Para $m < n$, es fácil de ver que 
% $$\norm{f_n - f_m} \leq \frac{2}{\sqrt{m}}.$$

% Así, dado $\varepsilon > 0$, por propiedad arquimediana, existe $N \in \mathbb{N}$ tal que 
% $$N > \frac{4}{\varepsilon^2} $$

% que verifica 
% $$\forall m,n \geq N ~\Rightarrow~   \norm{f_n - f_m} < \varepsilon$$

% de modo que es una sucesión de Cauchy. Sin embargo, $\{f_n\}$ converge en media a una función discontinua en $x = \frac{1}{2}$ (pruébelo!), y por lo tanto no converge en $\mathscr{C}_0[0,1]$.
% \end{ejemplo}

% \begin{defi}
% Un espacio normado es llamado \textbf{completo} si toda sucesión de Cauchy es convergente. A un espacio normado completo se le llama \textbf{espacio de Banach}. A un espacio pre-Hilbert que es completo se le llama \textbf{espacio de Hilbert}
% \end{defi}

\begin{defi}
El conjunto de funciones $\{\varphi_n(t)\}_{n=0, \pm 1, \pm 2, \dots}$ se dice \textbf{ortogonal} si 
$$\langle \varphi_n , \varphi_m \rangle = 0, \quad \mbox{para} ~ n \neq m.$$

Si además, $\norm{\varphi_n} = 1$ para cada $n \in \mathbb{Z}$, se dice que es un conjunto \textbf{ortonormal}, entonces podemos escribir 
$$\langle \varphi_n , \varphi_m \rangle = \delta_{nm}, \quad \forall \ n,m.$$
\end{defi}

\begin{ejemplo}
Como ejemplo de funciones ortonormales tenemos las $c_n(t) \in \mathscr{C}_0[-\pi,\pi]$ con $n = 0, \pm 1, \pm 2, \dots$ que se definen como
$$c_n(t) = \frac{1}{\sqrt{2\pi}} e^{i nt}.$$

En efecto, para $n \neq m$, se tiene que
\begin{align*}
    \langle c_n , c_m \rangle = \frac{1}{2\pi} \int_{-\pi}^{\pi} e^{i(n-m)t} \,dt &= \frac{1}{2\pi} \left[ -\frac{i}{n-m} e^{i(n-m) t}\right]_{-\pi}^{\pi} \\
    &= \frac{i}{2\pi(m-n)} [e^{i n\pi} + e^{-i m \pi} - e^{-in \pi} - e^{im \pi}] \\
    &= 0.
\end{align*}

Por otro lado, para $n = m$:
\begin{equation*}
  \langle c_n , c_n \rangle =\frac{1}{2\pi} \int_{-\pi}^{\pi} e^{i(n-n)t} \,dt = \frac{1}{2\pi} \int_{-\pi}^{\pi} 1 \,dt = 1.
\end{equation*}
$$\therefore  \langle c_n , c_m \rangle = \delta_{nm}.$$
\end{ejemplo}

\begin{ejemplo}
Pruebe que el conjunto de funciones 
$$\left\{ \frac{1}{\sqrt{2 \pi}}, \frac{\cos(nt)}{\sqrt{\pi}} ,  \frac{\sin(nt)}{\sqrt{\pi}} \right\}_{n=1}^{\infty}$$

es ortonormal en $\mathscr{C}[-\pi,\pi]$.
\\

\textbf{Solución}: Probemos primero la normalización.
\begin{align*}
    \int_{-\pi}^{\pi} \left( \frac{1}{\sqrt{2\pi}} \right)^2 \,dt &= 1, \\
    \int_{-\pi}^{\pi} \frac{\cos^2(nt)}{\pi} \,dt &=  \frac{1}{\pi} \int_{-\pi}^{\pi} \frac{1}{2} + \frac{1}{2} \cos(2n t) \,dt = 1, \\
    \int_{-\pi}^{\pi} \frac{\sin^2(nt)}{\pi} \,dt &=  \frac{1}{\pi} \int_{-\pi}^{\pi} \frac{1}{2} - \frac{1}{2} \cos(2n t) \,dt = 1.
\end{align*}

Para la ortogonalidad, tengamos en cuenta las siguientes identidades trigonométricas:
\begin{align*}
    \sin \alpha \sin \beta &= \frac{1}{2} [\cos(\alpha - \beta) - \cos(\alpha + \beta)], \\
    \cos \alpha \cos \beta &= \frac{1}{2} [\cos(\alpha - \beta) + \cos(\alpha + \beta)], \\
    \sin \alpha \cos \beta &= \frac{1}{2} [\sin(\alpha + \beta) + \sin(\alpha - \beta)].
\end{align*}

Entonces, 
\begin{align*}
    \int_{-\pi}^{\pi} \frac{1}{\sqrt{2} \pi} \cos(nt) \,dt &=  \left. \frac{1}{\sqrt{2}\pi n} \sin(nt) \right|_{-\pi}^{\pi} = 0, \\
     \int_{-\pi}^{\pi} \frac{1}{\sqrt{2} \pi} \sin(nt) \,dt &=   \left. - \frac{1}{\sqrt{2}\pi n} \cos(nt) \right|_{-\pi}^{\pi} = 0,\\
      \int_{-\pi}^{\pi} \frac{1}{\pi} \cos(n t) \sin(m t)\,dt &=  \frac{1}{2 \pi} \int_{-\pi}^{\pi} \sin(m+n)t + \sin(m-n)t \ dt = 0; \quad n,m \in \mathbb{N}. 
\end{align*}

Para todo $n, m \in \mathbb{N}$, $m \neq n$, se tiene que 
\begin{align*}
    \int_{-\pi}^{\pi} \frac{1}{\pi} \cos(n t) \cos(mt) \,dt &= \frac{1}{2\pi} \int_{-\pi}^{\pi} \cos(n-m)t + \cos(n+m) t\,dt \\
    &= \frac{1}{2\pi} \left[ \frac{1}{n-m} \sin(n-m)t + \frac{1}{n+m} \sin(n+m)t \right]_{-\pi}^{\pi} = 0. \\
     \int_{-\pi}^{\pi} \frac{1}{\pi} \sin(n t) \sin(mt) \,dt &=\frac{1}{2\pi} \int_{-\pi}^{\pi} \cos(n-m)t - \cos(n+m) t\,dt  \\
    &= \frac{1}{2\pi} \left[ \frac{1}{n-m} \sin(n-m)t - \frac{1}{n+m} \sin(n+m)t \right]_{-\pi}^{\pi} = 0.
\end{align*}

Por lo tanto, 
$$\left\{ \frac{1}{\sqrt{2 \pi}}, \frac{\cos(nt)}{\sqrt{\pi}} ,  \frac{\sin(nt)}{\sqrt{\pi}} \right\}_{n=1}^{\infty}$$

es ortonormal en $\mathscr{C}[-\pi,\pi]$.
\end{ejemplo}

\begin{defi}
Sea $S = \{\varphi_n(t)\}_{n=0, \pm 1, \pm 2, \dots}$. Se dice que $S$ es \textbf{linealmente independiente} (l.i.) si todo subconjunto finito de $S$ también lo es.
\end{defi}

\begin{propo} \label{LIortogonal}
Todo conjunto ortogonal en $\mathscr{C}[a,b]$ que no contenga al vector nulo es linealmente independiente.
\end{propo}

\begin{demo}
Sea $S = \{\varphi_n(t)\}_{n=0, \pm 1, \pm 2, \dots}$ ortogonal tal que $\varphi_n \not\equiv 0, \forall n$. Consideremos el subconjunto finito de $S$, $S' = \{\varphi_{i_1}, \dots, \varphi_{i_n}\}$ y además la combinación lineal
$$\alpha_1 \varphi_{i_1} + \alpha_2 \varphi_{i_2} + \cdots + \alpha_n \varphi_{i_n} \equiv 0.$$

Entonces, para un cierto $\varphi_{i_m}$, se tiene que
$$\langle \alpha_1 \varphi_{i_1} + \alpha_2 \varphi_{i_2} + \cdots + \alpha_n \varphi_{i_n},\varphi_{i_m} \rangle = \alpha_m \underbrace{\norm{\varphi_{i_m}}^2}_{\neq 0} = 0.$$

Por lo tanto, 
$$\alpha_m = 0, \quad m = 1, 2, \dots, n$$

probando así que $S'$ es linealmente independiente y en consecuencia $S$ es l.i.
\end{demo}

\section{Proceso de ortonormalización de Gram-Schmidt}

Sea $\{v_n\}_{n = 1,2, \dots}$ un conjunto linealmente independiente de funciones en $\mathscr{C}[a,b]$. Para construir un conjunto ortonormal debemos seguir los siguientes pasos:

\begin{enumerate}
    \item Construimos 
    $$\varphi_1 = \frac{v_1}{\norm{v_1}}$$ 
    
    tal que $\langle \varphi_1 , \varphi_1 \rangle = 1$. 
    
    \item Consideramos
    $$\overline{\varphi}_2 = v_2 - \langle v_2, \varphi_1   \rangle \varphi_1.$$
    
    Entonces, 
    $$\langle \overline{\varphi}_2, \varphi_1 \rangle = \langle v_2, \varphi_1 \rangle - \langle  v_2, \varphi_1  \rangle \langle \varphi_1, \varphi_1 \rangle = \langle v_2, \varphi_1 \rangle -  \langle  v_2, \varphi_1 \rangle  = 0.$$
    
    Normalizando, 
    $$\varphi_2 = \frac{\overline{\varphi}_2}{\norm{\overline{\varphi}_2}}.$$
    
    \item En general para un cierto $n \geq 2$, consideremos 
 $$\overline{\varphi}_n = v_n - \sum_{j=1}^{n-1} \langle  v_n, \varphi_j \rangle \varphi_j.$$
    
Entonces, para $1 \leq  i \leq n-1$, tenemos que
\begin{align*}
    \langle \Bar{\varphi}_n , \varphi_i \rangle &= \langle v_n , \varphi_i \rangle - \sum_{j=1}^{n-1} \langle v_n , \varphi_j \rangle \langle \varphi_j , \varphi_i\rangle \\
    &= \langle v_n, \varphi_i \rangle - \sum_{j=1}^{n-1} \langle v_n , \varphi_j \rangle \delta_{ji} \\
    &= \langle v_n , \varphi_i \rangle -  \langle v_n, \varphi_i \rangle = 0.
\end{align*}
    
Finalmente, normalizando    
$$\varphi_n = \frac{\overline{\varphi}_n}{\norm{\overline{\varphi}_n}}.$$
\end{enumerate}

El conjunto de funciones $\{\varphi_n\}_{n = 1,2, \dots}$ construido de la manera anterior es un conjunto ortonormal.

Geométricamente, el método se encuentra ilustrado en la figura \ref{fig:Gram-Schmidt}, donde se ha considerado las funciones como vectores y solo el proceso de ortogonalización.

\begin{figure}[H]
    \centering
    \includegraphics[scale = 0.7]{Figuras/Gram-Schmidt.pdf}
    \caption{Proceso de ortogonalización (sin la normalización) de Gram-Schmidt para tres funciones $\{v_1,v_2,v_3\}$.}
    \label{fig:Gram-Schmidt}
\end{figure}

\newpage
\section{Coeficientes de Fourier}

Ahora, definiremos un espacio de funciones más general que $\mathscr{C}[a,b]$, las funciones cuadrado integrables. \footnote{La condición de cuadrado integrable es usada, por ejemplo, en Mecánica Cuántica, pues constituye la base para que las funciones de onda describan el comportamiento de los sistemas físicos, consecuencia de la interpretación de Copenhague (probabilística) de la mecánica cuántica.}

\begin{defi}
    Definimos $\mathcal{L}^2[a,b]$ como el espacio de funciones $f:[a,b] \rightarrow \mathbb{C}$, tales que 
    $$\int_a^b |f(t)|^2 \,dt < \infty.$$
\end{defi}

\begin{teorema}
El espacio $\mathcal{L}^2[a,b]$ es un espacio vectorial con producto interno 
$$\langle f,g \rangle = \int_a^b f(t) g^* (t) \,dt$$

y norma
$$\norm{f} = \left( \int_a^b |f(t)|^2 \,dt \right)^{1/2}.$$
\end{teorema}

La demostración requiere verificar las propiedades del producto interno (escalar) dadas por \ref{ProductoEscalar}, la cual está fuera de los alcances de los contenidos de este apunte. \footnote{Para más información puede consultar bibliografía relacionada a la integral de Lebesgue.} 

\textbf{Observación:} Las funciones seccionalmente continuas son funciones cuadrado integrables.
\\

Sea $\{\varphi_{\nu}(t)\}_{\nu \in \mathbb{N}}$ un conjunto ortonormal de funciones tales que $\varphi_{\nu} \in \mathscr{C}[a,b]$ para todo $\nu \in \mathbb{N}$. Sea $f(t)$ una función cuadrado integrable en $[a,b]$. Deseamos aproximar $f(t)$ por una suma finita 
$$S_n(t) = \sum_{\nu = 1}^n C_{\nu} \varphi_{\nu}(t),$$

de manera que $\norm{f - S_n}$ sea mínimo. Es decir, el objetivo es encontrar los coeficientes $C_{\nu}$ de modo que el \textbf{error cuadrático medio}
$$M_n(f) = \norm{f-S_n}^2 = \int_a^b \left| f(t) - \sum_{\nu = 1}^n C_{\nu} \varphi_{\nu}(t) \right|^2 \,dt,$$

sea mínimo. Evaluemos el error cuadrático medio
\begin{align}
    M_n(f) &= \int_a^b \left( f(t) - \sum_{\nu = 1}^n C_{\nu} \varphi_{\nu}(t)  \right) \left( f(t) - \sum_{\nu = 1}^n C_{\nu} \varphi_{\nu}(t) \right)^* \,dt \nonumber \\
    &= \int_a^b |f(t)|^2 \, dt  + \sum_{\nu = 1}^n |C_{\nu}|^2 \int_a^b |\varphi_{\nu}(t)|^2 \,dt - \sum_{\nu = 1}^n C_{\nu}^* \int_a^b f(t) \varphi_{\nu}^*(t) \,dt \nonumber \\
    & \quad - \sum_{\nu = 1}^n C_{\nu} \int_a^b f^*(t) \varphi_{\nu}(t) \,dt  \nonumber \\
    &= \norm{f}^2 + \sum_{\nu = 1}^n |C_{\nu}|^2 - \sum_{\nu = 1}^n C_{\nu}^* \langle f, \varphi_{\nu} \rangle - \sum_{\nu = 1}^n C_{\nu} \langle f, \varphi_{\nu} \rangle^* + \sum_{\nu = 1}^n |\langle f, \varphi_{\nu} \rangle|^2 - \sum_{\nu = 1}^n |\langle f, \varphi_{\nu} \rangle|^2\nonumber  \\
    &= \norm{f}^2  - \sum_{\nu = 1}^n |\langle f, \varphi_{\nu} \rangle|^2 + \sum_{\nu = 1}^n |C_{\nu} - \langle f , \varphi_{\nu} \rangle|^2 \geq 0, \label{ErrorMedio}
\end{align}

ya que la norma es mayor o igual a cero siempre. Claramente el mínimo se obtiene cuando $C_{\nu} = \langle f, \varphi_{\nu} \rangle$. 

De lo anterior se desprende: 
\begin{shaded}
 \begin{equation}
 \sum_{\nu = 1}^n |C_{\nu}|^2 = \sum_{\nu = 1}^n |\langle f , \varphi_{\nu} \rangle|^2 \leq \norm{f}^2  \qquad \mbox{\textbf{Desigualdad de Bessel}} \label{D.Bessel}.
\end{equation}
 
\end{shaded}

Como el número a la derecha de la desigualdad es independiente de $n$, la suma está acotada superiormente. Siendo todos sus términos no negativos, tenemos que
$$\sum_{\nu = 1}^{\infty} |C_{\nu}|^2 < \infty ~\Rightarrow~ \lim_{\nu \to + \infty} |C_{\nu}|^2 = 0 ~\Rightarrow ~ \lim_{\nu \to + \infty} \langle f , \varphi_{\nu} \rangle = 0.$$

Luego,
\begin{equation*}
 \sum_{\nu = 1}^{\infty} |C_{\nu}|^2 = \sum_{\nu = 1}^{\infty} |\langle f , \varphi_{\nu} \rangle|^2 \leq \norm{f}^2    .
\end{equation*}

\begin{defi}
Los coeficientes $\langle f , \varphi_{\nu}\rangle$ son llamados los \textbf{coeficientes de Fourier} de $f$  respecto al sistema ortonormal $\{\varphi_{\nu}\}_{\nu = 1,2, \dots}$. La serie $\sum\limits_{\nu = 1}^{\infty} C_{\nu} \varphi_{\nu}(t)$ se llama \textbf{serie generalizada de Fourier} de $f$ relativa al sistema ortonormal $\{\varphi_{\nu}\}_{\nu = 1,2, \dots}$.

\end{defi}

\begin{defi}
Si un conjunto de funciones $\{\varphi_{\nu}\}$ en cierto espacio permite aproximar en la norma (en media), con sus combinaciones lineales, cualquier función $f$ del espacio tan bien como se quiera, es decir,
$$\left\Vert f - \sum_{\nu} C_{\nu} \varphi_{\nu} \right\Vert< \varepsilon, \qquad \mbox{para $\varepsilon$ arbitrario},$$

se dice que es un \textbf{conjunto completo} respecto a este espacio.
\end{defi}

Sean $C_{\nu} = \langle f , \varphi_{\nu} \rangle$ los coeficientes de Fourier de $f$ respecto del conjunto ortonormal  $\{\varphi_{\nu}\}$, entonces la completitud de este conjunto se puede expresar por 
$$\lim_{n \to + \infty} \left\Vert f - \sum_{\nu =1}^{n} C_{\nu} \varphi_{\nu} \right\Vert = 0,$$

es decir, 
$$f \sim \sum_{\nu = 1}^{\infty} C_{\nu} \varphi_{\nu}.$$

Lo anterior NO implica que $f(t) = \sum\limits_{\nu = 1}^{\infty} C_{\nu} \varphi_{\nu}(t)$ en algún otro sentido (convergencia puntual o uniforme). Si
$$f \sim \sum_{\nu = 1}^{\infty} C_{\nu} \varphi_{\nu},$$

entonces de la relación \eqref{ErrorMedio}, tenemos 
\begin{equation*}
    \lim_{n \to + \infty} \left\Vert f - \sum_{\nu = 1}^n  C_{\nu} \varphi_{\nu} \right\Vert^2 = \lim_{n \to + \infty} \left\{ \Vert f \Vert^2 - \sum_{\nu = 1}^{n} |C_n|^2 \right\} = \norm{f}^2  - \sum_{\nu = 1}^{\infty} |C_n|^2 = 0.
\end{equation*}

Lo que implica que
\begin{shaded}
 \begin{equation}
    \norm{f}^2  = \sum_{\nu =1 }^{\infty} |C_{\nu}|^2 \qquad \mbox{\textbf{Igualdad de Parseval}}
\end{equation}   
\end{shaded}

\textbf{Observación:} El conjunto ortogonal completo $\{\varphi_{\nu}\}$ se le conoce también como \textbf{base ortogonal} del espacio de funciones en cuestión.

\begin{ejemplo}
El conjunto $\left\{ \frac{1}{\sqrt{2 \pi}} e^{i n t} \right\}_{n \in \mathbb{Z}}$ es ortonormal completo respecto a $[-\pi,\pi]$.
$$f(t) \sim \sum_{n = - \infty}^{\infty} C_n \frac{e^{int}}{\sqrt{2\pi}} ~\Rightarrow~ \int_{-\pi}^{\pi} |f|^2 \,dt = \sum_{n=- \infty}^{\infty} |C_n|^2 = \norm{f}^2 .$$
\end{ejemplo}


\begin{teorema}{}{} 
Si el conjunto ortonormal $\{\phi_{\nu}\}$ es completo respecto a $\mathscr{C}[a,b]$, entonces en $\mathscr{C}[a,b]$ la única función ortonormal a todo $\varphi_{\nu}$ es $f(t) \equiv 0$.
\end{teorema}

\begin{demo}
Sea $f$ una función ortonormal a todo $\varphi_{\nu}$, si $f(t_0) \neq 0$ para algún $t_0 \in [a,b]$, la función también es no nula en una vecindad en torno a $t_0$ (por continuidad), por lo tanto 
$$\int_a^b |f(t)|^2 \,dt = \norm{f}^2  > 0,$$

pero usando la igualdad de Parseval, tenemos para la norma de $f$ que
$$\norm{f}^2  = \sum_{\nu} |C_{\nu}|^2 = \sum_{\nu} |\langle f ,\varphi_{\nu} \rangle|^2 > 0,$$

es decir, $f$ no es ortogonal a todos los $\varphi_{\nu}$, lo cual es una contradicción. Luego, $f$ debe ser idénticamente nula.
\end{demo}

\begin{teorema}
Sea $\{S_n(t) \in \mathscr{C}_0[a,b]\}$; si existe $F(t)$ tal que la sucesión $S_n(t) = \sum\limits_{\nu = 1}^n C_{\nu} \varphi_{\nu}(t)$ converge uniformemente a $F(t)$, entonces $F(t)$ es continua, es decir, $F(t) \in \mathscr{C}_0[a,b]$.
\end{teorema}

\begin{demo}
Por convergencia uniforme, dado $\varepsilon > 0$, $\exists N \in \mathbb{N}$ tal que
$$n \geq N \wedge \forall t \in [a,b] ~\Rightarrow~ |S_n(t) - F(t)| < \frac{\varepsilon}{3}.$$

Además, por la continuidad de $S_n$ para todo $t_0 \in [a,b]$, existe $\delta(\varepsilon, N, t_0)$ tal que
$$\forall t \in [a,b]: ~ 0 < |t-t_0| < \delta ~\Rightarrow~ |S_N(t) - S_N(t_0)| < \frac{\varepsilon}{3}.$$

Por lo tanto, 
\begin{align*}
 \forall t \in [a,b]: ~ 0 < |t-t_0| < \delta &\Rightarrow |F(t) - F(t_0)|  \\
 &= |F(t) - S_n(t) + S_n(t) - S_n(t_0) + S_n(t_0) - f(t_0)| \\
 &\leq |F(t) - S_n(t)| + |S_n(t) - S_n(t_0)| + |S_n(t_0) - F(t_0)| \\
 &\leq \frac{\varepsilon}{3} + \frac{\varepsilon}{3} + \frac{\varepsilon}{3} = \varepsilon .
\end{align*}
\end{demo}

Este teorema nos asegura que una función discontinua no puede ser aproximada uniformemente por una familia de funciones continuas (por ejemplo, las funciones sinusoidales).

\begin{teorema}
Si dos funciones $f,g \in \mathscr{C}[a,b]$ tienen igual expansión en base completa (en el sentido de aproximación en la norma), entonces $f(t) = g(t)$.
\end{teorema}

\begin{demo}
Sea
$$S(t) = \sum_{\nu = 1}^{\infty} \langle f, \varphi_{\nu}\rangle \varphi_{\nu}(t)$$

la aproximación en la norma para $f$ y $g$. Luego, 
$$\Vert f-S \Vert = \Vert g-S \Vert = 0.$$

Así, 
\begin{equation*}
    \Vert f-g \Vert = \Vert f-S+S-g \Vert \leq \Vert f-S \Vert + \Vert S-g \Vert = 0 +0 = 0 ~\Rightarrow~ f = g.
\end{equation*}

\end{demo}

% \section{Convergencia según Cesàro*}

% Si consideramos la serie
% $$\frac{1}{1-x} = \sum_{n=1}^{\infty} x^{n-1} = 1 + x + x^2 + x^3 + \cdots$$

% ella converge para $|x|< 1$. A pesar de lo anterior evaluemos la función y su expansión en serie en $x = -1$:
% $$\left. \frac{1}{1-x} \right|_{x= -1} \overset{?}{=} \frac{1}{2} = 1 - 1 + 1 -1 +1 -1 + \cdots$$

% ¿Será posible sumar la serie de modo que ésta sí converja al valor de la función en ese punto?

% \begin{defi}
% Sea $s_n$ la suma parcial $n$-ésima de la serie $\sum\limits_{n=1}^{\infty} a_n$ y sea $\{\sigma_n\}_{n\in \mathbb{N}}$ la sucesión de las medias aritméticas definidas por
% $$\sigma_n = \frac{s_1 + \cdots + s_n}{n}, \quad n = 1, 2, \dots$$

% La serie $\sum\limits_{n=1}^{\infty} a_n$ es \textbf{sumable de Cesàro} si $\{\sigma_n\}_{n\in \mathbb{N}}$ converge. Si $\lim\limits_{n \to + \infty} \sigma_n = S^*$, entonces $S^*$ se llama \textbf{suma de Cesàro} de $\sum\limits_{n=1}^{\infty} a_n$ y se escribe 
% $$ {\sum_{n=1}^{\infty}}^*   a_n = S^*.$$
% \end{defi}

% \begin{ejemplo}
% Sea $a_n = x^{n-1}$ con $x \neq 1$. Entonces
% $$s_n = \frac{1}{1-x} - \frac{x^n}{1-x} \qquad \mbox{(Demuéstrelo!!)}$$

% y
% \begin{align*}
%  \sigma_n = \frac{1}{n} \sum_{k=1}^n \left\{\frac{1}{1-x} - \frac{x^k}{1-x} \right\} &= \frac{1}{1-x} - \frac{1}{n(1-x)} \sum_{k =1}^{n} x^k   \\
%  &=\frac{1}{1-x} -  \frac{1}{n} \frac{x(1-x^n)}{(1-x)^2}.
% \end{align*}

% Por consiguiente, 
% $${\sum_{n=1}^{\infty}}^*  x^{n-1} = \lim_{n \to +\infty} \left\{ \frac{1}{1-x} -  \frac{1}{n} \frac{x(1-x^n)}{(1-x)^2}\right\} = \frac{1}{1-x}; \quad |x| \leq 1, x \neq 1. $$

% En particular,
% $${\sum_{n=1}^{\infty}}^*  (-1)^{n-1} = \frac{1}{2}.$$

% \end{ejemplo}

% Note que la idea de la definición de sumabilidad de Cesàro es encontrar una forma de dar significado a series que en otro caso serían divergentes.
% \\

% \textbf{Observación}: La convergencia ordinaria necesita que el $\lim\limits_{n \to + \infty} \sum\limits_{k = 1}^n a_{k}$ exista. 

% \hspace{2.45cm} La convergencia según Cesàro necesita que el $\lim\limits_{n \to + \infty} \frac{1}{n} \sum\limits_{k = 1}^n  \sum\limits_{l = 1}^{k} a_{l}$ exista.

% \begin{teorema}
% Si una serie es convergente con suma $S$, entonces es sumable de Cesàro con suma $S^* = S$.
% \end{teorema}

% Problemas similares a la serie discreta anterior ofrece calcular la integral, desde cero hasta infinito, de una función oscilante, que no decrece, del tipo $\int_0^{\infty} \sin(\omega x) \,dx$.

% En el espíritu del caso discreto, proponemos la siguiente definición.

% \begin{defi}
% Definimos una \textbf{integral de Cesàro} de la siguiente manera:
% \begin{equation}
%     ^* \int_0^{\infty} f(t) \,dt = \lim_{y \to \infty } \frac{1}{y} \left\{ \int_0^y \int_0^x f(t) \,dt dx \right\}. \label{IntegralCesaro1}
% \end{equation}

% \end{defi}


% Podemos encontrar una expresión alternativa para la integral de Cesàro integrando por partes la ecuación \eqref{IntegralCesaro1}:
% \begin{align*}
%     ^* \int_0^{\infty} f(t) \,dt &= \lim_{y \to \infty } \frac{1}{y} \left\{ \int_0^y \int_0^x f(t) \,dt dx \right\} \\
%     &= \lim_{y \to \infty } \frac{1}{y} \left\{\left. x \int_0^x f(t)\,dt \right|_0^y - \int_0^y x f(x) \,dx \right\} \\
%     &= \lim_{y \to \infty } \frac{1}{y} \left\{ y \int_0^y f(t) \,dt - \int_0^y xf(x) \,dx \right\}
% \end{align*}
% \begin{equation}
%     \Rightarrow ~  \boxed{^* \int_0^{\infty} f(t) \,dt = \lim_{y \to \infty} \int_0^y \left( 1 - \frac{x}{y} \right) f(x) \,dx} \label{IntegralCesaro2}
%  \end{equation}
 
%  \begin{ejemplo}
% Evalúe la integral de Cesàro de la función $f(x) = \sin(\omega x)$ con $\omega \neq 0$.
% \\

% \textbf{Solución:} Usando la ecuación \eqref{IntegralCesaro1}, obtenemos que
% \begin{align*}
%       ^* \int_0^{\infty} \sin(\omega t) \,dt &= \lim_{y \to \infty } \frac{1}{y} \left\{ \int_0^y \int_0^x \sin(\omega t) \,dt dx \right\} \\
%       &= \lim_{y \to \infty } \frac{1}{y} \int_0^y \left[ \frac{1 - \cos(\omega x)}{\omega}\right] \, dx \\
%       &= \lim_{y \to \infty} \left[ \frac{1}{\omega} - \frac{1}{\omega^2} \frac{\sin(\omega y)}{y}\right]
% \end{align*}
% \begin{equation}
%     \Rightarrow ~  \boxed{^* \int_0^{\infty} \sin(\omega t) \,dt = \frac{1}{\omega}} \label{CesaroSeno}
%  \end{equation}

%  \end{ejemplo}
 
%   \begin{ejemplo}
% Evalúe la integral de Cesàro de la función $f(x) = \cos(\omega x)$ con $\omega \neq 0$.
% \\

% \textbf{Solución:} Usando la ecuación \eqref{IntegralCesaro1}, obtenemos que
% \begin{align*}
%       ^* \int_0^{\infty} \cos(\omega t) \,dt &= \lim_{y \to \infty } \frac{1}{y} \left\{ \int_0^y \int_0^x \cos(\omega t) \,dt dx \right\} \\
%       &= \lim_{y \to \infty } \frac{1}{y} \int_0^y \frac{\sin(\omega x)}{\omega} \, dx \\
%       &= \lim_{y \to \infty} \left[ \frac{1}{\omega^2 y} - \frac{\cos(\omega y)}{\omega^2 y} \right]
% \end{align*}
% \begin{equation}
%     \Rightarrow ~  \boxed{^* \int_0^{\infty} \cos(\omega t) \,dt = 0 } \label{CesaroCoseno}
%  \end{equation}

%  \end{ejemplo}
\chapter{Análisis de Fourier}

En el curso Física Matemática I ya se discutió el estudio de la Serie de Fourier. En este curso, haremos un rápido resumen de dichos contenidos, pues son la base para introducir el concepto de la \emph{transformada de Fourier}, que será de utilidad para la resolución de algunas ecuaciones diferenciales parciales cuyas condiciones de borde son periódicas.

\section{Periodicidad y paridad de funciones}

% \subsection{Funciones periódicas}

\begin{defi} \marginnote{Función periódica}
Una función $f: \mathbb{R} \to \mathbb{C} $ se dice que es \textbf{periódica de período} $T$, con $T\neq 0$, si 
\begin{equation} \label{Periodica}
    \boxed{
f(t) = f(t + T), \quad \forall \ t \in \mathbb{R}.}    
\end{equation}



La constante $T$ la tomaremos como la  menor constante positiva que satisface la igualdad \eqref{Periodica}.
\end{defi}


\begin{propiedad} 
    \textbf{Propiedades de las funciones periódicas.}
    \begin{enumerate}
        \item Si $f$ es periódica de periodo $T$, entonces $$f(t) = f(t + nT), \quad n = 0, \pm 1, \pm 2, \dots$$
        
        \item Si $f(t)$ y $g(t)$ son funciones periódicas de período $T$, entonces la función
        $$h(t) = \alpha f(t) + \beta g(t); \quad \alpha, \beta \in \mathbb{C},$$
        tiene el mismo período $T$.
    
        \item En general, si la función 
        $$f(t) = \cos (\omega_1 t) + \cos (\omega_2 t)$$
        es periódica de período $T$, entonces es posible encontrar dos enteros $n$ y $m$ tales que 
        \begin{align}
            \omega_1 T &= 2\pi n,  \label{Periodica1}\\
             \omega_2 T &= 2\pi m. \label{Periodica2}
        \end{align}
        
        El cociente de \eqref{Periodica1} y \eqref{Periodica2} es
        $$\frac{\omega_1}{\omega_2} = \frac{n}{m} \ ,$$
        es decir, la relación $\omega_1/ \omega_2$ debe ser un número racional.
    \end{enumerate}
\end{propiedad}


\begin{ejemplo}
Encuentre el período de la función $f(t) = \cos \left(\frac{t}{3}\right) + \cos \left(\frac{t}{4}\right)$.

\textbf{Solución:} Si la función $f(t)$ es periódica con período $T$, entonces, de \eqref{Periodica},
$$\cos \frac{1}{3}(t + T) + \cos \frac{1}{4}(t + T) = \cos \frac{t}{3} + \cos \frac{t}{4}.$$

Como $\cos(\theta + 2\pi n) = \cos \theta, n \in \mathbb{Z}$, obtenemos que 
$$\frac{1}{3} T = 2\pi n, \quad \frac{1}{4}T = 2\pi m; \quad n,m \in \mathbb{Z}.$$

Por consiguiente $T = 6\pi n = 8\pi m$; cuando $n = 4$ y $m=3$, se obtiene el mínimo valor de $T$. Así, $T = 24\pi$.
\end{ejemplo}

\begin{defi} \marginnote{Función seccionalmente continua}
    Una función $f: [a,b] \longrightarrow \mathbb{C}$ es \textbf{seccionalmente continua} si $[a,b]$ tiene una partición finita $a = t_0 < t_1 < \cdots < t_n = b$ tal que $f$ es continua y acotada en cada intervalo abierto $(t_i, t_{i+1}), i = 0, \dots, n-1$.
    
    Denotaremos por $\mathscr{C}[a,b]$ al conjunto de las funciones complejas seccionalmente continuas.
\end{defi}


\begin{propo}
Sea $f: \mathbb{R} \longrightarrow \mathbb{C}$ una función periódica de período $T$. Sea $a \in \mathbb{R}$, entonces
$$ \int_{a-T/2}^{a + T/2} f(t) \,dt = \int_{- T/2}^{T/2} f(t) \,dt .$$
\end{propo}

\begin{demo}
Utilizando la propiedad de aditividad de las integrales,
\begin{equation*}
    \int_{a-T/2}^{a + T/2} f(t) \,dt = \int_{a - T/2}^{-T/2} f(t) \,dt + \int_{- T/2}^{a + T/2} f(t) \,dt.
\end{equation*}

Haciendo la sustitución  $t = t' - T ~\Rightarrow~ dt = dt'$ en la primera integral, obtenemos
\begin{align*}
   \int_{a - T/2}^{-T/2}f(t) \,dt + \int_{- T/2}^{a + T/2} f(t) \,dt   &= \int_{a + T/2}^{T/2} f(t'-T) \,dt' + \int_{- T/2}^{a + T/2} f(t) \,dt \\
   &= \int_{a + T/2}^{T/2} f(t'-T + T) \,dt' + \int_{- T/2}^{a + T/2} f(t) \,dt \\
   &= \int_{a + T/2}^{T/2} f(t') \,dt' + \int_{-T/2}^{a + T/2} f(t) \,dt \\
   &= \int_{-T/2}^{T/2} f(t) \,dt.
\end{align*}
\end{demo}

\begin{defi}\marginnote{Extensión periódica}
Sea $f: [a,b] \rightarrow \mathbb{R}$ seccionalmente continua, se llama \textbf{extensión periódica} de $f$ a la función $f_e: \mathbb{R} \rightarrow \mathbb{R}$,
\begin{equation} 
    \boxed{f_e(t) = f(t + k_0 (b-a))} \ , 
\end{equation}
donde $k_0 \in \mathbb{Z}$ es el único entero que verifica $t + k_0(b-a) \in [a,b].$
\end{defi}

\begin{ejemplo}
La extensión periódica de $f \in \mathscr{C}[-\pi,\pi]$ real es
$$f_e(t) = f_e(t + 2\pi)$$

\begin{figure}[H]
    \centering
    \includegraphics[scale = 0.45]{Figuras/Periocidad.pdf}
    \caption{Extensión periódica de una función real seccionalmente continua en $[-\pi,\pi]$.}
\end{figure}
\end{ejemplo}

% \subsection{Funciones pares e impares}
 
\begin{defi} \marginnote{Funciones pares e impares}
Sea $f: [-a,a] \longrightarrow \mathbb{R}$ perteneciente a $\mathscr{C}[-a,a]$.
Diremos que $f$ es una \textbf{función par} si y solo si, para todo $x$ en el intervalo $[-a,a]$, se cumple que
\begin{equation}
    f(-t) = f(t) \ .
\end{equation}
De forma similar, diremos que $f$ es una \textbf{función impar} si y solo si, para todo $x$ en el intervalo $[-a,a]$, se cumple que
\vspace{-0.1cm}
\begin{equation}
     f(-t) = -f(t) \ .
\end{equation}
\end{defi} 

\begin{propo}
    Sea $f: [-a,a] \longrightarrow \mathbb{R}$ integrable,
    \begin{align*}
        f ~\mbox{es par} &\Rightarrow \int_{-a}^a f(t) \,dt = 2 \int_0^a f(t) \,dt. \\
        f ~\mbox{es impar} &\Rightarrow \int_{-a}^a f(t) \,dt = 0.
    \end{align*}
\end{propo}

\begin{obs}{Observación}
    Toda función $f:[-a,a] \longrightarrow \mathbb{R}$ puede expresarse como la suma de una función par más otra impar: $f = f_p + f_i$ con 
    \begin{equation*}
        f_p(t) = \frac{f(t) + f(-t)}{2}, \quad f_i(t) = \frac{f(t) - f(-t)}{2} \ .
    \end{equation*}
\end{obs}

\begin{defi} \marginnote{Extensión par e impar}
Sea $f \in \mathscr{C}[0,a]$ real, entonces la \textbf{extensión par} y la \textbf{extensión impar} de $f$ están definidas, respectivamente, por:
\begin{equation*}
    E_f(t) = \left\{ \begin{array}{cll}
    f(-t)     & \mbox{si} & -a \leq t < 0 \\
    f(t)     & \mbox{si} & 0 \leq t \leq a
    \end{array} \right. , ~ O_f(t) = \left\{ \begin{array}{cll}
    -f(-t)     & \mbox{si} & -a \leq t < 0 \\
    f(t)     & \mbox{si} & 0 \leq t \leq a
    \end{array} \right. .
\end{equation*}
Ambas extensiones se encuentran definidas en el intervalo $[-a,a]$.
\end{defi}

\begin{figure}[H]
    \centering
    \includegraphics[width = \textwidth]{Figuras/Paridad.pdf}
    \caption{Extensión par e impar de una función real seccionalmente continua en $[0,a]$. }
\end{figure}

\section{Serie de Fourier trigonométrica}

% \subsection{Definición}

\begin{propo}
    En el espacio $\mathscr{C}[a,b]$, el conjunto formado por las funciones
    $$\left\{ 1, \cos\left( \frac{2n \pi}{T}x \right), \sin\left( \frac{2n \pi}{T}x \right) \right\}_{n=1}^{\infty}$$
    es un conjunto ortogonal, con $T = b-a$ el periodo de la función.
\end{propo}


\begin{defi} \marginnote{Sistema trigonométrico}
    Llamamos \textbf{sistema trigonométrico} al conjunto de funciones ortonormales en el espacio $\mathscr{C}[-\pi,\pi]$, definido como
    $$\left\{ \frac{1}{\sqrt{2\pi}}, \frac{\cos(nt)}{\sqrt{\pi}}, \frac{\sin(nt)}{\sqrt{\pi}} \right\}_{n=1}^{\infty}$$
\end{defi}

\begin{defi} \marginnote{Condiciones de Dirichlet}
    Una función $f$ satisface las llamadas \textbf{Condiciones de Dirichlet} si satisface
    \begin{enumerate}
        \item Se encuentra definida en un intervalo $(a,a+T)$.
        \item Tanto $f$ como su derivada son funciones seccionalmente continuas en el intervalo $(a,a+T)$.
        \item $f$ tiene un número finito de discontinuidades \emph{finitas}.
        \item $f$ es una función periódica de periodo $T$.
    \end{enumerate}

\end{defi}

\begin{defi} \marginnote{Serie de Fourier}
Sea $f \in \mathscr{C}[a, a+T]$ una función que satisface las condiciones de Dirichlet. Entonces, ella puede ser aproximada por la serie 
\begin{equation}
    \frac{a_0}{2} + \sum_{n=1}^{\infty} \left( a_n \cos\left( \frac{2n\pi}{T}x \right) + b_n \sin\left( \frac{2n\pi}{T}x \right) \right) \approx f(x) \ . \label{FourierTrigo}
\end{equation}

Esta expansión se denomina \textbf{serie trigonométrica de Fourier} o simplemente \textbf{serie de Fourier}, donde los \textit{coeficientes de Fourier} están dados por:
\begin{align*}
    a_0 &= \frac{2}{T} \int_{a}^{a+T} f(t) \,dt, \\
    a_n &= \frac{2}{T} \int_{a}^{a+T} f(t) \cos\left( \frac{2n\pi}{T}t \right) \,dt, \quad n = 1,2, \dots\\
    b_n &= \frac{2}{T} \int_{a}^{a+T} f(t) \sin\left( \frac{2n\pi}{T}t \right) \,dt. \quad n = 1,2, \dots
\end{align*}
\end{defi}

\begin{ejemplo} \label{EjemploFourier1}
    Consideremos la función $f(x) = x^2$ definida para $x\in [-\pi,\pi]$, la cual es continua con derivada $f'(x) = 2x$ también continua, luego la serie de Fourier de $f$ converge puntualmente a $f$ para todo $x \in (-\pi,\pi)$. Para los extremos $x = \pm \pi$ vemos que $f(\pi) = f(-\pi)$, por lo tanto la serie converge puntualmente a $f$ para todo $x \in [-\pi,\pi]$.
    
    Sus coeficientes de Fourier están dados por:
    \begin{align*}
        a_0 &= \frac{1}{\pi} \int_{-\pi}^{\pi} x^2 \,dx = \left. \frac{x^3}{3\pi} \right|_{-\pi}^{\pi} = \frac{2}{3} \pi^2, \\
        a_n &= \frac{1}{\pi} \int_{-\pi}^{\pi} x^2 \cos(n x)\,dx =   \left. \frac{1}{n\pi} x^2 \sin(nx)  \right|_{-\pi}^{\pi} - \frac{2}{n\pi} \int_{-\pi}^{\pi} x \sin(nx) \,dx\\
        &= \left.   \frac{2}{n^2\pi} x \cos(nx)\right|_{-\pi}^{\pi} - \frac{2}{n^2 \pi} \cancelto{0}{\int_{-\pi}^{\pi} \cos(nx) \,dx }\\
        &=  \frac{4}{n^2} \cos(n\pi) =  (-1)^n \frac{4}{n^2}, \quad n = 1,2,\dots\\
         b_n &= \frac{1}{\pi} \int_{-\pi}^{\pi} x^2 \sin(nx)\,dx = 0, \quad n = 1,2, \dots
    \end{align*}
    
    Entonces, su serie de Fourier es
    \begin{equation}
    f(x) = \frac{\pi^2}{3} + \sum_{n=1}^{\infty} (-1)^n \frac{4}{n^2} \cos(nx), \qquad x \in [-\pi,\pi].    \label{FourierCuadratica}
    \end{equation}
    
    Es claro que la serie de Fourier de $f(x) = x^2$ para todo $x\in \mathbb{R}$ representa la extensión periódica de los valores de $f(x)$ en el intervalo $[-\pi,\pi]$.
    
    La gráfica de $f$ en conjunto con diferentes sumas parciales de su serie de Fourier están representadas en la figura \ref{fig:EjemploFourier1}. 
    
    % \begin{figure}[htb]
        \centering
        \includegraphics[width = 0.8\textwidth]{Figuras/EjemploFourier1.pdf}
        \captionof{figure}{Serie de Fourier de la función $f(x) = x^2, -\pi \leq x \leq \pi$, truncada hasta $n = 4$.}
        \label{fig:EjemploFourier1}
    % \end{figure}
    % El teorema visto para convergencia uniforme nos garantiza que esta serie converge uniformemente a $f(x) = x^2$ en $[-\pi,\pi]$, es más, al aplicar el criterio de M de Weierstrass a la serie, ésta converge para todo $x \in \mathbb{R}$, pues
    % $$\forall  x \in \mathbb{R}: ~ \left|(-1)^n \frac{4}{n^2} \cos(nx)\right| \leq \frac{4}{n^2} = M_n ~~\mbox{y}~~  \sum\limits_{n=1}^{\infty} M_n < \infty.$$
\end{ejemplo}

% \textbf{Observación}: La serie de Fourier de $f$ converge en media a $f$, o sea, 
% \begin{shaded}
% $$f(t) \sim \frac{a_0}{2} + \sum_{n=1}^{\infty} (a_n \cos(nt) + b_n \sin(nt)).$$    
% \end{shaded}

% \begin{propo} \label{C.FourierCero}
% Los coeficientes de la serie trigonométrica de Fourier de $f \in \mathscr{C}[-\pi,\pi]$ convergen a cero cuando $n \to \infty$, es decir,
% $$\lim_{n \to + \infty} a_n = \lim_{n \to + \infty} b_n = 0.$$
% \end{propo}

% \begin{demo}
% Si denotamos el sistema trigonométrico por 
% $$\varphi_0(t) = \frac{1}{\sqrt{2\pi}}, ~ \varphi_{2n-1}(t) = \frac{1}{\sqrt{\pi}} \cos(nt), ~ \varphi_{2n(t)} = \frac{1}{\sqrt{\pi}} \sin(nt), \quad n = 1,2, \dots$$

% tenemos que la serie generalizada de Fourier queda
% $$\sum_{n=0}^{\infty} C_n \varphi_n(t) = C_0 \varphi_0(t) + \sum_{n=1}^{\infty} \left[ C_{2n-1} \varphi_{2n-1}(t) + C_{2n} \varphi_{2n}(t) \right] ,$$

% la cual corresponde a la serie trigonométrica de Fourier de $f \in \mathscr{C}[-\pi,\pi]$, donde 
% $$a_0 = \sqrt{\frac{2}{\pi}} C_0, ~ a_n = \frac{C_{2n-1}}{\sqrt{\pi}}, ~ b_n = \frac{C_{2n}}{\sqrt{\pi}}, \quad n = 1,2, \dots$$

% De lo discutido en el primer capítulo, una de las consecuencias de la desigualdad de Bessel \eqref{D.Bessel} es que 
% $$\lim_{n \to + \infty} C_n = \lim_{n \to + \infty} \langle f, \varphi_n \rangle = 0 ~\Rightarrow~ \lim_{n \to \infty} C_{2n-1} = \lim_{n \to \infty} C_{2n} = 0. $$

% Por lo tanto, 
% $$\lim_{n \to + \infty} a_n = \lim_{n \to + \infty} b_n = 0.$$

% \end{demo}

% ¿Convergerá puntual y/o uniformemente la serie de Fourier a $f(t)$? ¿Qué condiciones deben cumplirse?

% Antes de responder estas preguntas, primero justifiquemos que es suficiente trabajar con funciones a valores reales, a pesar de que los siguientes teoremas también son válidos para funciones a valores complejos. 

\subsection{Serie de Fourier de una función compleja de variable real}

Sea $f = u + iv \in \mathscr{C}[a,b]$, su serie de Fourier trigonométrica está dada por \eqref{FourierTrigo} con 
\begin{align*}
    a_0 &= \frac{2}{T} \int_{a}^{b} f(t) \,dt =  \frac{2}{T} \int_a^b u(t) \,dt + \frac{2}{T} \int_a^b v(t) \,dt,\\
    a_n &= \frac{2}{T} \int_a^b f(t) \cos\left( \frac{2n\pi}{T}t \right) \,dt = \frac{2}{T} \int_a^b u(t) \cos\left( \frac{2n\pi}{T}t \right) \,dt + \frac{2}{T} \int_a^b v(t) \cos\left( \frac{2n\pi}{T}t \right) \,dt , \quad n \in \mathbb{N} \\
    b_n &= \frac{2}{T} \int_a^b f(t) \sin\left( \frac{2n\pi}{T}t \right) \,dt = \frac{2}{T} \int_a^b u(t) \sin\left( \frac{2n\pi}{T}t \right) \,dt + \frac{2}{T} \int_a^b v(t) \sin\left( \frac{2n\pi}{T}t \right) \,dt, \quad n \in \mathbb{N}
\end{align*}

Entonces, su serie de Fourier nos queda
\begin{align}
  f(t) & \sim   \left\{ \frac{\real(a_0)}{2} + \sum_{n=1}^{\infty} \left[ \real(a_n) \cos\left( \frac{2n\pi}{T}t \right) + \real\left( \frac{2n\pi}{T}t \right) \sin\left( \frac{2n\pi}{T}t \right) \right] \right\}  \nonumber \\
   &  + i \left\{ \frac{\im(a_0)}{2} + \sum_{n=1}^{\infty} \left[\im(a_n) \cos\left( \frac{2n\pi}{T}t \right) + \im(b_n) \sin\left( \frac{2n\pi}{T}t \right) \right] \right\},
\end{align}

es decir, la serie de Fourier de $f = u + iv$ será dada por
\begin{equation}
    S_F(f) = S_F(u) + i S_F(v) \ ,
\end{equation}
donde $S_F$ representa a una serie de Fourier.


\subsection{Series de senos y cosenos}

\begin{defi} \marginnote{Serie de Fourier seno y serie de Fourier coseno}
    Dadas las extensiones par e impar de una función, $E_f, O_f: [a,b] \to \mathbb{R}$, es posible obtener el desarrollo en serie de Fourier de cada una de estas, que corresponden a los \textbf{desarrollos en serie de Fourier de coseno y de seno} de $f$, respectivamente. Estos son definidos como
    \begin{align*}
        E_f(t) & \sim \frac{a_0}{2}  + \sum_{n=1}^{\infty} a_n \cos\left( \frac{2n\pi}{T}t \right), & ~~\mbox{donde}~~ a_n = \frac{2}{T} \int_a^{b} f(t) \cos\left( \frac{2n\pi}{T}t \right)  dt \ , \\
        O_f(t) & \sim  \sum_{n=1}^{\infty} b_n \sin\left( \frac{2n\pi}{T}t \right), & ~~\mbox{donde} ~~ b_n = \frac{2}{T} \int_a^{b} f(t) \sin\left( \frac{2n\pi}{T}t \right) \ dt \ .
    \end{align*}

    En ambos casos, se ha hecho uso de las propiedades de las funciones pares e impares para hallar los coeficientes de las series.
\end{defi}

% Puesto que $E_f, O_f: [-\pi,\pi] \to \mathbb{R}$ son seccionalmente continuas, se puede obtener el desarrollo en serie de Fourier de estas, los cuales están definidos por: \footnote{La forma de las series seno y coseno, con sus respectivos coeficientes, se obtienen al aplicar las propiedades vistas para las funciones pares e impares.}
% $$ 

% y
% $$ .$$

% Estos son llamados \textbf{desarrollos en serie de Fourier de coseno y de seno de $f$}, respectivamente.

% \subsection{Convergencia puntual y uniforme}

% \begin{defi}
% Sea $f: [a,b] \longrightarrow \mathbb{R}$, para $t_0 \in [a,b]$ definimos
% \begin{align*}
%     f(t_0^+) &= \lim_{t \to t_0^+} f(t), \\
%     f(t_0^-) &= \lim_{t \to t_0^-} f(t),
% \end{align*}

% si existen los límites. 

% Una discontinuidad en $t_0$ tal que $f(t_0^+)$ y $f(t_0^-)$ existen se denomina \textbf{discontinuidad de salto} y $f(t_0^+) - f(t_0^-)$ recibe el nombre de \textbf{salto} de $f$ en $t_0$.
% \end{defi}

% \textbf{Observaciones:} 

% \begin{enumerate}
%     \item La magnitud del salto es $|f(t_0^+) - f(t_0^-)|$.
    
%     \item El salto se anula cuando $f(t_0) = f(t_0^+) = f(t_0^-)$, es decir, cuando $f$ es continua en $t_0$.
    
%     \item Una función $f: [a,b] \longrightarrow \mathbb{R}$ seccionalmente continua tiene discontinuidades de salto.
% \end{enumerate}

% \begin{defi}
% Sea $f:[a,b] \longrightarrow \mathbb{R}$, con una discontinuidad de salto en $t_0 \in [a,b]$, definimos la \textbf{derivada por la derecha} como 
% $$f'(t_0^+) = \lim_{h \to 0^+} \frac{f(t_0 + h ) - f(t_0^+)}{h}$$

% cuando el límite existe. Similarmente, definimos la \textbf{derivada por la izquierda} como 
% $$f'(t_0^-) = \lim_{h \to 0^-} \frac{f(t_0 + h ) - f(t_0^-)}{h}$$

% cuando el límite existe.
% \end{defi}

% \begin{teorema}[Convergencia puntual de la serie de Fourier] \label{Puntual}
% Sea $f(t)$ una función real seccionalmente continua en el intervalo $-\pi < t < \pi$. Su serie de Fourier trigonométrica converge al valor medio
% \vspace{-0.05cm}
% $$\frac{f(t^+) + f(t^-)}{2}$$

% para cada $t \in (-\pi,\pi)$ donde ambas derivadas laterales $f'(t^+)$ y $f'(t^-)$ existen.
% \end{teorema}

% \textbf{Observación:} Si denotamos por $f_e$ a la extensión periódica de $f$, a partir del teorema anterior, la expansión en serie de Fourier converge a $f_e$ para todo $x \in \mathbb{R}$ al extenderla periódicamente al valor medio
% $$\frac{f_e(t^+) + f_e(t^-)}{2}.$$

% De hecho, en los extremos $t = \pm \pi$, la serie converge a 
% $$\frac{f(-\pi^+) + f(\pi^-)}{2}.$$

% En efecto, observemos que
% $$f_e(-\pi^+) = f(-\pi^+) ~~\mbox{y}~~ f_e(-\pi^-) = f(\pi^-).$$

% Luego, el cociente
% $$\frac{f_e(-\pi^+) + f_e(-\pi^-)}{2} = \frac{f(-\pi^+) + f(\pi-)}{2}.$$

% Análogamente para $t = \pi$.

% \begin{teorema}[Convergencia uniforme] \label{C.Uniforme}
% Supóngase que $f$ es continua en $[-\pi,\pi]$, $f(-\pi) = f(\pi)$ y que $f'$ es continua por tramos, con discontinuidades de salto. Entonces la serie de Fourier trigonométrica de $f$ converge a $f$ absolutamente y uniformemente.
% \end{teorema}

% \subsection{Ejemplos}



% Podemos usar la expansión en serie de Fourier de $f(x) = x^2$ en $[-\pi,\pi]$ para probar que 
% $$\sum_{n=1}^{\infty} \frac{1}{n^2} = 1 + \frac{1}{4} + \frac{1}{9} + \cdots = \frac{\pi^2}{6}.$$

% En efecto, al evaluar $x = \pi$ en \eqref{FourierCuadratica}, obtenemos que 
% $$f(\pi) = \frac{\pi^2}{3} + \sum_{n=1}^{\infty} (-1)^n \frac{4}{n^2} \cos(n\pi) = \frac{\pi^2}{3} + \sum_{n=1}^{\infty} (-1)^{2n} \frac{4}{n^2} = \frac{\pi^2}{3} +  \sum_{n=1}^{\infty} \frac{4}{n^2}.$$

% Así,
% $$\pi^2 = \frac{\pi^2}{3} + 4 \sum_{n=1}^{\infty} \frac{1}{n^2} \Rightarrow \sum_{n=1}^{\infty} \frac{1}{n^2} = \frac{\pi^2}{6}.$$

\begin{ejemplo} \label{Signo}
Consideremos la función signo  definida por
$$f(x) := \left\{ \begin{array}{cc}
     -1,& - \pi \leq x < 0  \\
     1,&   0 \leq x \leq \pi
\end{array} \right. .$$

La función es seccionalmente continua con $x = 0$ punto de discontinuidad de salto y las derivadas laterales existen para todo $x \in (-\pi,\pi)$, luego la serie de Fourier de $f$ converge puntualmente a $f$ en los puntos de continuidad y a 
$$\frac{f(0^-) + f(0^+)}{2} = 0, \quad \mbox{en} ~ x = 0 ~~\mbox{y}$$
$$\frac{f(-\pi^+) + f(\pi^-)}{2} = 0, \quad \mbox{en} ~ x = \pm \pi. $$

Sus coeficientes de Fourier están dados por:
\begin{align*}
    a_0 &= \frac{1}{\pi} \int_{-\pi}^{\pi} f(x) \,dx = 0 , \\
    a_n &= \frac{1}{\pi} \int_{-\pi}^{\pi} f(x) \cos(n x)\,dx = 0, \quad n = 1,2,\dots
\end{align*}
\begin{align*}
     b_n &= \frac{1}{\pi} \int_{-\pi}^{\pi} f(x) \sin(nx) \,dx \\
     &= \frac{1}{\pi} \int_{-\pi}^0 (-1) \sin(nx)\,dx + \frac{1}{\pi} \int_{0}^{\pi} (1) \sin(nx) \,dx \\
     &= \left.  \frac{1}{\pi n} \cos(nx) \right|_{-\pi}^0 - \left. \frac{1}{\pi n} \cos(nx) \right|_{0}^{\pi} \\
     &= \frac{2}{\pi n} [1 - (-1)^n] \\
     &= \left\{ \begin{array}{cl}
         0, & n ~\mbox{par}  \\
         \frac{4}{\pi n}, &  n ~\mbox{impar}
     \end{array} \right. .
\end{align*}

Entonces, su serie de Fourier es 
$$f(x) =  \sum_{n ~impar} \frac{4}{\pi n} \sin(nx) = \sum_{k=1}^{\infty} \frac{4}{\pi} \frac{\sin[(2k-1)x]}{ (2k-1)}.$$

\textbf{Aclaración:} Note que a pesar de haber escrito que la función $f$ es igual a la serie, debemos tener en cuenta que en los punto $x = 0$ y $x = \pm \pi$ converge al valor medio del salto de la discontinuidad.

Es claro que la serie de Fourier de $f$ para todo $x\in \mathbb{R}$ representa la extensión periódica de los valores de $f(x)$ en el intervalo $[-\pi,\pi]$.

La gráfica de $f$ en conjunto con diferentes sumas parciales de su serie de Fourier están representadas en la figura \ref{fig:EjemploFourier2}.

\begin{figure}[H]
    \centering
    \includegraphics[scale = 0.65]{Figuras/EjemploFourier2.pdf}
    \caption{Serie de Fourier de la función signo truncada hasta $n = 4$.}
     \label{fig:EjemploFourier2}
\end{figure}

\end{ejemplo}

\section{Serie exponencial}

\begin{propo}
    En el espacio $\mathscr{C}[a,a+T]$, el conjunto formado por las funciones 
    \begin{equation}
        \left\{ \frac{1}{\sqrt{T}} \exp\left(i\frac{2n\pi}{T}x\right) \right\}_{n= - \infty}^{n = \infty}
    \end{equation}
    es un conjunto ortonormal.
\end{propo}

\begin{defi} \marginnote{Sistema exponencial}
    Llamamos \textbf{sistema exponencial} al conjunto de funciones ortonormales en el espacio $\mathscr{C}[-\pi,\pi]$, definido como 
    $$\left\{ \frac{1}{\sqrt{2\pi}} e^{int} \right\}_{n= - \infty}^{n = \infty}$$
\end{defi}

\begin{defi} \marginnote{Serie de Fourier exponencial}
Sea $f \in \mathscr{C}[a,a+T]$ una función con un número finito de discontinuidades. Entonces, ella puede ser aproximada por la serie 
\begin{equation}
     \sum_{n=- \infty}^{\infty} c_n \exp\left(i\frac{2n\pi}{T}x\right) \label{FourierExpo}
\end{equation}

Esta expansión se denomina \textbf{serie exponencial de Fourier}  donde los \textit{coeficientes de Fourier} están dados por:
\begin{equation*}
    c_n = \frac{1}{T} \int_{a}^{a+T} f(t) \exp\left(-i\frac{2n\pi}{T}t \right) \ dt \ .
\end{equation*}
\end{defi}

% \textbf{Observación}: La serie de Fourier de $f$ converge en media a $f$, o sea, 
% \begin{shaded}
%  $$f(t) \sim \sum_{n=- \infty}^{\infty} c_n e^{int}.$$   
% \end{shaded}

\begin{propo} \label{TrigoExpo}
La $n$-ésima suma parcial de la serie de Fourier trigonométrica de una función (real o compleja) es igual a la $n$-ésima suma parcial de la serie exponencial.
\end{propo}

\begin{demo}
La $n$-ésima suma parcial de la serie exponencial es
$$ s_n(t) = \sum_{k=-n}^n c_k e^{ikt}.$$

Separando la suma:
\begin{align*}
    s_n(t) &= \sum_{k=-n}^{-1} c_k e^{ikt} + c_0 + \sum_{k=1}^n c_k e^{ikt} \\
    &= c_0 + \sum_{k=1}^n c_k e^{ikt} + \sum_{k=1}^n c_{-k} e^{-ikt} \\
    &= c_0 + \sum_{k=1}^n [c_k e^{ikt} + c_{-k} e^{-ikt}]. 
\end{align*}

Usando la identidad de Euler, $e^{i\theta} = \cos(\theta) + i \sin(\theta)$, encontramos que
$$s_n(t) = c_0 +  \sum_{k=1}^n [(c_k + c_{-k}) \cos(kt) + i(c_k - c_{-k}) \sin(kt)].$$

Desarrollando los coeficientes de la serie exponencial de Fourier, tenemos que
\begingroup
\allowdisplaybreaks
\begin{align*}
    c_0 &= \frac{1}{2\pi} \int_{-\pi}^{\pi} f(t) \,dt, \\
    c_k + c_{-k} &= \frac{1}{2\pi} \int_{-\pi}^{\pi} f(t) e^{-ikt} \,dt + \frac{1}{2\pi} \int_{-\pi}^{\pi} f(t) e^{ikt} \,dt  \\
    &= \frac{1}{2\pi} \int_{-\pi}^{\pi} f(t) [e^{ikt} + e^{-ikt}] \,dt \\
    &= \frac{1}{\pi} \int_{-\pi}^{\pi} f(t) \cos(kt) \,dt; \quad k = 1,2, \dots\\
   i( c_k - c_{-k}) &= \frac{i}{2\pi} \int_{-\pi}^{\pi} f(t) e^{-ikt} \,dt - \frac{i}{2\pi} \int_{-\pi}^{\pi} f(t) e^{ikt} \,dt \\
   &= - \frac{i}{2\pi} \int_{-\pi}^{\pi} f(t) [e^{ikt} - e^{-ikt}] \,dt \\
   &= \frac{1}{\pi} \int_{-\pi}^{\pi} f(t) \sin(kt)\,dt; \quad k = 1,2, \dots
\end{align*}
\endgroup

Comparando las expresiones obtenidas con los coeficientes de la serie de Fourier trigonométrica, podemos concluir que 
$$c_0 = \frac{a_0}{2}, ~  c_k + c_{-k} = a_k, ~ i( c_k - c_{-k}) = b_k; \quad k = 1,2, \dots$$

Por lo tanto, 
$$ s_n(t) = \sum_{k=-n}^n c_k e^{ikt} = \frac{a_0}{2} + \sum_{k=1}^n (a_k \cos(kt) + b_k \sin(kt)).$$

\end{demo}
Una consecuencia inmediata de la proposición \ref{TrigoExpo} es que todos los teoremas vistos para la serie de Fourier trigonométrica son aplicables a la serie de Fourier exponencial.

\begin{propiedad} 
    \textbf{Propiedades de la Serie de Fourier exponencial}
    \begin{enumerate}
        \item Los coeficientes de las series \eqref{FourierTrigo} y \eqref{FourierExpo} están relacionados por 
        \begin{equation}
        a_0 = 2c_0,~~ a_n = c_n + c_{-n}, ~~ b_n = i(c_n - c_{-n}); \quad n = 1,2, \dots    \label{RelacionCoefi1}
        \end{equation}
        
        o bien, 
        \begin{equation}
            c_n = \left\{ \begin{array}{cl}
                \frac{1}{2} (a_n - ib_n), & n \geq 0  \\
            \frac{1}{2}(a_{-n} + i b_{-n}),     & n  \leq -1 
            \end{array} \right. . \label{RelacionCoefi2}
        \end{equation}

        \item Si $f(t)$ es una función real, entonces sus respectivos coeficientes complejos $c_n$ satisfacen la relación:
        $$c_n^* = \frac{1}{2\pi} \int_{-\pi}^{\pi} f(t) (e^{-int})^* \,dt = \frac{1}{2\pi} \int_{-\pi}^{\pi} f(t) e^{int} \,dt = c_{-n}\ .$$
    \end{enumerate}
\end{propiedad}




% \section{Diferenciación e integración de las series de Fourier}

% Supongamos que tenemos una serie de funciones
% $$\sum_{n=1}^{\infty} f_n(t),$$

% y queremos integrarla (o derivarla). Resulta tentador intercambiar la integral (o derivada) por la serie, es decir, integrar (o derivar) término a término. Sin embargo, este intercambio no es siempre posible porque puede romper la convergencia de la serie. Por ejemplo, las series de potencia se pueden integrar (o derivar) término a término en su región de convergencia sin problema. En nuestro caso, nos interesa saber que condiciones deben cumplirse para las series de Fourier.

% \begin{teorema}[Integración]
% Sea $f$ una función seccionalmente continua en el intervalo $-\pi < t < \pi$. Independiente si la serie \eqref{FourierTrigo} converge, la siguiente ecuación es válida cuando $-\pi \leq t \leq \pi$:
% \begin{equation*}
%   \int_{-\pi}^t f(s) \,ds = \frac{a_0}{2} (t + \pi) + \sum_{n=1}^{\infty} \frac{1}{n} \left\{ a_n \sin(nt) - b_n[\cos(nt) + (-1)^{n+1}] \right\}.   
% \end{equation*}
% \end{teorema}


% \begin{teorema}[Derivación]
% Sea $f$ una función continua en $[-\pi,\pi]$, donde $f(-\pi) = f(\pi)$, y $f'$ es seccionalmente continua en el intervalo $(-\pi,\pi)$. Entonces, la serie de Fourier
% \begin{equation*}
%     f(t) = \frac{a_0}{2} + \sum_{n=1}^{\infty} (a_n \cos(nt) + b_n \sin (nt) ), \quad t \in [-\pi,\pi],
% \end{equation*}

% donde 
% $$a_n = \frac{1}{\pi} \int_{-\pi}^{\pi} f(t) \cos(nt) \,dt, \quad b_n = \frac{1}{\pi} \int_{-\pi}^{\pi} f(t) \sin(nt) \,dt,$$

% es derivable para todo $t \in (-\pi,\pi)$ en el cual $f''(t)$ existe:
% \begin{equation*}
%     f'(t) = \sum_{n=1}^{\infty} (- n a_n \sin(nt) + nb_n \cos(nt)).
% \end{equation*}
% \end{teorema}


% \begin{ejemplo}
% Obtener el desarrollo en serie de Fourier de la función $x^3$ en el intervalo $[-\pi,\pi]$.
% \\

% \textbf{Solución}: Las funciones del tipo $x^n$, $n \in \mathbb{N}$ son continuas, derivables e integrables, por lo que podemos aplicar estas operaciones a las series de Fourier que se obtengan a partir de ellas.  En nuestro caso obtendremos el desarrollo de $x^3$ por integración de $x^2$, cuyo desarrollo en serie de Fourier ya obtuvimos en el ejemplo \ref{EjemploFourier1}:
% \begin{align*}
%    \forall x \in [-\pi,\pi]:  \int_{-\pi}^x t^2 \,dt &=   \int_{-\pi}^x   \frac{\pi^2}{3} + \sum_{n=1}^{\infty} (-1)^n \frac{4}{n^2} \cos(nt) \,dt\\
%     \Rightarrow ~ \frac{x^3}{3} + \frac{\pi^3}{3} &=  \frac{\pi^2}{3} (x+\pi) + \sum_{n=1}^{\infty} (-1)^n \frac{4 }{n^3}  \sin(nx)  \\
% \Rightarrow \qquad \quad    x^3 &= \pi^2 x + 12 \sum_{n=1}^{\infty} \frac{(-1)^n}{n^3} \sin(nx).
% \end{align*}

% Lo que obtuvimos es el desarrollo en serie de $x^3-\pi^2 x$. Para obtener el desarrollo de $x^3$, encontremos la serie de Fourier de $x$, para ello utilicemos el desarrollo de $x^2$, pero esta vez derivando:
% \begin{align*}
%     \frac{d}{dx} [x^2] &= \frac{d}{dx} \left[  \frac{\pi^2}{3} + \sum_{n=1}^{\infty} (-1)^n \frac{4}{n^2} \cos(nx)\right] \\
%    \Rightarrow \qquad 2x &= - \sum_{n=1}^{\infty} (-1)^n \frac{4}{n} \sin(nx) \\
%    \Rightarrow \qquad ~ x &= - 2 \sum_{n=1}^{\infty}  \frac{(-1)^n}{n} \sin(nx).
% \end{align*}

% Sustituyendo en $x^3$:
% \begin{align*}
%     x^3 &= -2 \pi^2 \sum_{n=1}^{\infty}  \frac{(-1)^n}{n} \sin(nx) + 12 \sum_{n=1}^{\infty} \frac{(-1)^n}{n^3} \sin(nx) \\
%     &= \sum_{n=1}^{\infty} \frac{(-1)^n}{n^3} \left[ 12 - 2\pi^2 n^2 \right] \sin(nx).
% \end{align*}
% \end{ejemplo}

% \section{Fenómeno de Gibbs}

% Cuando queremos aproximar una función $f$ con las sumas parciales de su serie de Fourier, se puede observar un comportamiento particular cerca de los puntos de discontinuidad aislados, por ejemplo, al graficar la serie de Fourier de la función signo para un $n$ alto, ver figura \ref{fig:GibssSign}, se comete un error considerable en las cercanías al punto de discontinuidad, y además este
% error no disminuye si se incluyen más términos en la serie. Este efecto indeseado se denomina \textbf{fenómeno de Gibbs}. 

% \begin{figure}[H]
%     \centering \includegraphics[scale = 0.65]{Figuras/FenomenoGibbs.pdf}
%     \caption{Fenómeno de Gibbs en la función signo para $n = 10,30,50$.}
%     \label{fig:GibssSign}
% \end{figure}

% Ilustremos este hecho con un análisis analítico de la función ya estudiada
% $$f(x) = \left\{ \begin{array}{cc}
%      -1,& - \pi \leq x < 0  \\
%      1,&   0 \leq x \leq \pi
% \end{array} \right. .$$

% Su serie de Fourier está dada por
% $$\sum_{n=1}^{\infty} \frac{4}{\pi} \frac{\sin[(2n-1)x]}{ (2n-1)}.$$

% Sea 
% $$s_n(x) = \frac{4}{\pi} \sum_{k=1}^n \frac{\sin[(2k-1)x]}{2k-1}$$

% su $n$-ésima suma parcial. Derivando y multiplicando por $\pi \sin(x)$, encontramos que 
% \begin{align*}
%    \pi (\sin x)s_n'(x) =4 \sin(x) \sum_{k=1}^n \cos[(2k-1)x] &= 4 \sum_{k=1}^n   \sin(x) \cos[(2k-1)x] \\
%    &= 4 \sum_{k=1}^n  \frac{1}{2}[\sin(x + (2k-1)x) + \sin(x - (2k-1)x)] \\
%    &= 2 \sum_{k=1}^n [\sin(2k x) - \sin([2k-2]x)] \\
%    &= 2 \sin(2nx).
% \end{align*}

% Luego, 
% \begin{equation*}
%   2 \sin(2nx_c) = 0 ~\Leftrightarrow~ x_c = \frac{m \pi}{2n}, \quad m \in \mathbb{Z}. 
% \end{equation*}

% Nos interesa encontrar el primer máximo de $s_n(x)$, así que tomaremos los valores $m = \pm 1$ de prueba, pues en $m = 0$ no se alcanza un máximo dado que $s_n(0) = 0$. Como $\sin(x_c) \neq 0$, para estos valores de $m$,  $s_n'(x_c) = 0$ y, en consecuencia, $x_c$ son los puntos críticos de $s_n$. 

% En la siguiente tabla se analizan los signos de $s_n'(x)$.

% \begin{figure}[H]
%     \centering
%     \includegraphics[scale = 0.62]{Figuras/EjemploGibbs.pdf}
%     \caption{Tabla con el análisis de signos de $s_n'(x)$ asociada a la función signo.}
% \end{figure}

% Por el criterio de la primera derivada, vemos que $s_n$ tiene un máximo en $x_n = \frac{\pi}{2n}$. El valor de ese máximo es
% $$s_n \left( \frac{\pi}{2n}\right) = \frac{4}{\pi} \sum_{k=1}^n \frac{\sin[(2k-1)\pi/2n]}{2k-1} = \frac{2}{\pi} \sum_{k=1}^n \frac{\sin[(2k-1)\pi /2n]}{(2k-1) \pi/2n} \left(\frac{\pi}{n} \right).$$

% Notemos que la sumatoria 
% $$ \sum_{k=1}^n \frac{\sin[(2k-1)\pi /2n]}{(2k-1) \pi/2n} \left(\frac{\pi}{n} \right)$$

% es una suma de Riemann para la función $\sin y/y$ en $[0,\pi]$ para la partición regular $0 = x_0 < x_1  <  \cdots< x_n =  \pi$ con $x_i = (\pi/n) i$, $i = 0,1, \dots, n$ y eligiendo el punto medio de cada intervalo $[x_{k-1},x_k]$,
% $$ \frac{x_{k-1} + x_k}{2} = (2k-1) \frac{\pi}{2n}, \quad  k = 1, \dots, n,$$

% para evaluar $\sin y/y$. Entonces, 
% $$\int_0^{\pi} \frac{\sin y}{y} \,dy \approx \frac{\pi}{2} s_n\left( \frac{\pi}{2n}\right).$$

% Como 
% $$\int_0^{\pi} \frac{\sin y}{y} \,dy ~~ \mbox{converge} ~\Rightarrow~ \lim_{n \to + \infty} s_n\left( \frac{\pi}{2n} \right) = \frac{2}{\pi} \int_0^{\pi} \frac{\sin y}{y} \,dy.$$

% Usando un método numérico de integración (o su calculadora de integrales favorita), 
% $$\int_0^{\pi} \frac{\sin y}{y} \,dy \approx 1.85193\dots$$

% Por lo tanto, 
% $$\lim_{n \to + \infty} s_n\left( \frac{\pi}{2n} \right) \approx 1.179.$$

% Así, las aproximaciones exceden el valor real de $f(0^+) = 1$ por $0.179$ o $8.95\%$ del salto de $f(0^-)$ a $f(0^+)$. 

% En general, se puede demostrar el siguiente teorema, debido a M. B$\hat{\mbox{o}}$cher \cite{Bôcher}:

% \begin{teorema}
% Sea $f$ una función de variable real, con período $2\pi$. Supongamos que $f$ y $f'$ son ambas continuas excepto para un número finito de discontinuidades de salto en el intervalo $[-\pi,\pi]$. Sea $s_n(x)$ la suma parcial de orden $n$ de Fourier. Entonces, en un punto $a$ de discontinuidad, las gráficas de las funciones $s_n(x)$ convergen al segmento vectical (ver figura \ref{Gibbs}) de longitud 
% $$L = \frac{2}{\pi} Si(\pi) |f(a^+) - f(a^-)| \approx 1.179 |f(a^+) - f(a^-)|$$

% centrada en el punto 
% $$\left(a, \frac{f(a^+) + f(a^-)}{2} \right),$$

% donde $Si(x)$ es la función seno integral definida por
% $$Si(x) = \int_0^x \frac{\sin t}{t} \,dt.$$
% \end{teorema}

% \begin{figure}[H]
%     \centering
%     \includegraphics[scale = 0.51]{Figuras/Gibbs.pdf}
%     \caption{Fenómeno de Gibbs en un punto de discontinuidad. Adaptado de \cite{UFRO}, pág. 202.}
%     \label{Gibbs}
% \end{figure}
\input{./Cap-Integrales-Convergencia.tex}
\chapter{Transformada de Fourier}

% \section{Definiciones}

En el capítulo anterior, aprendimos que la serie de Fourier de $f \in \mathscr{C}[-L/2,L/2]$ está dada por 
\begin{equation} \label{Transformada1}
  f(x) = \sum_{n=-\infty}^{\infty} c_n e^{i \frac{2n\pi}{L}x} \ ,  
\end{equation}
donde 
\begin{equation} \label{Transformada2}
  c_n = \frac{1}{L} \int_{-L/2}^{L/2} f(x) e^{-i\frac{2n\pi}{L}x} \,dx, \quad n \in \mathbb{Z} \ . 
\end{equation}

Una consecuencia inmediata de la expansión en serie de Fourier es que la función $f(x)$ representada por la serie resulta periódica, con período $L$. Por lo tanto, decimos que la serie de Fourier permite \emph{expandir funciones periódicas}. 

Sin embargo, no todas las funciones son periódicas, y nos interesará expandirlas dentro de algún intervalo de validez. Necesitamos, entonces, algún modo de expandir, en una base ortonormal, funciones no periódicas. 

Podemos decir que el conjunto de coeficientes $\{c_n\}$ también definen a $f(x)$. Este conjunto de números $c_n$ puede ser entendido como una función en la variable $n$, escrita como $c(n)$, definida para un conjunto \emph{discreto} de valores de la variable independiente (en lugar de un intervalo continuo).  
\begin{defi}\marginnote{Espectro de Fourier}
    Se define como el \textbf{espectro de Fourier} a la función de variable discreta $c(n)$, definida a partir de los coeficientes de Fourier \eqref{Transformada2}.
\end{defi}

El espectro de una función puede ser graficado, asumiendo $c(n)$ real, como se observa en la figura \ref{fig:espectro-fourier}.

\vspace{-0.5cm}
\begin{figure}[H]
    \centering
    \includegraphics[width = 12cm]{Figuras/Espectro1.pdf}
    \caption{Espectro de Fourier.}
    \label{fig:espectro-fourier}
\end{figure}

En lugar de graficar $c$ vs $n$, podemos graficar $c$ vs $k$, el \emph{número de onda}, que corresponde a la frecuencia asociada a la parte espacial:
$$k = \frac{2\pi n}{L}.$$

Si $L \to \infty$, entonces las frecuencias se encuentran estrechamente espaciadas debido a que la diferencia entre valores consecutivos de $k$ es
$$\Delta k = \frac{ 2\pi \Delta n}{L}  = \frac{2\pi}{L}, \quad \mbox{pues}~ \Delta n = 1.$$

En otras palabras, para $L \to \infty$, $\Delta k$ es pequeño. Con este cambio de escala, el espectro de Fourier puede parecerse a lo mostrado en la figura \ref{Espectro1}.

\begin{figure}
    \centering
    \includegraphics[scale = 0.4]{Figuras/Espectro2.pdf}
    \caption{Espectro de Fourier cuando $L \to + \infty$.}
    \label{Espectro1}
\end{figure}

Es natural especular sobre la posibilidad de un espectro continuo cuando $L$ tiende al infinito  de tal forma que todas las frecuencias están presentes. Puede ser instructivo considerar la siguiente derivación heurística: Sabemos que una función puede ser expandida como una serie de Fourier tal como se muestra en \eqref{Transformada1}. Luego, la transición $L \to \infty$ puede resultar difícil de realizar directamente ya que $c_n$ aparentemente tiende a cero. Seguimos entonces la idea de usar las frecuencias $k = 2\pi n/L$ tal que
$\Delta k = (2\pi/L ) \Delta n = 2\pi/L$ para valores de $k$ adyacentes y definimos
\begin{equation}
    c_L(k) = \frac{L}{\sqrt{2 \pi}} c_n \ .
\end{equation}

Usando las definiciones anteriores en las ecuaciones \eqref{Transformada1} y \eqref{Transformada2}, obtenemos que la función y sus coeficientes de Fourier se pueden escribir como: 
\begin{align*}
    f(x)&= \sum_{Lk/2\pi = -\infty}^{\infty} \frac{\sqrt{2\pi}}{L} c_L(k) e^{ikx} \left( \frac{\Delta k L}{2\pi}\right) = \sum_{Lk/2\pi = -\infty}^{\infty}  \frac{1}{\sqrt{2\pi}} c_L(k) e^{ikx} \Delta k , \\
  c_L(k) &= \frac{L}{\sqrt{2\pi}} \frac{1}{L} \int_{-L/2}^{L/2} f(x) e^{-ikx} dx = \frac{1}{\sqrt{2\pi}} \int_{-L/2}^{L/2} f(x) e^{-ikx} dx.
\end{align*}

Al hacer $L \to \infty$, la función $f$ puede considerarse como una función no-periódica arbitraria definida en todo el intervalo $(-\infty, \infty)$, mientras que la primera suma ``se convierte'' en una integral:
\begin{align*}
    f(x)& = \frac{1}{\sqrt{2\pi}} \int_{-\infty}^{\infty} c(k) e^{ikx} \,dk, \\
  c(k) & = \lim_{L\to + \infty} c_L(k) = \frac{1}{\sqrt{2\pi}} \int_{-\infty}^{\infty} f(x) e^{-ikx} dx.
\end{align*}

\begin{defi} \marginnote{Transformada de Fourier}
    Dada una función $f$ no periódica definida en $\mathscr{C}(-\infty, \infty)$, definimos su \textbf{transformada de Fourier} como 
    \begin{equation}\label{T.Fourier}
        \boxed{\tilde{f}(k) := \frac{1}{\sqrt{2\pi}} \int_{-\infty}^{\infty} f(x) e^{-ikx} dx \ .} 
    \end{equation}  
\end{defi}

Note que la transformada de Fourier es la extensión natural del concepto de series de Fourier para funciones no periódicas. Además, al ser $n$ una variable discreta, y $k$ continua, podemos decir que la transformada de Fourier es la generalización del concepto de series de Fourier cuando las funciones pertenecen a un espacio vectorial de dimensión continua.

\begin{defi}\marginnote{Transformada inversa de Fourier}
    Se define la \textbf{transformada inversa de Fourier} como
      \begin{equation}\label{I.Fourier}
     \boxed{f(x) = \frac{1}{\sqrt{2\pi}} \int_{-\infty}^{\infty} \tilde{f}(k) e^{ikx} \,dk \ .} 
    \end{equation}  
\end{defi}

\begin{obs}{Observaciones}
    \begin{itemize}
        \item Otras notaciones usadas son: $\tilde{f}(k) = \hat{f}(k) = g(k) = \mathcal{F}\{f(x)\}(k)$.
        
        \item El factor $1/\sqrt{2\pi}$ en la definición \eqref{T.Fourier} es convencional. Lo importante es que se cumpla la identidad conocida como \textbf{integral de Fourier}
        \begin{equation}
            f(x) = \int_{-\infty}^{\infty} \left[\frac{1}{2\pi} \int_{-\infty}^{\infty} f(\xi) e^{-ik\xi} d\xi \right] e^{ikx} \,dk.
          \label{IntegralFourier}
        \end{equation}
       
        % Por ejemplo, en lugar de estos factores, podría introducirse un $\alpha$ en \eqref{I.Fourier} y $1/(2\pi \alpha)$ en \eqref{T.Fourier}, con $\alpha$ una constante arbitraria. Algunas elecciones populares son: $\alpha = 1$ y $\alpha = 1/\sqrt{2\pi}$ \cite{Rubilar}.
    
        \item Al igual que el factor $1/\sqrt{2\pi}$ en la definición \eqref{T.Fourier}, la función $e^{-ikx}$ es convencional y puede ser reemplazada por $e^{ikx}$, siempre y cuando se verifique \eqref{IntegralFourier} \cite{Butkov, Riley}.
        
        % \item En el caso que $f(x)$ sea real, tenemos que \eqref{IntegralFourier} se puede escribir como 
        % \begin{equation}
        %     f(x) = \frac{1}{\pi} \int_{0}^{\infty} \int_{-\infty}^{\infty} f(\xi) \cos k(x-\xi)  \, d\xi \,dk.  \label{IntegralFourierReal}
        % \end{equation}
    
        % \colorlet{shadecolor}{blue!10} 
        % \begin{shaded}
        % En efecto, la relación \eqref{IntegralFourier} también se puede expresar como 
        % $$f(x) = \frac{1}{2\pi}  \int_{-\infty}^{\infty} \int_{-\infty}^{\infty} f(\xi) e^{-ik\xi} e^{ikx} d\xi  \,dk = \frac{1}{2\pi}  \int_{-\infty}^{\infty} \int_{-\infty}^{\infty} f(\xi) e^{ik(x-\xi)} d\xi  \,dk.$$
        
        % Como $f(x)$ es real, se igualan las partes reales para así obtener
        % $$f(x) = \frac{1}{2\pi} \int_{-\infty}^{\infty} \int_{-\infty}^{\infty} f(\xi) \cos k(x-\xi) d\xi  \,dk. $$
        
        % Puesto que $\cos k(x-\xi)$ es par con respecto a $k$, tenemos que 
        % $$f(x) = \frac{2}{2\pi} \int_{0}^{\infty} \int_{-\infty}^{\infty} f(\xi) \cos k(x-\xi)  \, d\xi \,dk =\frac{1}{\pi} \int_{0}^{\infty} \int_{-\infty}^{\infty} f(\xi) \cos k(x-\xi)  \, d\xi \,dk. $$  
        % \end{shaded}
        % \colorlet{shadecolor}{green!20}
        
        \item Es común en Física trabajar con funciones del tiempo, $f = f(t)$. En este caso, se acostumbra usar la frecuencia $\omega$ en lugar del número de onda $k$, de modo que la transformada de Fourier adopta la forma
        $$
        f(t) = \frac{1}{\sqrt{2\pi}} \int_{- \infty}^{\infty} \Tilde{f}(\omega) e^{i \omega t} d\omega,
        $$
    
        donde
        $$
        \Tilde{f}(\omega) = \frac{1}{\sqrt{2\pi}} \int_{- \infty}^{\infty} f(t) e^{- i \omega t} dt.
        $$
        
        \item En 3 dimensiones, la integral de Fourier está dada por:
        \begin{align*}
             f(\vec{r}\,) &:= \frac{1}{(2\pi)^{3/2}} \int_{\mathbb{R}^3} \tilde{f}(\vec{k}) e^{i (\vec{k} \cdot \vec{r})} d^3k, \\
             \tilde{f}(\vec{k}\,) &:= \frac{1}{(2\pi)^{3/2}} \int_{\mathbb{R}^3} f(\vec{r}) e^{-i (\vec{k} \cdot \vec{r})} d^3x.
        \end{align*}
        
        En general, en $n$ dimensiones:
         \begin{align*}
             f(\vec{r}\,) &:= \frac{1}{(2\pi)^{n/2}} \int_{\mathbb{R}^n} \tilde{f}(\vec{k}) e^{i (\vec{k} \cdot \vec{r})} d^n k, \\
             \tilde{f}(\vec{k}\,) &:= \frac{1}{(2\pi)^{n/2}} \int_{\mathbb{R}^n} f(\vec{r}) e^{-i (\vec{k} \cdot \vec{r})} d^n x.
        \end{align*}
       
    \end{itemize}    
\end{obs}


% \textbf{Observaciones:}

¿Cómo aseguramos la existencia de la transformada de Fourier de una función? Para ello, necesitamos introducir el concepto de \emph{funciones absolutamente integrables}, tras lo cual podemos plantear el teorema de existencia de la transformada de Fourier.

\begin{defi} \marginnote{Función absolutamente integrable}
Si $f(x)$ es tal que 
$$\int_{-\infty}^{\infty} |f(x)| \,dx < \infty,$$

entonces se dice que $f \in  L^1$ o que es \textbf{absolutamente integrable}.
\end{defi}

\begin{teorema}
Si $f \in L^1$, entonces la transformada de Fourier $\tilde{f}(k) = \mathcal{F}\{f(x)\}(k)$ existe y $\lim\limits_{k \to \pm \infty} \tilde{f}(k) = 0$.
\end{teorema}

\begin{demo}

Demostraremos solo la primera parte del teorema.

Notemos que
$$e^{-ikx} = \cos(kx) - i \sin(kx) ~\Rightarrow~ |e^{-ikx}| = 1.$$

Luego,
$$ \int_{-\infty}^{\infty} |f(x) e^{-ikx}| dx =  \int_{- \infty}^{\infty} |f(x)| \,dx < \infty.$$

En consecuencia, $f(x) e^{-ikx}$ es absolutamente integral y
$$\frac{1}{2\pi} \int_{-\infty}^{\infty} f(x) e^{-ikx} dx$$

es finita, es decir, $\tilde{f}(k)$ existe. 
\end{demo}

\begin{obs}{Observación}
    La condición de que $f$ sea absolutamente integrable es suficiente pero no necesaria para la existencia de la transformada de Fourier.
\end{obs}

\begin{teorema}
    Sea $f(x)$ una función seccionalmente continua en cada intervalo finito del eje $x$, y supongamos que es absolutamente integrable en $(-\infty, + \infty)$. Entonces la \textbf{integral de Fourier} satisface
\begin{equation}
    \frac{1}{\pi} \int_0^{+\infty} \int_{-\infty}^{+\infty} f(\xi) \cos k(\xi-x) \,d\xi dk = \frac{f(x^+) + f(x^-)}{2} \ ,    
\end{equation}
donde ambas derivadas laterales, $f'(x^+)$ y $f'(x^-)$, existen.
\end{teorema}

\begin{demo}
Consulte el cápitulo 6 <<Fourier Integrals and Applications>> en \cite{Brown}.
\end{demo}

% \section{Ejemplos}

\begin{ejemplo} \label{PulsoCuadrado}
    \textbf{Función pulso cuadrado.} Consideremos la función 
    \begin{equation*}
        f(x) = \left\{ \begin{array}{cl}
            1,& |x|<a  \\
            0,& |x|> a
        \end{array} \right. \ .
    \end{equation*}

    Su transformada de Fourier es 
    \begin{align}
        \tilde{f}(k) & = \frac{1}{\sqrt{2\pi}} \int_{-\infty}^{\infty} f(x) e^{-ikx} dx \nonumber \\
        & = \frac{1}{\sqrt{2\pi}}\int_{-a}^a (1)  e^{-ikx} dx \nonumber \\
        & = \frac{1}{\sqrt{2\pi}} \left[ - \frac{1}{ik} e^{-ikx} \right]_{-a}^a \nonumber\\
        & = \frac{1}{\sqrt{2\pi} i k} [e^{ika} - e^{-ika}] \nonumber\\
        & = \sqrt{\frac{2}{\pi}} \frac{\sin(ka)}{k} \ . \label{TransPulsoCuadrado}
    \end{align}

    \begin{figure}[H]
        \centering
        \includegraphics[scale = 0.55]{Figuras/EjemploTransformada1.pdf}
        \caption{Pulso cuadrado y su transformada de Fourier, con $a = 5$.}
        \label{Espectro2}
    \end{figure}

\end{ejemplo}

\begin{ejemplo}
    \textbf{Distribución gaussiana.} Considere la gaussiana
    \begin{equation*}
        f(x) = n e^{-\beta x^2}, \quad  \beta > 0 \ .
    \end{equation*}
    Su transformada de Fourier está dada por 
    \begin{equation*}
        \tilde{f}(k) =  \frac{n}{\sqrt{2\pi}} \int_{-\infty}^{\infty} e^{-\beta x^2} e^{-ikx} dx =  \frac{n}{\sqrt{2\pi}} \int_{-\infty}^{\infty} e^{-\beta x^2-ikx} dx .
    \end{equation*}

    Notemos que 
    \begin{align*}
        -\beta x^2-ikx &= - \beta \left( x^2 + \frac{ik}{\beta}x \right) \\
        &= - \beta \left( x^2 + \frac{ik}{\beta} x + \left( \frac{ik}{2\beta} \right)^2 - \left( \frac{ik}{2\beta} \right)^2 \right) \\
        &= - \beta \left( x + \frac{ik}{2\beta} \right)^2 + \beta \left( \frac{ik}{2\beta} \right)^2 \\
        &= - \beta \left( x + \frac{ik}{2\beta} \right)^2 - \left( \frac{k^2}{4\beta} \right).
    \end{align*}

    Luego, su transformada de Fourier puede escribirse como
    \begin{equation*}
        \tilde{f}(k) =  \frac{n}{\sqrt{2\pi}} \int_{-\infty}^{\infty} e^{-\beta \left( x + ik/2\beta \right)^2 - \left(k^2/4\beta \right)}  dx = \frac{n}{\sqrt{2\pi}} e^{- \left( k^2/4\beta \right)} \int_{-\infty}^{\infty} e^{-\beta \left( x + ik/2\beta \right)^2} \,dx. 
    \end{equation*}

    Haciendo el cambio de variable $u = x + \frac{ik}{2\beta}$, obtenemos que \footnote{}
    \begin{equation}\label{eq:Fourier-Gaussiana}
        \hat{f}(k) = \frac{n}{\sqrt{2\pi}} e^{- \left( k^2/4\beta \right)} \int_{-\infty}^{\infty} e^{-\beta u^2} \, du = \frac{n}{\sqrt{2\beta}} e^{- \left( k^2/4\beta \right)} \ ,
    \end{equation}
    donde hemos usado que 
    \begin{equation}
    \int_{-\infty}^{\infty} e^{-\beta x^2} \,dx = \sqrt{\frac{\pi}{\beta}}, \quad \beta > 0 \ . 
    \end{equation} 

    \begin{figure}[H]
        \centering
        \includegraphics[width = 0.8\textwidth]{Figuras/EjemploTransformada2.pdf}
        \caption{Distribución gaussiana y su transformada de Fourier para $n=1$ y $\beta =1$ .}
        \label{Espectro3}
    \end{figure}
        \footnoterule
        
        {\footnotesize
        $^2$ En estricto rigor se debería calcular una integral compleja, vea \cite{Arfken}.
        }



    \begin{figure}[H]
        \centering
        \includegraphics[scale = 0.6]{Figuras/EjemploTransformada3.pdf}
        \caption{Distribución gaussiana y su transformada de Fourier para $n=1$ y $\beta =0.25$ .}
        \label{Espectro4}
    \end{figure}
\end{ejemplo}


\section{Propiedades de la transformada de Fourier}

\begin{propiedad} 
\textbf{Propiedades de la Transformada de Fourier.} Sean las funciones $f, g \in L^1$ y los escalares $\alpha, \beta \in \mathbb{C}$.

\begin{enumerate}
    \item \textbf{Linealidad}: $$\mathcal{F}\{\alpha f(x) + \beta g(x)\}(k) = \alpha \mathcal{F}\{f(x)\}(k) + \beta \mathcal{F}\{g(x)\}(k).$$ 
    
    \item Si $f$ es real, entonces 
    % $$\tilde{f}(-k) = \tilde{f}^\ast(k).$$
    $$ \mathcal{F}\{f(x)\}(-k) = (\mathcal{F}\{f(x)\}(k))^*.$$
    
    \item \textbf{Traslación}: $$\mathcal{F}\{f(x+a)\}(k) = e^{ika} \mathcal{F}\{f(x)\}(k), \quad a \in \mathbb{R}.$$
    
    \item \textbf{Cambio de escala:} $$\mathcal{F}\{f(\alpha x)\}(k) = \frac{1}{|\alpha|}\mathcal{F}\{f(x)\}\left(\frac{k}{ \alpha}\right), \quad\alpha \neq 0.$$
    
    \item  \textbf{Atenuación}: $$\mathcal{F}\{f(x)e^{-ax}\}(k) =  \mathcal{F}\{f(x)\}(k-ia), \quad a \in \mathbb{C}.$$
    
    \item Si $f$ es una función par, entonces $\tilde{f}$ es una función real.
    
    \item Si $f$ es una función impar, entonces $\tilde{f}$ es una función puramente imaginaria, es decir, $\tilde{f}(k) = - \tilde{f}(-k)$.

    % \item \textbf{Escalonamiento:} $$\mathcal{F}\{f(\alpha x)\}(k) = \frac{1}{|\alpha|}\mathcal{F}\{f(x)\}\left(\frac{k}{ \alpha}\right), \quad\alpha \neq 0.$$
\end{enumerate}
\end{propiedad}

\begin{demo}
Demostraremos los puntos desde el 1 hasta el 5, y volveremos más tarde a los puntos 6 y 7.

\begin{enumerate}
    \item Por la definición \eqref{T.Fourier} de la transformada de Fourier y usando el hecho de que las funciones son absolutamente convergentes, tenemos
    \begin{align*}
        \mathcal{F}\{\alpha f(x) + \beta g(x)\}(k) &= \frac{1}{\sqrt{2\pi}} \int_{-\infty}^{\infty} [\alpha f(x) + \beta g(x)] e^{-ikx} dx \\
        &= \frac{\alpha}{\sqrt{2\pi}} \int_{-\infty}^{\infty}  f(x) e^{-ikx} dx + \frac{\beta}{\sqrt{2\pi}} \int_{-\infty}^{\infty}  g(x) e^{-ikx} dx \\
        &= \alpha \mathcal{F}\{f(x)\}(k) + \beta \mathcal{F}\{g(x)\}(k).
    \end{align*}
    
    \item Por la definición \eqref{T.Fourier} de la transformada de Fourier y suponiendo que $f$ es real:
    \begin{align*}
        \mathcal{F}\{f(x)\}(-k) &= \frac{1}{\sqrt{2\pi}} \int_{-\infty}^{\infty}  f(x)  e^{-(-ikx)} dx  \\
        &= \frac{1}{\sqrt{2\pi}} \int_{-\infty}^{\infty}  f(x)  (e^{-ikx})^* dx  \\
        &= \frac{1}{\sqrt{2\pi}} \int_{-\infty}^{\infty}  [f(x)  e^{-ikx}]^* dx \\
        &= (\mathcal{F}\{f(x)\}(k))^*.
    \end{align*}

    \item Por la definición \eqref{T.Fourier} de la transformada de Fourier, se tiene que
    \begin{equation*}
        \mathcal{F}\{f(x+a)\}(k) = \frac{1}{\sqrt{2\pi}}\int_{-\infty}^\infty f(x+a) e^{-ikx} dx \ .
    \end{equation*}

    Al hacer la sustitución $s = x+a$, $ds = dx$, tenemos
    \begin{align*}
        \frac{1}{\sqrt{2\pi}} \int_{-\infty}^\infty f(x+a)e^{-ikx}dx & = \frac{1}{\sqrt{2\pi}} \int_{-\infty}^\infty f(s) e^{-ik(s-a)} ds \\
        & = \frac{1}{\sqrt{2\pi}} \int_{-\infty}^\infty f(s) e^{-iks+ika} ds \\
        & = e^{ika} \left(\frac{1}{\sqrt{2\pi}} \int_{-\infty}^\infty e^{-iks} ds \right) \\
        & = e^{ika} \mathcal{F}\{f(x)\} (k) \ .
    \end{align*}

    \item Por la definición \eqref{T.Fourier} de la transformada de Fourier, se tiene que
    \begin{equation*}
        \mathcal{F} \{ f(\alpha x) \}(k) = \frac{1}{\sqrt{2\pi}} \int_{-\infty}^\infty f(\alpha x)e^{-ikx} dx \ .
    \end{equation*}

    Supondremos $\alpha > 0$. Haciendo el cambio de variable $u = \alpha x$, $du = \alpha dx$, tenemos
    \begin{equation*}
        \frac{1}{\sqrt{2\pi}} \int_{-\infty}^\infty f(\alpha x) e^{-ikx} dx = \frac{1}{\alpha} \left(\frac{1}{\sqrt{2\pi}} \int_{-\infty}^\infty f(u) e^{-i(k/\alpha)u} du \right) = \frac{1}{\alpha} \mathcal{F}\{f(x)\} \left( \frac{k}{\alpha} \right) \ .
    \end{equation*}

    Ahora, si $\alpha < 0$, hacemos el mismo cambio de variable que antes, obteniendo
    \begin{align*}
        \frac{1}{\sqrt{2\pi}} \int_{-\infty}^\infty f(\alpha x)e^{-ikx} dx & = \frac{1}{\alpha} \left( \frac{1}{\sqrt{2\pi}} \int_{\infty}^{-\infty} f(u) e^{-i(k/\alpha)u} du \right) \\ 
        & = - \frac{1}{\alpha} \left(\frac{1}{\sqrt{2\pi}} \int_{-\infty}^\infty f(u) e^{-i(k/\alpha)u} du \right) \\
        & = - \frac{1}{\alpha} \mathcal{F}\{f(x)\} \left(\frac{k}{\alpha}\right) \ .
    \end{align*}

    Por lo tanto, concluimos que, para $\alpha \neq 0$, se tendrá que
    \begin{equation*}
        \mathcal{F}\{(\alpha x)\}(k) = \frac{1}{|\alpha|} \mathcal{F}\{f(x)\} \left( \frac{k}{\alpha} \right) \ .
    \end{equation*}

    \item Por la definición \eqref{T.Fourier} de la transformada de Fourier, se tiene que
    \begin{align*}
        \mathcal{F}\{ f(x)e^{-ax} \}(k) & = \frac{1}{\sqrt{2\pi}} \int_{-\infty}^\infty f(x) e^{-ax} e^{-ikx} dx \\
        & = \frac{1}{\sqrt{2\pi}} \int_{-\infty}^\infty f(x) e^{-ikx-ax} dx \\
        & = \frac{1}{\sqrt{2\pi}} \int_{-\infty}^\infty f(x) e^{-ikx+i^2ax} dx \\
        & = \frac{1}{\sqrt{2\pi}} \int_{-\infty}^\infty e^{-ix(k-ia)} dx \\
        & = \mathcal{F}\{f(x)\}(k-ia) \ .
    \end{align*}
\end{enumerate}
\end{demo}

\begin{propo}\marginnote{Transformada de Fourier de una derivada}
Sea $f(x)$ con transformada de Fourier $\mathcal{F}\{f(x)\}$ y $\lim\limits_{x \to \pm \infty} f(x) = 0$. Entonces, 
\begin{equation}
    \boxed{\mathcal{F}\{f'(x)\} = i k \mathcal{F}\{f(x)\}\ ,} 
\end{equation}
y en general, 
\begin{equation}
    \mathcal{F}\{f^{(n)} (x)\} = (i k)^n \mathcal{F}\{f(x)\} \ ,
\end{equation}
\end{propo}

\begin{demo}
Demostraremos el caso para la primera derivada, pues derivadas más altas se deducen a partir de este. Usando la definición \eqref{T.Fourier} de la transformada de Fourier, tenemos que 
\begin{align*}
    \mathcal{F}\{f'(x)\} &= \frac{1}{\sqrt{2\pi}} \int_{- \infty}^{\infty} f'(x) e^{-ikx} \,dx \\
    &= \frac{1}{\sqrt{2\pi}} \int_{- \infty}^{\infty} \left\{ \frac{d}{dx}\left[ f(x) e^{-ikx} \right]  - (-ik) f(x) e^{-ikx} \right\}\,dx \\
    &= \frac{1}{\sqrt{2\pi}} \left. f(x) e^{-ikx} \right|_{-\infty}^{\infty} + \frac{ik}{\sqrt{2\pi}} \int_{- \infty}^{\infty}  f(x) e^{-ikx} \,dx \\
    &=  ik \mathcal{F}\{f(x)\} + \left. f(x) e^{-ikx} \right|_{-\infty}^{\infty}.
\end{align*}

Como  $\lim\limits_{x \to \pm \infty} f(x) = 0$, obtenemos
$$\mathcal{F}\{f'(x)\} = i k \mathcal{F}\{f(x)\}.$$
\end{demo}

% En general,
% \begin{shaded}
% $$,$$    
% \end{shaded}


Dado que hemos entendido la transformada de Fourier como una extensión de las series de Fourier, como es el caso de que las condiciones de existencia de la transformada de Fourier son las mismas \emph{condiciones de Dirichlet} para la existencia de los coeficientes de Fourier. Por ello, esperaríamos una analogía a la fórmula de Parseval propia de las series de Fourier. Esta corresponde al siguiente teorema.

\begin{teorema}[de Parseval]
Si $f(x)$ y $g(x)$ son funciones reales y si $\tilde{f}(k)$ y $\tilde{g}(k)$ son sus correspondientes transformadas de Fourier, entonces

\begin{itemize}
    \item[(i)] (Primer teorema)
    $$\int_{-\infty}^{\infty} |\tilde{f}(k)|^2\, dk = \int_{-\infty}^{\infty} |f(x)|^2 \, dx. $$
    
    \item[(ii)] (Segundo teorema)
    $$\int_{-\infty}^{\infty} \tilde{f}(k) \tilde{g}(-k) \, dk = \int_{-\infty}^{\infty} f(x) g(x) \, dx. $$
    
\end{itemize}
\end{teorema}

\begin{demo}
Notemos que (i) es consecuencia de (ii) al tomar $g(x) = f(x)$ real tal que $f^*(x) = f(x)$ y $\tilde{g}(-k) = \tilde{f}^*(k)$. Luego, nos bastará demostrar el segundo teorema de Parseval.

Usando la definición \eqref{T.Fourier}, tenemos que 
$$\tilde{g}(-k) = \frac{1}{\sqrt{2\pi}} \int_{-\infty}^{\infty} g(x) e^{ikx} \,dx.$$

Luego, 
$$\int_{-\infty}^{\infty} \tilde{f}(k) \tilde{g}(-k) \,dk = \int_{-\infty}^{\infty} \tilde{f}(k) \,dk \int_{-\infty}^{\infty} \frac{1}{\sqrt{2\pi}} g(x) e^{ikx} \,dx.$$

Supongamos que podemos intercambiar el orden de integración, por ejemplo, al suponer que las integrales 
$$\int_{-\infty}^{\infty} \tilde{f}(k) e^{ikx} \,dk ~~\mbox{y}~~ \int_{-\infty}^{\infty} g(x) e^{ikx} \,dx$$
son absolutamente integrables. Entonces,
$$\int_{-\infty}^{\infty} \tilde{f}(k) \tilde{g}(-k) \,dk = \int_{-\infty}^{\infty} g(x) \left( \frac{1}{\sqrt{2\pi}} \int_{-\infty}^{\infty} \tilde{f}(k) e^{ikx} \,dk\right) \,dx.$$

Aplicando la transformada inversa de Fourier dada por \eqref{I.Fourier}, concluimos que
$$\int_{-\infty}^{\infty} \tilde{f}(k) \tilde{g}(-k) \,dk = \int_{-\infty}^{\infty} f(x) g(x)  \,dx.$$

\end{demo}

\begin{ejemplo}
    Use el teorema de Parseval para evaluar
    $$\int_{-\infty}^{\infty}  \frac{\sin^2(x)}{x^2} \,dx.$$

    \textbf{Solución:} Esta integral puede ser calculada usando el teorema del residuo. En nuestro caso, usaremos el primer teorema de Parseval, teniendo en cuenta el resultado de la transformada de Fourier del pulso cuadrado en el ejemplo \ref{PulsoCuadrado}. 

    Para $a = 1$ en la ecuación \eqref{TransPulsoCuadrado}, tenemos que
    $$\int_{- \infty}^{\infty} |\Tilde{f}(k)|^2 \,dk = \int_{- \infty}^{\infty} \frac{2}{\pi} \frac{\sin^2(k)}{k^2}  \,dk = \frac{2}{\pi} \int_{-\infty}^{\infty}  \frac{\sin^2(k)}{k^2} \,dk.$$

    Por el primer teorema de Parseval,
    \begin{align*}
        \int_{- \infty}^{\infty} |\Tilde{f}(k)|^2 \,dk &= \int_{-\infty}^{\infty} |f(x)|^2 \,dx \\
        \Rightarrow \frac{2}{\pi} \int_{-\infty}^{\infty}  \frac{\sin^2(k)}{k^2} &=  \int_{-1}^1 \,dx = 2 \ . 
    \end{align*}

Por lo tanto,
$$\int_{-\infty}^{\infty}  \frac{\sin^2(k)}{k^2} \,dk = \pi.$$
\end{ejemplo}

\section{Transformadas seno y coseno}

Dependiendo de la paridad de las funciones a las cuales le aplicamos una transformada de Fourier, es posible utilizar una forma \emph{abreviada} de transformada, de manera similar a lo que podemos hacer para una serie de Fourier.

Notemos que, dada una función real $f(x)$ impar, tenemos que
\begin{align*}
    \tilde{f}(k) & = \frac{1}{\sqrt{2\pi}} \int_{-\infty}^{\infty} f(x) e^{-ikx} dx \\
    & = \frac{1}{\sqrt{2\pi}}  \int_{-\infty}^0 f(x) e^{-ikx} \,dx + \frac{1}{\sqrt{2\pi}} \int_{0}^{\infty} f(x) e^{-ikx} \, dx \\
    & = -\frac{1}{\sqrt{2\pi}}  \int_{\infty}^0 f(-x) e^{ikx} \,dx + \frac{1}{\sqrt{2\pi}}  \int_{0}^{\infty} f(x) e^{-ikx} \, dx \ , \quad x \to -x \mbox{ en la integral 1} \ , \\
    & = -\frac{1}{\sqrt{2\pi}} \int_{0}^{\infty} f(x) e^{ikx} \,dx + \frac{1}{\sqrt{2\pi}}  \int_{0}^{\infty} f(x) e^{-ikx} \, dx \ , \quad f(-x) = -f(x) \ ,  \\
     &= -\frac{1}{\sqrt{2\pi}}  \int_{0}^{\infty} f(x) [e^{ikx} - e^{-ikx}  ] \, dx \ , \quad \int_a^b = - \int_b^a \ , \\
    & = -\frac{2i}{\sqrt{2\pi}}  \int_{0}^{\infty} f(x) \sin(kx) \,dx \\
    & = -i \left(\sqrt{\frac{2}{\pi}}  \int_{0}^{\infty} f(x) \sin(kx) \,dx \right) \ .
    %  \equiv - i  \tilde{f}_S(k),
\end{align*}

\begin{defi}\marginnote{Transformada seno de Fourier}
    Se define la \textbf{transformada seno de Fourier}, $\tilde{f}_S(k) = \mathcal{F}_S(k)$ como
    \begin{equation}
        \tilde{f}_S(x) = \sqrt{\frac{2}{\pi}} \int_0^\infty f(x) \sin(kx) dx \ ,
    \end{equation}
    cuya transformada inversa es dada por
    \begin{equation}
        f(x) = \mathcal{F}^{-1}_S\{\tilde{f}_S(k)\} = \sqrt{\frac{2}{\pi}} \int_0^\infty \tilde{f}_S(k) \sin(kx) dx \ .
    \end{equation}

    Dada una función $f(x)$ real e impar, su transformada de Fourier será puramente imaginaria, y puede obtenerse como
    \begin{equation*}
        \tilde{f}(k) = -i \tilde{f}_S(k) \ .
    \end{equation*}
\end{defi}

De forma análoga, dada una función $f(x)$ real y par, tenemos

\begin{align*}
    \tilde{f}(k) & = \frac{1}{\sqrt{2\pi}} \int_{-\infty}^{\infty} f(x) e^{-ikx} dx \\
    & = \frac{1}{\sqrt{2\pi}} \int_{-\infty}^0 f(x) e^{-ikx} \,dx + \frac{1}{\sqrt{2\pi}} \int_{0}^{\infty} f(x) e^{-ikx} \, dx  \\
    & = -\frac{1}{\sqrt{2\pi}} \int_{\infty}^0 f(-x) e^{ikx} \,dx + \frac{1}{\sqrt{2\pi}} \int_{0}^{\infty} f(x) e^{-ikx} \, dx \ , \quad x \to -x \mbox{ en la integral 1} \ , \\
    & = \frac{1}{\sqrt{2\pi}} \int_{0}^{\infty} f(x) e^{ikx} \,dx + \frac{1}{\sqrt{2\pi}} \int_{0}^{\infty} f(x) e^{-ikx} \, dx \ , \quad f(-x) = f(x) \ , \\
    & = \frac{1}{\sqrt{2\pi}} \int_{0}^{\infty} f(x) [e^{ikx} + e^{-ikx}  ] \,dx \\
    & = \sqrt{\frac{2}{\pi}} \int_{0}^{\infty} f(x) \cos(kx) \,dx \ .
\end{align*}

\begin{defi}\marginnote{Transformada coseno de Fourier}
    Se define la \textbf{transformada coseno de Fourier}, $\tilde{f}_C(k) = \mathcal{F}_C(k)$ como
    \begin{equation}
        \tilde{f}_C(x) = \sqrt{\frac{2}{\pi}} \int_0^\infty f(x) \cos(kx) dx \ ,
    \end{equation}
    cuya transformada inversa es dada por
    \begin{equation}
        f(x) = \mathcal{F}^{-1}_C\{\tilde{f}_C(k)\} = \sqrt{\frac{2}{\pi}} \int_0^\infty \tilde{f}_C(k) \cos(kx) dx \ .
    \end{equation}

    Dada una función $f(x)$ real y par, su transformada de Fourier será real, y puede obtenerse como
    \begin{equation*}
        \tilde{f}(k) = \tilde{f}_C(k) \ .
    \end{equation*}
\end{defi}

Nuevamente, la elección del factor $\sqrt{\frac{2}{\pi}}$ es convencional, y otras elecciones pudieron ser hechas, mientras se satisfaga la integral de Fourier \eqref{IntegralFourier}.

% \begin{ejemplo}[Paridad]
% Si $f(x) \in \mathbb{R}$ e impar, entonces
% \begin{align*}
%      \tilde{f}(k) &= \frac{1}{2\pi} \int_{-\infty}^{\infty} f(x) e^{-ikx} dx \\
% &= \frac{1}{2\pi} \int_{-\infty}^0 f(x) e^{-ikx} \,dx + \frac{1}{2\pi} \int_{0}^{\infty} f(x) e^{-ikx} \, dx  \\
%  &= -\frac{1}{2\pi} \int_{\infty}^0 f(-x) e^{ikx} \,dx + \frac{1}{2\pi} \int_{0}^{\infty} f(x) e^{-ikx} \, dx \\
%      &= -\frac{1}{2\pi} \int_{0}^{\infty} f(x) e^{ikx} \,dx + \frac{1}{2\pi} \int_{0}^{\infty} f(x) e^{-ikx} \, dx  \\
%       &= -\frac{1}{2\pi} \int_{0}^{\infty} f(x) [e^{ikx} - e^{-ikx}  ] \,dx \\
%      &= -\frac{2i}{2\pi} \int_{0}^{\infty} f(x) \sin(kx) \,dx \equiv - i  \tilde{f}_S(k),
%      \end{align*}
     
%      donde $\tilde{f}_S$ es conocida como la \textbf{transformada seno de Fourier} de la función $f(x)$, y viene definida por \cite{Mauch} 
%      $$\boxed{\tilde{f}_S(k) = \frac{1}{\pi}  \int_{0}^{\infty} f(x) \sin(kx) \,dx}$$
     
% Análogamente a la definición de la  transformada seno de Fourier, si $f(x) \in \mathbb{R}$ y par, entonces 
% \begin{align*}
%      \tilde{f}(k) = \frac{1}{2\pi} \int_{-\infty}^{\infty} f(x) e^{-ikx} dx &= \frac{1}{2\pi} \int_{-\infty}^0 f(x) e^{-ikx} \,dx + \frac{1}{2\pi} \int_{0}^{\infty} f(x) e^{-ikx} \, dx  \\
%      &= -\frac{1}{2\pi} \int_{\infty}^0 f(-x) e^{ikx} \,dx + \frac{1}{2\pi} \int_{0}^{\infty} f(x) e^{-ikx} \, dx \\
%      &= \frac{1}{2\pi} \int_{0}^{\infty} f(x) e^{ikx} \,dx + \frac{1}{2\pi} \int_{0}^{\infty} f(x) e^{-ikx} \, dx \\
%      &= \frac{1}{2\pi} \int_{0}^{\infty} f(x) [e^{ikx} + e^{-ikx}  ] \,dx \\
%      &= \frac{2}{2\pi} \int_{0}^{\infty} f(x) \cos(kx) \,dx \equiv  \tilde{f}_C(k),
%      \end{align*}
     
% donde $\tilde{f}_C$ es conocida como la \textbf{transformada coseno de Fourier} de la función $f(x)$, y viene definida por \cite{Mauch}
%      $$\boxed{\tilde{f}_C(k) = \frac{1}{\pi}  \int_{0}^{\infty} f(x) \cos(kx) \,dx}$$

% \textbf{Observación: } Esta forma de escribir la transformada seno y coseno de Fourier es convencional, por el factor $1/\pi$.
% \end{ejemplo}

\section{Delta de Dirac}

Es común en física utilizar el concepto de \emph{pulso con duración infinitamente corta}. Por ejemplo, un cuerpo en movimiento por un golpe repentino alcanza un momentum igual al impulso del golpe, matemáticamente,
\begin{equation*}
    mv = I = \int_{t_0}^{t_0+\tau} F(t) dt \ ,
\end{equation*}
donde $F(t)$ es la fuerza y $\tau$ es la duracción de la acción de la fuerza. Al referirnos a un \emph{golpe}, insinuamos que la duración es lo suficientemente pequeña como para que el cambio en el momentum sea casi instantáneo. Sin embargo, para que esto sea posible, la fuerza debería haber sido infinita durante el golpe, y cero en otros lados.

Sin embargo, lo más probable es que la función se parezca a la figura X, donde $h$ es muy grande y $\tau$ muy pequeño, tal que el área debajo de la curva corresponde al impulso $I$. Para esto, necesitaríamos conocer la forma exacta de $F(t)$, lo que no siempre es posible. Para resolver este problema, aproximamos un pulso de esta forma por la \emph{`función'\footnote{Existe toda una discusión respecto al hecho de que este elemento no es una función propiamente dicha. Esta será omitida durante el curso.} Delta de Dirac}, que será de gran utilidad en diferentes áreas de la Física.

\begin{defi}\marginnote{Delta de Dirac}
    Se define la \textbf{Delta de Dirac} centrada en $x=a$ como la función
    \begin{equation}
        \delta(x-a) = \left\{ \begin{array}{cc}
            0 \ , & \quad x \neq a \ , \\
            \infty \ , & \quad x = a \ ,
        \end{array} \right.
    \end{equation}
    tal que la integral de $\delta(x)$ está normalizada,
    \begin{equation}
        \int_{-\infty}^{\infty} \delta(x-a) dx = 1 \ ,
    \end{equation}
    y que para cualquier función $f(x)$ continua, satisface
    \begin{equation}
        \int_{-\infty}^{\infty} \delta(x-a) f(x) dx = f(a) \ .
    \end{equation}
\end{defi}

\begin{propiedad}
    \textbf{Propiedades de la Delta de Dirac.}
    \begin{enumerate}
        \item Si $\delta'(x)$ denota a la derivada de la delta de Dirac, y $f'(x)$ representa la derivada de $f(x)$, entonces se satisface
        \begin{equation}
            \int_{-\infty}^{\infty} \delta'(x) f(x) \, dx = - f'(0) \ .
        \end{equation}

        Esta idea se puede generalizar a derivadas de orden superior, tal que, asumiendo que $f$ es $m$ veces diferenciable
        \begin{equation}
            \int_{-\infty}^{\infty} \delta^{(m)}(x) f(x) \, dx = - f^{(m)}(0) \ .
        \end{equation}

        \item Dada la \textbf{función escalón de Heaviside}, definida como
        \begin{equation}
            H(x) = \begin{array}{cc}
                0 \ , & x < 0 \ , \\
                1 \ , & x \geq 0 \ ,
            \end{array}
        \end{equation}
        entonces la delta de Dirac puede entenderse como su derivada, es decir,
        \begin{equation}
            \delta(x) = \frac{dH}{dx} \ .
        \end{equation}
        
        \item Dada una función continua $\phi(x)$, la delta de Dirac satisface
        \begin{equation}
            \phi(x+a) \delta(x) = \phi(a) \delta(x) \ ,
        \end{equation}
        y en particular,
        \begin{equation}
            x \delta(x) = 0 \ .
        \end{equation}
        \item A partir de las reglas de cambio de variables, podemos obtener que
        \begin{equation}
            \delta(g(x)) = \sum_i \frac{\delta(x-x_i)}{|g'(x_i)|} \ ,
        \end{equation}
        donde $x_i$ son las raíces de la función $g$, es decir, $g(x_i) = 0$, y que además satisfacen $g'(x_i) \neq 0$, para todo $x_i$.
        \item Como caso particular de la propiedad anterior, tenemos que
        \begin{equation}
            \delta(ax) = \frac{1}{|a|}\delta(x) \ , \qquad a \neq 0 \ .
        \end{equation}
        Como consecuencia,
        \begin{equation}
            \delta(-x) = \delta(x) \ .
        \end{equation}
    \end{enumerate}
\end{propiedad}

\subsection{Representación integral}

La delta de Dirac puede ser representada como
\begin{equation} \label{eq:Dirac-Integral}
    \delta(x-a) = \frac{1}{2\pi} \int_{-\infty}^{\infty} e^{ik(x-a)} dk \ .
\end{equation}

Podemos observar que esta definición es similar a la transformada de Fourier inversa de $e^{-ika}$. En efecto, utilizando la definición de la delta de Dirac, tenemos que
\begin{equation}
    \tilde{f}(k) = \frac{1}{\sqrt{2\pi}} \int_{-\infty}^\infty \delta(x-a) e^{-ikx} dx = \frac{1}{\sqrt{2\pi}} e^{-ika} \ .
\end{equation}

\begin{ejemplo}
    Determine la transformada de Fourier de las funciones $\sin(\alpha x)$ y $\cos(\alpha x)$.

    Observamos que, para $f(x) = \sin(\alpha x)$, tendremos
    \begin{align*}
        \mathcal{F}\{\sin(\alpha x)\}(k) & = \frac{1}{\sqrt{2\pi}} \int_{-\infty}^\infty \sin(\alpha x) e^{-ikx} dx \\
        & = \frac{1}{\sqrt{2\pi}} \int_{-\infty}^\infty \left( \frac{e^{i\alpha x} - e^{-i\alpha x}}{2i} \right) e^{-ikx} dx \\
        & = \frac{1}{2i} \left\{ \frac{1}{\sqrt{2\pi}} \int_{-\infty}^\infty e^{i(\alpha - k)x} dx - \frac{1}{\sqrt{2\pi}} \int_{-\infty}^\infty e^{-i(\alpha+k)x} dx \right\} \\
        & = \frac{1}{2i} \left\{ \sqrt{2\pi} \delta(\alpha-k) - \sqrt{2\pi} \delta(\alpha+k) \right\} \ , 
    \end{align*}
    donde hemos usado la definición de la delta de Dirac \eqref{eq:Dirac-Integral}. Luego,
    \begin{equation}
        \boxed{\mathcal{F}\{\sin(\alpha x)\}(k) = \sqrt{\frac{\pi}{2}}i[\delta(k+\alpha) - \delta(k-\alpha)] \ .}
    \end{equation}

    De forma análoga, para $f(x) = \cos(\alpha x)$, tendremos
    \begin{align*}
        \mathcal{F}\{\cos(\alpha x)\}(k) & = \frac{1}{\sqrt{2\pi}} \int_{-\infty}^\infty \cos(\alpha x) e^{-ikx} dx \\
        & = \frac{1}{\sqrt{2\pi}} \int_{-\infty}^\infty \left( \frac{e^{i\alpha x} + e^{-i\alpha x}}{2} \right) e^{-ikx} dx \\
        & = \frac{1}{2} \left\{ \frac{1}{\sqrt{2\pi}} \int_{-\infty}^\infty e^{i(\alpha - k)x} dx + \frac{1}{\sqrt{2\pi}} \int_{-\infty}^\infty e^{-i(\alpha+k)x} dx \right\} \\
        & = \frac{1}{2} \left\{ \sqrt{2\pi} \delta(\alpha-k) + \sqrt{2\pi} \delta(\alpha+k) \right\} \ , 
    \end{align*}
    donde hemos usado la definición de la delta de Dirac \eqref{eq:Dirac-Integral}. Luego,
    \begin{equation}
        \boxed{\mathcal{F}\{\cos(\alpha x)\}(k) = \sqrt{\frac{\pi}{2}}[\delta(k+\alpha) + \delta(k-\alpha)] \ .}
    \end{equation}

\end{ejemplo}

\subsection{Delta de Dirac tridimensional}

Por último, mencionaremos que es común en Física utilizar una delta de Dirac en tres (o más) dimensiones para describir distribuciones puntuales en el espacio, como lo puede ser una carga eléctrica puntual. Para ello, utilizamos una definición análoga al caso unidimensional, donde
\begin{equation}
    \delta^{(3)}(\vec{x}-\vec{a}) = 0 \, \quad \vec{x} \neq \vec{a} \ ,
\end{equation}
donde $\vec{x} = x \hat{x} + y \hat{y} + z\hat{z}$ es el vector posición de nuestro sistema coordenado, y $\vec{a} = a_x \hat{x} + a_y \hat{y} + a_z \hat{z}$ es un vector cualquiera que localiza a nuestro punto en el espacio. Para cualquier campo escalar, se cumplirá queda
\begin{equation}
    \int_V \delta^{(3)} (\vec{x}-\vec{a}) f(\vec{x}) dV = \left\{\begin{array}{cc}
        f(\vec{a}) \ , & \text{si } \vec{a} \in V \ , \\
        0 \ , & \text{si } \vec{a} \notin V \ .
    \end{array}\right.
\end{equation}

En coordenadas cartesianas, esta corresponde únicamente al producto de tres deltas de Dirac unidimensionales, una por cada coordenada del sistema, tal que
\begin{equation}
    \delta^{(3)}(\vec{x} - \vec{a}) = \delta(x-a_x) \delta(y - a_y) \delta(z - a_z) \ ,
\end{equation}
mientras que en un sistema de coordenadas curvilíneas ortogonales, cuyas coordenadas son $(\xi_1, \xi_2, \xi_3)$, y factores de escala 
\begin{equation}
    h_i = \left\| \frac{\partial \vec{x}}{\partial \xi_i} \right\| = \left[ \left( \frac{\partial x}{\partial \xi_i} \right)^2 + \left( \frac{\partial y}{\partial \xi_i} \right)^2 + \left( \frac{\partial z}{\partial \xi_i} \right)^2 \right]^{1/2} \ , \quad i = 1, 2, 3 \ ,
\end{equation}
la delta de Dirac será dada por
\begin{equation}
    \delta^{(3)}(\vec{x}-\vec{x}_0) = \frac{1}{h_1 h_2 h_3} \delta(\xi_1 - \xi_{10}) \delta(\xi_2 - \xi_{20}) \delta(\xi_3 - \xi_{30}) \ .
\end{equation}

\begin{demo}
    Proponemos un ansatz de la forma
    \begin{equation*}
        \delta^{(3)} (\x-\x_0) = A(\xi_1, \xi_2, \xi_3) \delta(\xi_1 - \xi_{10}) \delta(\xi_2 - \xi_{20}) \delta(\xi_3 - \xi_{30}) \ ,
    \end{equation*}
    donde $A(\xi_1, \xi_2, \xi_3)$ es una función que depende de las coordenadas.

    Este ansatz deberá satisfacer la condición de normalización de la delta de Dirac, 
    \begin{align*}
        \int_{\mathbb{R}^3} \delta^{(3)} (\x - \x_0) dV & = \int A(\xi_1, \xi_2, \xi_3) \delta(\xi_1 - \xi_{10}) \delta(\xi_2 - \xi_{20}) \delta(\xi_3 - \xi_{30}) h_1 h_2 h_3 d\xi_1 d\xi_2 d\xi_3 \\ 
        & = A(\xi_{10}, \xi_{20}, \xi_{30}) h_1(\xi_{10}, \xi_{20}, \xi_{30}) h_2(\xi_{10}, \xi_{20}, \xi_{30}) h_3(\xi_{10}, \xi_{20}, \xi_{30}) \\
        & = 1 \ ,
    \end{align*}
    con lo que podemos proponer que $A(\xi_1, \xi_2, \xi_3) = (h_1 h_2 h_3)^{-1}$, con lo que la delta de Dirac en un sistema curvilíneo ortogonal arbitrario está dada por
    \begin{equation*}
        \delta^{(3)}(\vec{x}-\vec{x}_0) = \frac{1}{h_1 h_2 h_3} \delta(\xi_1 - \xi_{10}) \delta(\xi_2 - \xi_{20}) \delta(\xi_3 - \xi_{30}) \ .
    \end{equation*}
\end{demo}

En particular, para \textbf{coordenadas esféricas}, la delta de Dirac es dada por
\begin{equation}
    \boxed{\delta^{(3)}(\vec{x} - \vec{x}_0) = \frac{1}{r^2 \sin\theta} \delta(r-r_0) \delta(\theta-\theta_0) \delta(\phi - \phi_0) \ ,}
\end{equation}
mientras que en \textbf{coordenadas cilíndricas}, será dada por
\begin{equation}
    \boxed{\delta^{(3)}(\x - \x_0) = \frac{1}{\rho} \delta(\rho - \rho_0) \delta(\phi-\phi_0) \delta(z-z_0) \ .}
\end{equation}

Mención aparte merece el caso en que el problema tiene simetría azimutal o axial. De acuerdo con la propiedad de normalización, $\int_V \delta^{(3)}(\vec{x}-\x_0) = 1$. Si omitimos la existencia de la delta en la coordenada azimutal $\phi$, esta integral se cumplirá únicamente si definimos la delta como
\begin{align*}
    \delta^{(3)}(\x - \x_0) & = \frac{1}{2\pi} \left( \frac{1}{r^2\sin\theta} \delta(r-r_0) \delta(\theta-\theta_0) \right) \ ,
    & = \frac{1}{2\pi} \left( \frac{1}{\rho} \delta(\rho - \rho_0) \delta(z-z_0) \right) \ ,
\end{align*}
ya que la integral de normalización incluirá un término de la forma
\begin{align*}
    \int_0^{2\pi} r^2 \sin \theta d\phi & = 2\pi r^2 \sin \theta \ , \\
    \int_0^{2\pi} \rho d\phi & = 2\pi \rho \ .
\end{align*}

Si en el caso esférico tenemos simetría axial, entonces
\begin{align*}
    \delta^{(3)}(\x - \x_0) = \frac{1}{2r^2} \delta(r - r_0) \delta(\phi - \phi_0) \ ,
\end{align*}
ya que
\begin{equation*}
    \int_0^\pi r^2 \sin \theta d\theta = 2r^2 \ .
\end{equation*}

Por último, si tenemos tanto simetría axial como azimutal, entonces
\begin{align*}
    \delta^{(3)}(\x - \x_0) = \frac{1}{4\pi r^2} \delta(r - r_0) \ ,
\end{align*}
ya que
\begin{equation*}
    \int_0^{2\pi} d\phi \int_0^\pi r^2 \sin \theta d\theta = 4\pi r^2 \ .
\end{equation*}

\begin{ejemplo}
    Considere un anillo de carga $Q$ distribuida uniformemente y radio $a$ ubicado en el plano $xy$ con su centro en el origen. Encuentre la densidad volumétrica de carga en coordenadas cilíndricas y en coordenadas esféricas.

    Recordamos que una densidad de carga, integrada sobre la región de interés, debe corresponderse con la carga total del sistema, es decir,
    \begin{equation*}
        \int_V \rho(\x) dV = Q \ .
    \end{equation*}

    Además, para una carga puntual $Q$ situada en el punto $\x_0$, podemos describir su densidad como
    \begin{equation*}
        \rho(\x) = Q \delta^{(3)}(\x-\x_0) \ .
    \end{equation*}

    Usaremos un enfoque similar a aquel de la carga puntual, pero considerando que el anillo es un objeto en dos dimensiones. Dado que la carga se distribuye uniformemente sobre el anillo, podemos suponer que el sistema tiene simetría azimutal. Además, como se ubica en el plano $xy$, esto quiere decir que $z_0 = 0$, y que $\theta_0 = \pi/2$. De esta forma, la densidad de carga del sistema será dada por
    \begin{align*}
        \rho(\x) & = Q \frac{1}{2\pi \rho} \delta(\rho - a) \delta(z) \ , \\
        & = Q \frac{1}{2\pi r^2 \sin\theta} \delta(r-a) \delta(\theta - \pi/2) \ .
    \end{align*}
    
    Es directo comprobar que, en ambos casos,
    \begin{equation*}
        \int_V \rho(\x) = Q \ .
    \end{equation*}
\end{ejemplo}

\section{Convolución}

Imaginemos algún tipo de \emph{estímulo} físico, por ejemplo, una fuerza en el tiempo, $f(t)$. Consideremos que la respuesta a este estímulo puede ser modelada mediante la función $g(x,t) = g(x-t)$, como puede ser, por ejemplo, el movimiento de una partícula en la posición $x$ frente a esta fuerza. Si el sistema es \emph{lineal}, entonces la respuesta total en el punto $x$ al estímulo global $\{f(t) : t \in \mathbb{R} \}$ será la \emph{suma} de todas las contribuciones infinitesimales $[dt f(t)] g(x,t)$. Matemáticamente, a estas contribuciones las llamamos \textbf{convolución}.

\begin{defi}\marginnote{Convolución}
Sean $f(x)$ y $g(x)$ dos funciones reales, se define la operación \textbf{convolución} de dos funciones $f$ y $g$ como 
\begin{equation}
 (f*g)(x) := \int_{-\infty}^{\infty} f(y) g(x-y) \,dy.   \label{Convolucion}
\end{equation}

\end{defi}

% \textbf{Idea física:} Sea $f(t)$ algún \emph{estímulo} físico, como puede serlo una fuerza en el tiempo $t$, la densidad de carga en la posición $x$, etc. Sea $g(x,t) = g(x-t)$ la respuesta en $x$ a un estímulo en $t$. Si el sistema es \textit{lineal}, la respuesta total en el punto $x$ al estímulo global $\{f(t) : t \in \mathbb{R}\}$ será la "suma" de todas las contribuciones $[dt\, f(t)] g(x,t)$, que es la convolución $(f*g)(x)$. 

Si nuestro estímulo es el potencial electrostático debido a una densidad de carga $\rho(\Vec{x})$, nuestra respuesta total, que corresponde al potencial electrostático $\phi(\x)$, se puede escribir como
\begin{equation}
    \phi(\Vec{x}) = \int_{V} \frac{\rho(\Vec{x}\,')}{|\Vec{x} - \Vec{x}\,'|} \, dV' = \rho * f(\Vec{x}) \ ,
\end{equation}
donde  $f(\Vec{x}) = 1/|\Vec{x}|$.
\begin{figure}[htbp]
    \centering
    \includegraphics[width = 8cm]{Figuras/Distribucion-Cargas.pdf}
    \caption{Distribución de carga de densidad $\rho(\Vec{x})$.}
    \label{fig:PotencialDistribucion}
\end{figure}

Matemáticamente, podemos entender la convolución $f * g$ como \emph{el grado de traslape} entre dos pulsos, $f(y)$ y $g(-y)$, cuando uno $g(-y)$ se encuentra desplazado en $x$ unidades. Esta idea se puede observar en la figura \ref{fig:IdeaConvolucion}.

\begin{figure}[htbp]
    \centering
    \includegraphics[width=10cm]{Figuras/Idea-Convolucion.pdf}
    \caption{Idea matemática de la convolución. En (a), se expresa cada función en términos de la variable de integración $y$. En (b), se refleja la gráfica de $g(y)$ con respecto al eje vertical, es decir, $g(y) \rightarrow g(-y)$. En (c), se traslada la gráfica de $g(-y)$, $x$ unidades. Luego, se traslapan las gráficas de $f(y)$ y $g(x-y)$ de tal forma que el área sombreada corresponde al valor de $f * g$ para ese valor de $x$.}
    \label{fig:IdeaConvolucion}
\end{figure}

% Entonces, $f * g$ mide el grado de traslape entre $f(y)$ y $g(-y)$, luego de trasladar $g$ a una distancia $x$.

\begin{propiedad}\textbf{Propiedades de la convolución.}

Sean $f(x)$, $g(x)$ y $h(x)$  funciones reales. Entonces, se verifican las siguientes propiedades:

\begin{enumerate}
    \item \textbf{Conmutatividad:}$$f(x) * g(x) = g(x) * f(x).$$
    
    \item \textbf{Asociatividad:} $$[f(x)*g(x)]*h(x) = f(x)*[g(x)*h(x)].$$
    
    \item \textbf{Distributividad:} $$f(x)*[g(x)+h(x)] = f(x)*g(x) + f(x)*h(x).$$ 
\end{enumerate}
\end{propiedad}

\newpage

\begin{teorema}[de convolución de Fourier]
Sean $f(x)$, $g(x)$ y $h(x)$  funciones reales y sean $\tilde{f}(k)$, $\tilde{g}(k)$ y $\tilde{h}(k)$ sus correspondientes transformadas de Fourier. 

\begin{itemize}
    \item Si $\tilde{h}(k) = \tilde{f}(k) \tilde{g}(k)$, entonces 
$$ h(x) = \frac{1}{\sqrt{2\pi}} (f * g)(x) = \frac{1}{\sqrt{2\pi}} \int_{-\infty}^{\infty} f(y) g(x-y) \,dy.$$ 

    \item Si $h(x) = f(x) g(x)$, entonces
    \begin{equation}
        \Tilde{h}(k) = \frac{1}{\sqrt{2\pi}}(\Tilde{f} * \Tilde{g})(k) = \frac{1}{\sqrt{2\pi}} \int_{-\infty}^{\infty} \Tilde{f}(y) \Tilde{g}(k-y) \,dy.
    \end{equation}
% $$$$
\end{itemize}
\end{teorema}

\begin{demo}
\ 

\begin{itemize}
    \item Supongamos que  $\tilde{h}(k) = \tilde{f}(k) \tilde{g}(k)$. Aplicando la transformada de Fourier inversa dada por \eqref{I.Fourier}, tenemos que 
\begin{align*}
   h(x) = \mathcal{F}^{-1} \{\tilde{h}(k)\} & = \mathcal{F}^{-1} \{\tilde{f}(k) \tilde{g}(k)\} \\
   & = \frac{1}{\sqrt{2\pi}} \int_{-\infty}^{\infty} \tilde{f}(k) \tilde{g}(k) e^{ikx} \,dk \\
   & = \frac{1}{\sqrt{2\pi}} \int_{-\infty}^{\infty}  \left( \frac{1}{\sqrt{2\pi}} \int_{-\infty}^{\infty} f(y) e^{-ik y} \,dy \right) \tilde{g}(k)  e^{ikx}  \,dk.
\end{align*}

Si el intercambio de orden de integración es posible, entonces 
\begin{align*}
  h(x) &= \frac{1}{\sqrt{2\pi}} \int_{-\infty}^{\infty}  f(y)  \left( \frac{1}{\sqrt{2\pi}} \int_{-\infty}^{\infty}  \tilde{g}(k)  e^{ikx} e^{-ik y}\,d k \right) \,dy \\
   & = \frac{1}{\sqrt{2\pi}} \int_{-\infty}^{\infty}  f(y)  \left( \frac{1}{\sqrt{2\pi}} \int_{-\infty}^{\infty}  \tilde{g}(k)  e^{ik(x-y)}\,d k \right) \,dy \\
   & = \frac{1}{\sqrt{2\pi}} \int_{-\infty}^{\infty}  f(y) g(x-y) \,dy = \frac{1}{\sqrt{2\pi}} (f * g)(x).
\end{align*}

Como la convolución es conmutativa:
$$h(x) = \frac{1}{2\pi} (f * g)(x) = \frac{1}{2\pi} (g * f)(x)  = \frac{1}{\sqrt{2\pi}} \int_{-\infty}^{\infty}  f(y) g(x-y) \,dy.$$

\item  Supongamos que $h(x) = f(x) g(x)$. Aplicando la transformada de Fourier dada por \eqref{T.Fourier}, tenemos que 
\begin{align*}
   \Tilde{h}(k) &=  \mathcal{F} \{f(x) g(x)\} \\
   &= \frac{1}{\sqrt{2\pi}} \int_{-\infty}^{\infty} f(x) g(x) e^{-ikx} \,dx \\
   &=  \frac{1}{\sqrt{2\pi}} \int_{-\infty}^{\infty} \left( \frac{1}{\sqrt{2\pi}} \int_{-\infty}^{\infty} \Tilde{f}(y) e^{iyx} \,dy \right) g(x) e^{-ikx} \,dx.
\end{align*}

Si el intercambio de orden de integración es posible, entonces 
\begin{align*}
   \Tilde{h}(k) &=  \frac{1}{\sqrt{2\pi}} \int_{-\infty}^{\infty} \Tilde{f}(y) \left( \frac{1}{\sqrt{2\pi}} \int_{-\infty}^{\infty} g(x) e^{iyx}  e^{-ikx}\,dx \right) \,dy \\
   &=  \frac{1}{\sqrt{2\pi}} \int_{-\infty}^{\infty} \Tilde{f}(y) \left( \frac{1}{\sqrt{2\pi}} \int_{-\infty}^{\infty} g(x) e^{-i(k-y)x} \,dx \right) \,dy \\
   &=  \frac{1}{\sqrt{2\pi}} \int_{-\infty}^{\infty} \Tilde{f}(y) \Tilde{g}(k-y) \,dy. 
\end{align*}

Por lo tanto,
$$\Tilde{h}(k) =  \frac{1}{\sqrt{2\pi}} (\Tilde{f} * \Tilde{g})(k) = \frac{1}{\sqrt{2\pi}} \int_{-\infty}^{\infty} \Tilde{f}(y) \Tilde{g}(k-y) \,dy.$$
\end{itemize}

\end{demo}


\begin{ejemplo}
    Sabiendo que \cite{Mauch}
    $$\mathcal{F}\left\{ \frac{2c}{x^2+c^2} \right\} = e^{-c|k|}, \quad \text{para} ~ c > 0.$$

    Podemos usar el teorema de convolución para encontrar la transformada de Fourier de
    \begin{equation}
      f(x) = \frac{1}{x^4+5x^2+4} = \frac{1}{(x^2+1)(x^2+4)}.    \label{EjConvo}
    \end{equation}
  
    En efecto,
    \begin{align*}
        \mathcal{F}\{f(x)\} &=  \mathcal{F}\left\{ \frac{1}{8} \frac{2}{x^2+1} \frac{4}{x^2+4} \right\} \\
        &= \frac{1}{8} \left( \int_{-\infty}^{\infty} e^{-|y|}e^{-2|k-y|} dy \right) \\
        &= \frac{1}{8} \left( \int_{-\infty}^0 e^{y}e^{-2|k-y|} dy +   \int_0^{\infty} e^{-y}e^{-2|k-y|} dy \right).
    \end{align*}

Si $k > 0$,
\begin{align*}
    \mathcal{F}\{f(x)\} &= \frac{1}{8} \left(  \int_{-\infty}^0 e^{y}e^{-2(k-y)} dy +  \int_0^k e^{-y}e^{-2(k-y)} dy + \int_k^{\infty} e^{-y}e^{2(k-y)} dy \right)  \\
    &= \frac{1}{8} \left(  \int_{-\infty}^0 e^{-2k+3y} dy +  \int_0^k e^{-2k+y} dy + \int_k^{+\infty} e^{2k-3y} dy \right) \\
    &= \frac{1}{8} \left( \frac{1}{3} e^{-2k} + e^{-k} - e^{-2k} + \frac{1}{3} e^{-k} \right) \\
    &= \frac{1}{6} e^{-k} - \frac{1}{12} e^{-2k}. 
\end{align*}

Si $k < 0$,
\begin{align*}
    \mathcal{F}\{f(x)\} &= \frac{1}{8} \left(  \int_{-\infty}^k e^{y}e^{-2(k-y)} dy +  \int_k^0 e^{y}e^{2(k-y)} dy + \int_0^{\infty} e^{-y}e^{2(k-y)} dy \right)  \\
    &= \frac{1}{8} \left(  \int_{-\infty}^k e^{-2k+3y} dy +  \int_k^0 e^{2k-y} dy + \int_0^{\infty} e^{2k-3y} dy \right) \\
    &= \frac{1}{8} \left( \frac{1}{3} e^{k} - e^{2k} + e^k + \frac{1}{3} e^{2k} \right) \\
    &= \frac{1}{6} e^{k} - \frac{1}{12} e^{2k}. 
\end{align*}

Por lo tanto, para $k$ positivo como negativo,
$$\boxed{\mathcal{F}\{f(x)\} =  \frac{1}{6} e^{-|k|} - \frac{1}{12} e^{-2|k|}} $$

Una mejor forma de encontrar la transformada de Fourier de \eqref{EjConvo} es, en primer lugar, descomponer la función en fracciones parciales,
$$f(x) = \frac{1}{3} \frac{1}{x^2+1} - \frac{1}{3} \frac{1}{x^2+4},$$

para luego hacer usar de la linealidad de la transformada.
\begin{align*}
     \mathcal{F}\{ f(x)\} &= \frac{1}{6} \mathcal{F} \left\{ \frac{2}{x^2+1}\right\} - \frac{1}{12} \mathcal{F} \left\{ \frac{4}{x^2+4} \right\}  \\
     &= \frac{1}{6} e^{-|k|} - \frac{1}{12} e^{-2|k|}.
\end{align*}

\end{ejemplo}

% \section{Aplicación de la transformada de Fourier}

% La transformada de Fourier es útil para resolver ecuaciones diferenciales en el dominio $(-\infty, \infty)$ con condiciones de borde homogéneas en el infinito. En particular, en ecuaciones diferenciales \underline{lineales} con \underline{coeficientes constantes}, debido a la propiedad de linealidad de la transformada.

% A continuación se ilustra el procedimiento a seguir mediante ejemplos.

% \begin{ejemplo}
%     Encuentre la solución general de la ecuación diferencial
%     $$y''(x) - y(x) = e^{-\alpha |x|}, \quad y(\pm \infty) = 0, \quad \alpha > 0, \alpha \neq 1.$$

%     \textbf{Solución:} La solución del caso homogéneo 
%     $$y''(x) - y(x) = 0,$$

%     está dada por
%     $$y_h(x) = c_1 e^{x} + c_2 e^{-x}, \quad c_1,c_2 \in \mathbb{R}.$$

%     Nos queda por encontrar la solución particular, para ello haremos uso de la transformada de Fourier. 

%     Primero, determinemos 
    
%     \begin{align*}
%         \mathcal{F}\left\{ e^{-\alpha |x|} \right\} &= \frac{1}{2\pi} \int_{-\infty}^{\infty} e^{-\alpha |x|} e^{-ikx} dx \\
%         &= \frac{1}{2\pi} \left(  \int_{-\infty}^{0} e^{x(\alpha -ikx)} dx +  \int_{0}^{\infty} e^{-x(\alpha + ik)} dx\right) \\
%         &= \frac{1}{2\pi} \left( \frac{1}{\alpha -ik} + \frac{1}{\alpha +ik} \right) \\
%         &= \frac{\alpha/\pi}{\alpha^2 + k^2}.
%     \end{align*}

%     Luego, apliquemos la transformada de Fourier a la ecuación diferencial:
%     \begin{align*}
%         \mathcal{F}\{ y''(x)\} - \mathcal{F}\{y(x)\ &=  \mathcal{F}\left\{ e^{-\alpha |x|} \right\} \\
%         \Rightarrow - k^2 \mathcal{F}\{y(x)\} - \mathcal{F}\{y(x)\} &= \frac{\alpha/\pi}{\alpha^2 + k^2}.
%     \end{align*}

%     Despejando la transformada de Fourier de la solución.
%     \begin{align*}
%         \mathcal{F}\{y(x)\} &= \frac{-\alpha/\pi}{(k^2 + \alpha^2)(k^2+1)} \\
%         &= - \frac{\alpha}{\pi} \frac{1}{\alpha^2-1} \left( \frac{1}{k^2+1} - \frac{1}{k^2+ \alpha^2}\right) \\
%         &= \frac{1}{\alpha^2-1} \left( \frac{\alpha/\pi}{k^2 + \alpha^2} - \alpha \frac{1/\pi}{k^2+1} \right.)
%     \end{align*}

%     Tomando la transformada inversa, obtenemos que
%     $$y(x) = \frac{e^{-\alpha|x| - \alpha e^{-|x|} }}{\alpha^2-1}.$$

%     Por lo tanto, la solución general es
%     $$\boxed{y(x) = \frac{e^{-\alpha|x| - \alpha e^{-|x|} }}{\alpha^2-1} + c_1 e^{x} + c_2 e^{-x}, \quad c_1,c_2 \in \mathbb{R}}$$
% \end{ejemplo}

% \begin{ejemplo}
%   Consideremos un oscilador armónico amortiguado sometido a una fuerza externa $g(t)$. La ecuación de movimiento del oscilador está dada por
% \begin{equation}
%  \ddot{x}(t) + 2 \alpha \dot{x}(t) + \omega_0^2 x(t) = f(t), \label{EDO-Oscilador}   
% \end{equation}

% donde $f(t) = g(t)/m$ y $\alpha$ es una constante asociada al amortiguamiento del sistema. En los primeros cursos de Ecuaciones Diferenciales Ordinarias (EDO) se trabaja con $f(t)$ sinusoidal, pero gracias a la transformada de Fourier, podemos extender este resultado para funciones $f(t)$ arbitrarias. 

% Aplicando la transformada de Fourier en la variable temporal, a saber,
% $$\mathcal{F}\{x(t)\} = \frac{1}{2\pi} \int_{-\infty}^{\infty} f(t) e^{-i\omega t} dt, $$

% a ambos lados de la ecuación diferencial \eqref{EDO-Oscilador}, obtenemos 
% \begin{align}
%     \mathcal{F}\left\{ \ddot{x}(t) + 2 \alpha \dot{x}(t) + \omega_0^2 x(t)\right\} &= \mathcal{F}\left\{ f(t)\right\} \nonumber\\
%     \Rightarrow   \mathcal{F}\left\{ \ddot{x}(t) \right\} + 2\alpha \mathcal{F}\left\{ \dot{x}(t) \right\} + \omega_0^2 \mathcal{F}\{x(t)\} &= \mathcal{F}\left\{ f(t)\right\}. \label{EDO-Transformada}
% \end{align}

% Si asumimos que 
% $$\lim_{x \to \pm \infty} x(t) = \lim_{x \to \pm \infty} \dot{x}(t) = 0,$$

% tenemos 
% \begin{align*}
%      \mathcal{F}\left\{ \ddot{x}(t) \right\} &= (i\omega)^2 \mathcal{F}\{x(t)\} = - \omega^2 \mathcal{F}\{x(t)\},\\
%       \mathcal{F}\left\{ \dot{x}(t) \right\} &= i \omega \mathcal{F}\{x(t)\}.
% \end{align*}

% Además, si definimos $F(\omega) := \mathcal{F}\left\{ f(t)\right\}$, la ecuación \eqref{EDO-Transformada} nos queda
% $$ - \omega^2 \mathcal{F}\{x(t)\} + 2 \alpha \omega i \mathcal{F}\{x(t)\} + \omega_0^2 \mathcal{F}\{x(t)\} = F(\omega).$$

% Despejando la transformada de Fourier de la solución:
% $$  \mathcal{F}\{x(t)\} = \frac{F(\omega)}{-\omega^2 - 2 \alpha i \omega + \omega_0^2}.$$

% Tomando la transformada inversa, obtenemos la solución 
% $$\boxed{x(t) = \int_{-\infty}^{\infty} \frac{F(\omega)}{(\omega_0^2-\omega^2) - 2 \alpha \omega i} e^{i\omega t} d\omega }$$  
% \end{ejemplo}
\input{./Cap-Transformada-de-Laplace.tex}

\nocite{*} %Referencia todo, incluso lo que no está citado.
\printbibliography[title={Referencias}]

\end{document}
