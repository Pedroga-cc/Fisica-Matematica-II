\documentclass[letterpaper,12pt]{report}
\usepackage[spanish]{babel}
% \usepackage[T1]{fontenc}
\usepackage[utf8]{inputenc}
\usepackage{graphicx}
\usepackage{amsfonts,amsmath,color,amssymb,float, amsthm,mathrsfs}  
\usepackage[right=3cm,left=2cm,top=2cm,bottom=2cm,headsep=0.7cm,footskip=0.5cm]{geometry}
\usepackage{enumerate}
\usepackage{wrapfig} 
\usepackage[rflt]{floatflt} 
\usepackage{framed}
\usepackage[most]{tcolorbox}
\usepackage{xcolor} 
\colorlet{shadecolor}{green!20}
\setlength\FrameSep{0.5ex}
\usepackage{thmtools}
\usepackage{esint}
\usepackage{cancel}
\usepackage{listings} 
\usepackage{pstricks, caption}
\usepackage[colorlinks]{hyperref}
\usepackage{csquotes}
\usepackage{fullpage}
\usepackage{enumitem}
\usepackage{etoolbox}
\usepackage{tcolorbox}
\usepackage{tikz}
\usepackage{multicol}
\usetikzlibrary{arrows,babel}
\usepackage[font=small]{caption}
\usepackage{mathtools}

\usepackage[LGR,T1]{fontenc} %% LGR encoding is needed for loading the package gfsneohellenic


\decimalpoint
\newcommand{\grad}{^\circ}
\newlength{\drop}
\DeclareMathOperator{\sign}{sgn}
\DeclareMathOperator{\Log}{Log}
\providecommand{\norm}[1]{\lVert#1\rVert}
\DeclareMathOperator{\real}{Re}
\DeclareMathOperator{\im}{Im}

%%%%%%%%%Entornos: Teoremas, Defs,etc%%%%%%%
\theoremstyle{definition}

\newtheorem{defis}{Definiciones}[section]
\newtheorem{corolario}{Corolario}[section] 

\newtheorem{protoexample}{Ejemplo}[section]
\newenvironment{ejemplo}
   {\colorlet{shadecolor}{orange!15}\begin{shaded}\begin{protoexample}}
   {\end{protoexample}\end{shaded}}

\newtheorem*{protoproof}{Demostración}
\newenvironment{demo}
   {\colorlet{shadecolor}{blue!10}\begin{shaded}\begin{protoproof}}
   {\end{protoproof} \end{shaded}}

\newenvironment{Rdbleftbar}{%
  \def\FrameCommand{\textcolor{red}{\vrule width0.6pt}\hspace{0.15em}\textcolor{red}{\vrule width0.6pt} \hspace{0.5em}}%
  \MakeFramed {\advance\hsize-\width \FrameRestore}}%
 {\endMakeFramed}

 \newenvironment{Bdbleftbar}{%
  \def\FrameCommand{\vrule width0.6pt\hspace{0.15em}\vrule width0.6pt \hspace{0.5em}}%
  \MakeFramed {\advance\hsize-\width \FrameRestore}}%
 {\endMakeFramed}

\newenvironment{Gdbleftbar}{%
  \def\FrameCommand{\textcolor{green}{\vrule width0.6pt}\hspace{0.15em}\textcolor{green}{\vrule width0.6pt} \hspace{0.5em}}%
  \MakeFramed {\advance\hsize-\width \FrameRestore}}%
 {\endMakeFramed}
 
% With parskip

\usepackage[indent=0.6cm]{parskip}

\declaretheorem[%
name            =Teorema,%
numberwithin=chapter,
postheadhook    =\vspace*{\parskip} \begin{Rdbleftbar}\vspace*{-5pt},%
prefoothook     =\vspace*{-1pt}\end{Rdbleftbar}%
]{teorema}

\declaretheorem[%
name            =Definición,%
numberwithin=chapter,
postheadhook    =\vspace*{\parskip}\begin{Bdbleftbar}\vspace*{-5pt},%
prefoothook     =\vspace*{-1pt}\end{Bdbleftbar}%
]{defi}

\declaretheorem[%
name            = Proposición,%
numberwithin=chapter,
postheadhook    =\vspace*{\parskip}\begin{Gdbleftbar}\vspace*{-5pt},%
prefoothook     =\vspace*{-1pt}\end{Gdbleftbar}%
]{propo}  

\declaretheorem[%
name            = Lema,%
numberwithin=chapter,
postheadhook    =\vspace*{\parskip}\begin{Gdbleftbar}\vspace*{-5pt},%
prefoothook     =\vspace*{-1pt}\end{Gdbleftbar}%
]{lema}   
%%%%%%%%%%%%%%%%%%%%%%%%%%%%%%%%%%%%%%%%%%%%%%

%%%%%%%%%%%%Fancy Chapters%%%%%%%%%%%%%%%%%
%Options: Sonny, Lenny, Glenn, Conny, Rejne, Bjarne, Bjornstrup
\usepackage[Bjornstrup]{fncychap}
% \usepackage{tgschola}

\usepackage{libertinust1math}
\usepackage{montserrat}

\ChTitleVar{\bfseries\huge\sffamily\fontfamily{Montserrat-LF}\selectfont}
%%%%%%%%%%%%%%%%%%%%%%%%%%%%%%%%%%%%%%%%%%%

% \usepackage[book]{FiraSans}

\usepackage[default]{sourcesanspro}

\usepackage{titlesec}

\titleformat*{\section}{\Large\bfseries\sffamily\fontfamily{Montserrat-LF}\selectfont}
\titleformat*{\subsection}{\large\slshape\bfseries\sffamily\fontfamily{Montserrat-LF}\selectfont}


%Referencias
\usepackage[backend=biber]{biblatex}
\addbibresource{Referencias.bib}

\begin{document}

\begin{titlepage}
 \drop=0.1\textheight
    \centering
    \vspace*{\baselineskip}
    \rule{\textwidth}{1.6pt}\vspace*{-\baselineskip}\vspace*{2pt}
    \rule{\textwidth}{0.4pt}\\[\baselineskip]
    {\scshape\bfseries\Huge\fontfamily{Montserrat-LF}\selectfont Física Matemática II} \\[0.2\baselineskip]
    \rule{\textwidth}{0.4pt}\vspace*{-\baselineskip}\vspace{3.2pt}
    \rule{\textwidth}{1.6pt}\\[\baselineskip]
    {\Large Autor: \par}
{\large Pedro A. Contreras-Corral \par}
\vfill
{\large Diciembre 2024 \par}
{\large v1.0 \par}
\end{titlepage}

\chapter*{Prefacio}

Este documento ha sido preparado por Pedro Contreras Corral como material de apoyo para el curso Física Matemática II. En cuanto a su contenido, se han utilizado como base las notas de clase de las ocasiones en que el curso fue dictado por el profesor Guillermo Rubilar, y las profesoras Ariana Muñoz e Ivana Sebestova, además del apunte preparado por Alejandro Saavedra como ayudante de esta última, que también fue utilizado como base para el \emph{template} de este documento.


Agradezco los aportes de Lixin Lai y Fernanda Mella al facilitarme sus notas de las clases de la profesora Ivana Sebestova, así como los de José Huenchual por sus notas del curso de la profesora Ariana Muñoz.

\tableofcontents

\input{./Cap-Series-Fourier.tex}
% \input{./Cap-Integrales-Convergencia.tex}
\input{./Cap-Transformada-de-Fourier.tex}
\chapter{Ecuaciones Diferenciales en Física}

En física, es muy común que diversas situaciones sean modeladas no por ecuaciones que únicamente contengan potencias (enteras o semienteras) de alguna variable física, sino que incluyan derivadas de estas.

En sus cursos de Mecánica y de Ecuaciones Diferenciales, probablemente se familiarizaron con los casos del Oscilador Armónico y del Oscilador amortiguado, que son descritos por las ecuaciones \eqref{eq:mas} y \eqref{eq:amortiguado}, respectivamente.

\begin{gather}
    \frac{d^2x}{dt^2} = - \omega^2 x \ , \label{eq:mas} \\
    \frac{d^2x}{dt^2} + 2\gamma \frac{dx}{dt} + \omega^2_0 x = 0 \ . \label{eq:amortiguado}
\end{gather}

Ambas ecuaciones consisten en ecuaciones diferenciales \emph{ordinarias}, debido a que la función a derivar, $x$, depende de una única variable, de modo que todas las derivadas en ella son totales.

Sin embargo, muchas otras situaciones físicas no pueden ser descritas únicamente en términos de funciones de una sola variable, incluso cuando dicha función solo dependa de la posición. Por ejemplo, podríamos querer describir el potencial eléctrico de una distribución de cargas esférica, cuya densidad dependa de qué tan alejados de su centro nos encontremos (variable $r$), así como también del ángulo que forma con respecto de su polo norte (variable $\theta$), de modo que este potencial será una función tanto de $r$ como de $\theta$. Este tipo de sistemas serán descritos por ecuaciones diferenciales \emph{parciales} (EDPs).

Hasta el día de hoy, el desarrollo de métodos para resolver EDPs es un área de investigación activa en matemáticas, por lo que en este curso nos limitaremos a algunos de los métodos más tradicionales y que son la base para la descripción de la física de los siglos XVIII y XIX, como lo son la mecánica hamiltoniana, la teoría clásica de campos o la electrodinámica clásica.

\section{Algunas ecuaciones básicas}

\subsection{Ecuación de Laplace}

Esta corresponde a la expresión
\begin{equation}
    \nabla^2 \Phi = 0 \ ,
\end{equation}
la que surge en el estudio de diferentes sistemas físicos, como por ejemplo:
\begin{itemize}
    \item el \textbf{potencial electrostático} en una región sin cargas. 
    \item un \textbf{fluido irrotacional incompresible} en un movimiento estacionario, cuyo campo de velocidades es descrito como $\vec{v} = -\nabla \Phi$.
    \item el \textbf{potencial gravitatorio}.
    \item una \textbf{distribución de temperatura estacionaria}, donde en este caso $\Phi = T(x,t)$ corresponde al campo de temperaturas de un material.
\end{itemize}

\subsection{Ecuación de Poisson}

Esta corresponde, por llamarlo de una manera, a la generalización de la ecuación de Laplace para cualquier sistema con una función \emph{fuente} $f$ conocida. Es dada por la expresión
\begin{equation}
    \nabla^2\Phi = f(\vec{x}) \ .
\end{equation}

Esta corresponde a una ecuación \emph{inhomogénea}, cuyas soluciones generales pueden escribirse como $\Phi = \Phi_h + \Phi_p$, donde el primer término corresponde a la solución de la \emph{ecuación homogénea} donde $f(x) = 0$, que en este caso corresponde a la ecuación de Laplace, y el segundo término es una \emph{solución particular} de la ecuación de Poisson.

Esta puede surgir en situaciones similares a la ecuación de Laplace, pero en las cuales existen \emph{fuentes} para los respectivos campos. Por ejemplo, para el caso electrostático, puede existir una fuente con densidad de carga $\rho(\vec{x})$, de modo que la ecuación que describe el potencial eléctrico en dicha región será
\begin{equation}
    \nabla^2 \phi = - \frac{\rho(\vec{x})}{\varepsilon_0} \ .
\end{equation}

\subsection{Ecuación de Helmholtz}


\subsection{Ecuación de difusión de calor dependiente del tiempo}


\subsection{Ecuación de onda dependiente del tiempo}


\subsection{Ecuación de Klein-Gordon}


\subsection{Ecuación de Schrödinger dependiente del tiempo}

\subsection{Ecuaciones de Maxwell}


\section{Condiciones de borde}

\section{Encontrando soluciones para ecuaciones diferenciales parciales}

Como se mencionó al inicio de este capítulo, las EDPs siguen siendo un área de investigación activa en matemáticas, por lo que no todas ellas tendrán soluciones exactas, o analíticas. Un ejemplo de esto son las \emph{ecuaciones de Navier-Stokes}, que describen el comportamiento de un fluido viscoso, y surgen en el estudio de sistemas como la atmósfera terrestre o las corrientes oceánicas. Estas ecuaciones son particularmente conocidas por ser uno de los \emph{problemas del milenio}, de modo que existe una recompensa monetaria para quien pueda demostrar la existencia (o inexistencia) de soluciones analíticas para cualquier conjunto de condiciones de borde.

Sin embargo, las ecuaciones que listamos en secciones previas sí puden ser resueltas de forma analítica mediante uno de tres métodos que estudiaremos en el curso. Dos de ellos, el método de separación de variables y el método de las funciones de Green los veremos en capítulos siguientes del curso. El tercero corresponde al \textbf{método de las transformadas integrales}, donde diferentes transformadas son útiles para diferentes condiciones de contorno. En este curso, analizaremos únicamente a la transformada de Fourier.

\subsection{Método de la transformada de Fourier}
\chapter{El Método de Separación de Variables}

Ya observamos que podemos hacer uso del método de la transformada de Fourier para reducir una ecuación diferencial en dos variables diferentes a una EDO en una sola variable. Sin embargo, no siempre será cómodo calcular la transformada de Fourier de una función, por lo que sería agradable tener una forma más general de hacer funcionar esta idea.

Para ello, hacemos uso del \textbf{método de separación de variables}, gracias al cual podemos reducir una EDP lineal de $n$ variables en un conjunto de $n$ EDOs para $n$ funciones auxiliares, cada una asociada a una variable independiente de la EDP. Esto lo podemos hacer al proponer que nuestro sistema puede ser descrito mediante una \emph{solución separable}, que consiste en el producto de todas las funciones auxiliares que encontremos mediante la solución de las EDOs.

Para realizar la separación de las variables, haremos uso de $(n-1)$ \emph{constantes de separación}, las que son escogidas, en primera instancia, de forma arbitraria para luego determinarlas gracias a las condiciones de contorno del sistema.

Finalmente, una vez hallamos las soluciones de cada una de las EDOs y planteamos la solución separable del sistema, consideraremos que la solución más general consiste en la \emph{superposición} o \emph{combinación lineal} de todas las posibles soluciones separables del sistema.

Como ejemplo bastante ilustrativo, durante el capítulo resolveremos la ecuación de Helmholtz en diferentes sistemas coordenados, dando origen a diferentes \emph{funciones especiales}, que serán discutidas en mayor profundidad en los capítulos siguientes.

\section{Resolviendo la ecuación de Helmholtz}

Recordemos que la ecuación de Helmholtz es dada por la expresión
\begin{equation}\label{eq:Helmholtz}
    \nabla^2 \psi + k^2 \psi = 0 \ ,
\end{equation}
donde $k$ es una constante asociada al sistema.

\subsection{Coordenadas cartesianas}

En un sistema cartesiano, el operador laplaciano se define simplemente como
\begin{equation}
    \nabla^2 = \frac{\partial^2}{\partial x^2} + \frac{\partial^2}{\partial y^2} + \frac{\partial^2}{\partial z^2} \ .
\end{equation}

De esta forma, dado que nuestra ecuación posee tres variables independientes, podremos reducirla a un sistema de 3 EDOs en las variables $x$, $y$ y $z$. Antes de realizar este procedimiento, plantearemos una solución de la forma
\begin{equation}\label{eq:ansatz}
    \psi(x,y,z) = X(x)Y(y)Z(z) \ .
\end{equation}

Una duda totalmente razonable es por qué considerar una solución de este estilo. La verdad no hay una respuesta directa a esto, más allá de ``\emph{esperemos que funcione}''. Si no fuera el caso, y nuestro sistema comienza a complicarse, puede que sea una mejor idea utilizar algún método alternativo. Sin embargo, cabe mencionar que si los operadores diferenciales (derivadas $n$-ésimas) son aditivos, es decir, no tenemos combinaciones de las variables, una solución de este estilo suele funcionar.

Evaluando la expresión \eqref{eq:ansatz} en la ecuación \eqref{eq:Helmholtz}, podemos escribirla como
\begin{equation}
    YZ \frac{d^2X}{dx^2} + XZ \frac{d^2 Y}{dy^2} + XY \frac{d^2Z}{dz^2} + k^2 XYZ = 0 \ ,
\end{equation}
donde ahora utilizamos derivadas totales en lugar de parciales, puesto que cada una de las funciones depende únicamente de una variable.

Dividimos ahora por la solución $XYZ$, donde hemos asumido que $\psi(x,y,z) \neq 0$, de modo que, luego de reordenar los términos, la ecuación resulta en
\begin{equation} \label{eq:sep_1}
    \frac{1}{X} \frac{d^2 X}{dx^2} = -k^2 - \frac{1}{Y} \frac{d^2 Y}{dy^2} - \frac{1}{Z} \frac{d^2 Z}{dz^2} \ .
\end{equation}

Hemos llegado al paso en donde se aprecia esta \emph{separación de variables}. Notemos que el lado izquierdo de la ecuación \eqref{eq:sep_1} contiene únicamente términos asociados a la variable $x$, mientras que el lado derecho aún tiene dependencia en $y$ y en $z$. La única posibilidad de que ambos lados sean iguales, dado que dependen de variables distintas, es que ambos son a su vez iguales a una \emph{constante de separación}, que en este caso asumiremos real y llamaremos $\lambda_1$,
\begin{align}
    \frac{1}{X} \frac{d^2 X}{dx^2} = \lambda_1 \ , \\
    -k^2 - \frac{1}{Y} \frac{d^2Y}{dy^2} - \frac{1}{Z} \frac{d^2Z}{dz^2} = \lambda_1 \ . \label{eq:EDO_de_y_z}
\end{align}

Notemos que, reordenando términos en la ecuación \eqref{eq:EDO_de_y_z}, podemos nuevamente separar las variables mediante una constante $\lambda_2$, de modo que hemos \emph{dividido la EDP original, dependiente de tres variables, en un sistema de tres EDOs},
\begin{align}
    \frac{d^2X}{dx^2} - \lambda_1 X & = 0 \ , \label{eq:EDO_de_x}  \\
    \frac{d^2Y}{dy^2} - \lambda_2 Y & = 0 \ , \label{eq:EDO_de_y}  \\
    \frac{d^2Z}{dz^2} + (k^2 + \lambda_1 + \lambda_2) Z & = 0 \ . \label{eq:EDO_de_z} 
\end{align}

Cada una de estas EDOs son resolubles mediante los métodos vistos en su primer curso de ecuaciones diferenciales, y las soluciones dependerán del valor y del signo de las constantes de separación. Analicemos las posibles soluciones de la ecuación \eqref{eq:EDO_de_x}:

\begin{equation}
    X_{\lambda_1}(x) = \left\{
    \begin{array}{cc} 
            c_1 \sinh(\sqrt{\lambda_1} x) + c_2 \cosh(-\sqrt{\lambda_1}x) \ , & \text{si } \lambda_1 > 0 \ , \\
            c_1 + c_2 x \ , & \text{si } \lambda_1 = 0 \ , \\
            c_1 \cos(\sqrt{-\lambda_1}x) + c_2 \sin(\sqrt{-\lambda_1}x) \ , & \text{si } \lambda_1 < 0 \ .
    \end{array}
    \right.
\end{equation}

Aquí es importante no olvidar que tenemos una motivación física para realizar estos cálculos, lo que nos ayudará a determinar el signo de la constante de separación. Para la mayoría de los problemas físicos, la solución para $X(x)$ que tiene sentido es aquella en que $\lambda_1$ es negativo, de modo que la solución es una oscilación en la coordenada $x$. Así mismo, podemos hacer un análisis análogo para cada una de las otras ecuaciones, obteniendo soluciones denotadas por $Y_{\lambda_2}(y)$ y $Z_{\lambda_1 \lambda_2}(z)$, donde es importante indicar el subíndice, ya que esta solución es válida para un valor particular de las constantes de separación.

De esta forma, una solución particular a nuestra EDP es dada por
\begin{equation} \label{eq:sol_particular_cartesianas}
    \psi_{\lambda_1 \lambda_2}(x,y,z) = X_{\lambda_1}(x) Y_{\lambda_2}(y) Z_{\lambda_1 \lambda_2}(z) \ ,
\end{equation}
y la solución general corresponderá a una combinación lineal de la solución \eqref{eq_sol_particular_cartesianas}, correspondiendo a una suma sobre todos los valores posibles de $\lambda_1$ y $\lambda_2$, es decir,
\begin{equation}
    \psi(x,y,z) = \sum_{\lambda_1 \lambda_2} C_{\lambda_1 \lambda_2} \psi_{\lambda_1 \lambda_2}(x,y,z) \ ,
\end{equation}
donde los coeficientes $C_{\lambda_1 \lambda_2}$ serán obtenidos al imponer las condiciones de contorno del problema, que por lo general nos llevará a un conjunto finito de valores para $\lambda_1$ y $\lambda_2$.

\subsection{Coordenadas cilíndricas}
\chapter{El método de series para EDO y una introducción al problema de Sturm-Liouville}

\section{Resolver EDOs por el método de series}

\section{El problema de Sturm-Liouville}

Para el caso de una EDO de segundo orden, lineal (no hay combinación de derivadas en un mismo término) y homogénea (igualable a 0) de la forma
\begin{equation} \label{eq:Sturm-Liouville}
    \frac{d}{dx}\left[ p(x) \frac{dy}{dx} \right] + q(x) y + \lambda r(x)y = 0 \ , \quad a \leq x \leq b
\end{equation}
donde $p(x)$, $p'(x)$, $q(x)$ y $r(x)$ son funciones reales y continuas en el intervalo $[a,b]$ y $\lambda$ es un parámetro a determinar se conoce como una \textbf{ecuación de Sturm-Liouville}. Cabe mencionar que, desde una perspectiva matemáticamente formal, deben también satisfacerse que tanto $p(x)$ como $r(x)$ son funciones definidas positivas en $[a,b]$. A la función $r(x)$ se le denomina \emph{función peso}.

Si revisamos el capítulo anterior, podemos darnos cuenta que todas las EDO que surgen de separar la ecuación de Helmholtz en diferentes sistemas coordenados corresponden a ecuaciones de Sturm-Liouville, donde ya hemos definido diferentes constantes de separación, salvo $\lambda$.

¿Por qué nos interesa esto? Recordamos que un \emph{operador diferencial} es un operador lineal, es decir, puede actuar sobre un elemento de un espacio vectorial (en este caso, una función) y transformarlo en otro elemento del mismo espacio vectorial. Gracias a ello, podemos definir el \emph{operador de Sturm-Liouville} como
\begin{equation}
    \mathcal{L} = \frac{d}{dx}\left[p(x) \frac{d}{dx}\right] - q(x) \ ,
\end{equation}
de modo que podemos escribir la ecuación de Sturm-Liouville como
\begin{equation}
    \mathcal{L}y = -\lambda r(x) y \ ,
\end{equation}
que corresponde a la expresión de un problema de autovalores, como los vistos en Álgebra Lineal. Por ende, nos interesaría poder determinar el valor de $\lambda$ a partir de la teoría de operadores, pero esta perspectiva escapa a los objetivos del curso. Sin embargo, podrán encontrar en el apéndice X un desarrollo siguiendo este camino.

La perspectiva que sí nos interesa analizar en este curso, es uno de los resultados de entender la ecuación de Sturm-Liouville como un problema de autovalores: la \textbf{ortogonalidad} de sus autofunciones. Si recordamos, una propiedad de los autovectores de un operador es que, cuando están asociados a diferentes autovalores, ellos son ortogonales entre sí. Una \emph{autofunción} es el equivalente a un autovector en el espacio de funciones, de modo que, usando un producto interno apropiado, las funciones asociadas a dos autovalores $\lambda_m$ y $\lambda_n$, denotadas como $y_m$ e $y_n$ respectivamente, satisfacen la relación
\begin{equation}
    \int_a^b w(x) y_m(x) y_n(x) = 0 \ , \quad n \neq m \ .
\end{equation}

En el caso del problema de Sturm-Liouville, la función peso que definimos en el producto interno corresponde a la función peso $r(x)$ del problema, de modo que sus soluciones deberán satisfacer
\begin{equation}
    \int_a^b r(x) y_m(x) y_n(x) = 0 \ , \quad n \neq m \ .
\end{equation}

También es posible mostrar que los autovalores de esta ecuación siempre serán valores reales, demostración que se encuentra en el apéndice X.

\section{Condiciones de contorno de Sturm-Liouville}

Sea un conjunto de condiciones de contorno \emph{homogéneas} (es decir, si tengo un conjunto de funciones $f_i$ que satisfacen la condición de contorno, cualquier combinación lineal de ellas también la satisface) en $x=a$ y en $x=b$. Si para dos funciones cualesquiera, $f(x)$ y $g(x)$, que satisfacen estas condiciones de contorno se verifica que
\begin{equation} \label{eq:SM-condicion}
    p(x)  \left.  \left[ f^\ast(x) \frac{dg}{dx} - g(x)\frac{df^\ast}{dx} \right]\right|_a^b = 0 \ ,
\end{equation}
entonces dichas condiciones de contorno se denominan \textbf{condiciones de contorno de Sturm-Liouville}, las que pueden clasificarse en tres clases, que discutiremos a continuación.

\subsection{Condiciones de contorno periódicas}

En este caso, nos referimos a condiciones de contorno de la forma
\begin{equation}
    \begin{dcases}
        y(a) = y(b) \ , \\
        y'(a) = y'(b) \ ,
    \end{dcases}
\end{equation}
donde además se verifica que $p(a) = p(b)$. En particular, esta última condición es necesaria para que se satisfaga \eqref{eq:SM-condicion}.

\subsection{Condiciones de contorno de tipo Robin, o regulares}

Estas condiciones son de la forma
\begin{equation}
    \begin{dcases}
        \alpha_1 y(a) + \alpha_2 y'(a) = 0 \ , \quad \alpha_1, \alpha_2 \in \mathbb{R} \ , \\
        \beta_1 y(b) + \beta_2 y'(b) = 0 \ , \quad \beta_1, \beta_2 \in \mathbb{R} \ , 
    \end{dcases}
\end{equation}
donde $\alpha_1$ y $\alpha_2$ no son iguales a cero simultáneamente, ni tampoco lo son $\beta_1$ y $\beta_2$.

Este caso corresponde a una generalización (más precisamente, una combinación lineal) de las condiciones de Dirichlet y de Neumann. En efecto, observamos que cuando $\alpha_2 = \beta_2 = 0$, recuperamos las condiciones de Dirichlet, $y(a) = y(b) = 0$; mientras que al hacer $\alpha_1 = \beta_1 = 0$, recuperamos las condiciones de Neumann, $y'(a) = y'(b) = 0$.

Se puede comprobar que estas condiciones satisfacen la ecuación \eqref{eq:SM-condicion} tomando complejo conjugado en la primera ecuación del sistema, y observando que se puede interpretar como un sistema de ecuaciones de dos incógnitas, es decir,
\begin{equation}
    \begin{pmatrix}
        f^\ast(a) & {f^\ast}'(a) \\ g(a) & g'(a)
    \end{pmatrix}
    \begin{pmatrix}
        \alpha_1 \\ \alpha_2
    \end{pmatrix}
    = 
    \begin{pmatrix}
        0 \\ 0
    \end{pmatrix} \ ,
\end{equation}
donde el sistema tendrá infinitas soluciones, de modo que
\begin{equation}
    \left| \begin{array}{cc}
        f^\ast(a) & {f^\ast}'(a) \\ g(a) & g'(a)
    \end{array}
    \right| = f^\ast(a) g'(a) - g(a) {f^\ast}'(a) = 0 \ .
\end{equation}

Mediante un procedimiento análogo para $b$, tenemos que
\begin{equation}
    f^\ast(b) g'(b) - g(b) {f^\ast}'(b) = 0 \ ,
\end{equation}
de modo que se satisface la expresión \eqref{eq:SM-condicion}.

\subsection{Condiciones de contorno singulares}

Corresponden a aquellas soluciones que no son ni regulares ni periódicas. Muchas veces corresponden a funciones $p(x)$ que se anulan en los contornos, o coeficientes que se vuelven infinitos en uno o ambos contornos.

Como caso ilustrativo, consideremos el caso en que $p(a) = p(b) = 0$. En este caso, diremos que $x=a$ y $x=b$ son \emph{puntos singulares} de la ecuación de Sturm-Liouville, como se observa al escribir la ecuación \eqref{eq:Sturm-Liouville} en la forma
\begin{equation}
    y''(x) + \frac{p'(x)}{p(x)} y'(x) + \frac{q(x)}{p(x)} y(x) + \lambda \frac{r(x)}{p(x)} y(x) = 0 \ .
\end{equation}

De esta forma, deberá cumplirse que, en general, las soluciones también sean singulares en estos puntos.
\chapter{Funciones de Legendre}

Como vimos anteriormente en el capítulo \ref{chap:MSV}, al resolver la ecuación de Helmholtz en coordenadas esféricas, podemos obtener dos ecucaciones, según el valor de la constante de separación $m^2$. 

\section{Ecuación de Legendre}

Analizaremos en primer lugar el caso en que $m=0$, correspondiente a resolver la ecuación de Laplace, donde obtenemos la EDO
\begin{equation}
    \frac{1}{\sin\theta} \frac{d}{d\theta}\left( \sin\theta \frac{d\Theta}{d\theta} \right) + \lambda \Theta = 0 \ .
\end{equation}

Por comodidad, consideremos el cambio de variables $x = \cos\theta$, y llamaremos a $\Theta(\theta) = \Theta(\arccos(x)) = y(x)$, de modo que
\begin{equation}
    \frac{d}{d\theta} = \frac{dx}{d\theta} \frac{d}{dx} = -\sin\theta \frac{d}{dx} \ ,
\end{equation}
con lo que podremos escribir la ecuación como
\begin{equation}
    \frac{1}{\sin\theta}(-\sin\theta)\frac{d}{dx}\left( \sin\theta (-\sin\theta) \frac{dy}{dx} \right) + \lambda y = \frac{d}{dx}\left( (1-x^2) \frac{dy}{dx} \right) + \lambda y = 0 \ ,
\end{equation}
expresión que nombraremos como \textbf{ecuación de Legendre}.

Observamos que hemos escrito la ecuación de Legendre en la forma de una ecuación de Sturm-Liouville, con $p(x) = 1-x^2$ y $q(x) = r(x) = 1$. Observamos que, dado que $x=\cos\theta$, nuestra ecuación está definida en el intervalo $[-1,1]$. Nos interesa que estos casos extremos posean soluciones finitas, lo que nos permitirá encontrar el valor de $\lambda$ para esta ecuación.

\subsection{Resolviendo la ecuación de Legendre}

En primer lugar, escribimos la ecuación de forma más explícita, es decir,
\begin{equation}\label{eq:legendre_explicit}
    (1-x^2)y''(x) - 2x y'(x) + \lambda y(x) = 0 \ .
\end{equation}

Utilizando el método de series, planteamos soluciones de la forma
\begin{equation}
    \begin{dcases}
        y(x) & = \sum_{k=0}^{\infty} a_k x^k \\
        y'(x) & = \sum_{k=1}^{\infty} k a_k x^{k-1} \\
        y''(x) & = \sum_{k=2}^{\infty} (k-1)k a_k x^{k-2} \\
        & = \sum_{k=0}^{\infty} (k+1) (k+2) a_{k-2} x^k 
    \end{dcases} \ ,
\end{equation}
de modo que sustituyendo en la expresión \eqref{eq:legendre_explicit} tenemos 
\begin{align}
    \sum_{k=0}^{\infty} (k+1)(k+2) a_{k+2} x^k - \sum_{k=0}^{\infty} (k-1) k a_k x^k - 2 \sum_{k=1}^{\infty} k a_k x^k + \lambda \sum_{k=0}^{\infty} a_k x^k & = 0 \ , \\
    \sum_{k=0}^{\infty} \left[ (k+1)(k+2) a_{k+2} - k(k-1)a_k  - 2k a_k + \lambda a_k \right]x^k & = 0 \ ,
\end{align}
donde hemos usado que $\sum_{k=1}^{\infty} k = \sum_{k=0}^{\infty} k$. Dado que $x^k = 0$ corresponde a la solución trivial, analizamos la relación entre los coeficientes de nuestra ecuación,
\begin{equation}
    (k+1)(k+2) a_{k+2} - [(k-1) k + 2k - \lambda]a_k = 0 \ ,
\end{equation}
de donde encontramos la siguiente relación de recurrencia,
\begin{equation}\label{eq:recurrencia_legendre}
    a_{k+2} = \frac{k(k+1) - \lambda}{(k+1)(k+2)}a_k \ ,
\end{equation}
de modo que dados unos valores para $a_0$ y $a_1$ podremos determinar todos los coeficientes de nuestra expresión. Luego, si desarrollamos los términos pares e impares por separado, podemos observar la siguiente recurrencia,
\begin{align*}
    a_2 & = \frac{-\lambda}{2} a_0 \\
    a_4 & = \frac{2(2+1) - \lambda}{3 \cdot 4} a_2 = \frac{-\lambda(2(2+1)-\lambda)}{4!}a_0 \\
    a_6 & = \frac{4(4+1) - \lambda}{5 \cdot 6} a_4 = \frac{-\lambda (2(2+1)-\lambda)(4(4+1)-\lambda)}{6!} a_0 \\
    & \qquad \vdots \\
    a_n & = \frac{a_0}{n!}\prod_{\ell=0}^{n-2}[\ell (\ell+1) - \lambda] \ ,
\end{align*}
mientras que para el caso de coeficientes impares,
\begin{align*}
    a_3 & = \frac{1(1+1) - \lambda}{2 \cdot 3}a_1 \\
    a_5 & = \frac{(3(3+1)-\lambda)}{4 \cdot 5}a_3 = \frac{(1(1+1) - \lambda)(3(3+1)-\lambda)}{5!} a_1 \\
    a_7 & = \frac{5(5+1) - \lambda}{6 \cdot 7} = \frac{(1(1+1) - \lambda)(3(3+1)-\lambda)(5(5+1) - \lambda)}{7!} a_1 \\
    & \qquad \vdots \\
    a_n & = \frac{a_1}{n!}\prod_{\ell=1}^{n-2}[\ell (\ell+1) - \lambda] \ ,
\end{align*}
de modo que la solución general para la ecuación de Legendre será la combinación lineal de la solución para los valores pares y la solución para los valores impares,
\begin{align}
    y(x) & = a_0 y_{1}(x) + a_1 y_{2}(x) \\
    & = a_0 \left( \sum_{m=0}^{\infty} b_{m} x^{2m} \right) + a_1 \left( \sum_{m=0}^{\infty} c_{m} x^{2m+1} \right) \\
    & = \sum_{m=0}^{\infty} \frac{a_0}{(2m)!}\left( \prod_{\ell=0}^{2m-2}[\ell (\ell+1) - \lambda] \right)x^{2m} + \sum_{m=0}^{\infty} \frac{a_1}{(2m+1)!}\left( \prod_{\ell=1}^{2m-1}[\ell (\ell+1) - \lambda] \right)x^{2m+1} \ .
\end{align}

Con esto, podemos darnos cuenta de que la solución general de la ecuación de Legendre es una serie infinita. ¿Cómo imponemos la condición de finitud a estas expresiones? Podemos hacerlo al estudiar su convergencia para el intervalo en que se encuentran definidas.

Según el criterio de la razón (véase el capítulo 7 del apunte de Física Matemática I), la serie converge para $|x|<1$, independientemente del valor de $\lambda$. Nos queda analizar los casos extremos, es decir, $x=\pm1$.

Para $\lambda \neq n(n+1)$, la serie diverge, puesto que ella se convierte en una serie numérica sin ningún tipo de restricción, que crece indefinidamente. Volveremos a este caso más tarde.

Para $\lambda = n (n+1)$, existirá algún término $a_n=0$ para la solución par o la solución impar, y siguiendo la relación de recurrencia \eqref{eq:recurrencia_legendre}, todo término superior a este se anulará. Luego, basta imponer que el primer coeficiente de la solución de paridad opuesta se anule para así obtener una solución convergente. Es decir, si $n$ es par, $a_1=0$, mientras que si $n$ es impar, escogemos $a_0 = 0$.

De esta forma, nos gustaría poder escribir una expresión general para las soluciones que son válidas en todo el dominio, es decir, cuando $\lambda = n(n+1)$. Notamos que $\ell(\ell+1)-n(n+1) = (\ell - n)(\ell + n + 1)$, podemos encontrar que, para $n$ par,
\begin{align}
    y_{1,n}(x) & = \sum_{m=0}^{n/2} b_m x^{2m} = \sum_{m=0}^{n/2} \left(\frac{1}{(2m)!} \prod_{\ell=0}^{2m-2} (\ell-n)(\ell+n+1) \right) x^{2m} \\
    & = d_{n,1} P_n(x) = \frac{(-1)^{n/2} 2^n \left[ \left(\frac{n}{2}\right)! \right]^2}{n!}P_n(x) \ ,
\end{align}
mientras que para $n$ impar,
\begin{align}
    y_{2,n}(x) & = \sum_{m=0}^{n/2} b_m x^{2m} = \sum_{m=0}^{n/2} \left(\frac{1}{(2m)!} \prod_{\ell=0}^{2m-2} (\ell-n)(\ell+n+1) \right) x^{2m} \\
    & = d_{n,2} P_n(x) = \frac{(-1)^{(n+1)/2} 2^{n-1} \left[ \left(\frac{n-1}{2}\right)! \right]^2}{n!}P_n(x) \ ,
\end{align}
donde en ambos casos las funciones $P_n(x)$ corresponden a los \textbf{polinomios de Legendre}, definidos como
\begin{equation}
    P_n(x) = \sum_{k=0}^{n/2} \frac{(-1)^k (2n-2k)}{2^n k! (n-k)! (n-2k)!}x^{n-2k} \ .
\end{equation}

\subsection{Función Generatriz}

Para cualquier conjunto de \emph{polinomios ortogonales} $\{P_n(x)\}_{n\in \mathbb{N}}$, se denomina \textbf{función generatriz} (o generadora) de dicho conjunto a la función $G(x,t)$, tal que cuando se desarrolla en una serie de Taylor para $t$, los coeficientes de dicha expansión son los polinomios $P_n(x)$:
\begin{equation}
    G(x,t) = \sum_{n=0}^{\infty} P_n(x)t^n \ .
\end{equation}

En particular, la función generatriz para los polinomios de Legendre es dada por la expresión
\begin{equation}
    G(x,t) = \frac{1}{\sqrt{1-2xt + t^2}} \ .
\end{equation}

\subsection{Propiedades de los Polinomios de Legendre}


\subsection{Funciones de Legendre de segunda especie}


\section{Ecuación asociada de Legendre}

Hasta ahora, analizamos el caso en que $m=0$ en la EDO axial \eqref{eq:Helmholtz_esferica} que se obtiene de la ecuación de Helmholtz. En lo que resta del capítulo, analizaremos el caso en que $m \neq 0$.

En este caso, la ecuación se puede escribir como
\begin{equation}
    \frac{1}{\sin\theta} \frac{d}{d\theta}\left( \sin\theta \frac{d\Theta}{d\theta} \right) + \left(\lambda - \frac{m^2}{\sin^2\theta} \right) \Theta = 0 \ .
\end{equation}

Realizando nuestro cambio de variable $x = \cos\theta$, y desarrollando de forma más explícita la ecuación, esta toma la forma de la \textbf{ecuación asociada de Legendre},
\begin{equation}
    (1-x^2) y''(x) - 2xy'(x) + \left( \lambda - \frac{m^2}{1-x^2} \right) y(x) = 0 \ ,
\end{equation}
cuyas soluciones se encuentran definidas en el intervalo $[-1,1]$.

\subsection{Resolviendo la ecuación asociada de Legendre}

Para hacer más fácil el proceso de utilizar el método de series, realizamos la sustitución $y(x) = (1-x^2)^{m/2} u(x)$, y resolvemos para $u(x)$. Esto resulta en la EDO
\begin{equation}
    (1-x^2)u''(x) - 2x(m+1)u'(x) + (\lambda - m(m+1))u(x) = 0 \ ,
\end{equation}
a partir de la cual podemos plantear una solución de la forma
\begin{equation}
    u(x) = \sum_{k=0}^{+\infty} a_k x^k 
\end{equation} 
y encontrar la siguiente relación de recurrencia
\begin{equation}\label{eq:recurrencia_asociadas}
    a_{k+2} = \frac{k^2 + (2m+1)k - \lambda + m(m+1)}{(k+1)(k+2)} a_k \ ,
\end{equation}
que al igual que para la ecuación de Legendre, resultará en series que convergen para $|x|<1$ y, en general, divergen para $|x|=1$, salvo para ciertos valores de $\lambda$. Para encontrar dichos valores, necesitamos que el numerador de la ecuación \eqref{eq:recurrencia_asociadas} sea cero, es decir,
\begin{equation}
    \lambda = m(m+1) + k(k+1) + 2mk = (m+k)(m+k+1) \ ,
\end{equation}
con lo que, definiendo $n = m+k \geq m$, observamos que nuevamente requeriremos que $\lambda = n(n+1)$, de modo que la serie terminará luego del $(n-m)$-ésimo término.

Podemos hallar de forma explícita las soluciones a esta EDO diferenciando repetidamente la ecuación de Legendre \eqref{eq:legendre_explicit}, al hacer uso de la regla de Leibniz, es decir,
\begin{equation}
    \frac{d^m}{dx^m}\left( f_1(x) \cdot f_2(x) \right) = \sum_{s=0}^{m} \binom{m}{s} \frac{d^{m-s}}{dx^{m-s}}f_1(x) \frac{d^s f_2(x)}{dx^s} \ ,
\end{equation}
de donde obtenemos que
\begin{align*}
    \frac{d^m}{dx^m}\left[ (1-x^2)y''(x) \right] & = \binom{m}{m} (1-x^2) \frac{d^m }{dx^m}y'' + \binom{m}{m-1} (-2x) \frac{d^{m-1}}{dx^{m-1}} y'' + \binom{m}{m-2} (-2) \frac{d^{m-2}}{dx^{m-2}}y'' \\
    & = (1-x^2) \left(\frac{d^m y}{dx^m}\right)'' - 2mx \left(\frac{d^m y}{dx^m}\right)' - m(m-1)\left(\frac{d^m y}{dx^m}\right)  \\
    \frac{d^m}{dx^m}\left[ 2x y'(x) \right] & = \binom{m}{m} 2x \frac{d^m}{dx^m}y' + \binom{m}{m-1} 2 \frac{d^{m-1}}{dx^{m-1}}y' \\
    & = 2x \left(\frac{d^2 y}{dx^2}\right)' + 2m \left(\frac{d^2 y}{dx^2}\right) \\
    \frac{d^m}{dx^m}\left[ n(n+1)y'(x) \right] & = n(n+1) \frac{d^m}{dx^m}y(x) \ ,
\end{align*}
donde hemos usando que $\binom{m}{m} = 1$, $\binom{m}{m-1} = m$, y $\binom{m}{m-2} = \frac{m(m-1)}{2}$. Así, la EDO resulta en
\begin{equation}
    (1-x^2)\left(\frac{d^m y}{dx^m}\right)'' - (2mx + 2x) \left(\frac{d^m y}{dx^m}\right)' + [m(m-1) + 2m + n(n+1)]\left(\frac{d^m y}{dx^m}\right) = 0 \ ,
\end{equation}
donde al hacer la sustitución $f(x) = y^{(m)}(x)$, obtenemos la EDO para $u(x)$ cuando $\lambda = n(n+1)$. De esta forma, concluímos que las funciones $u(x)$ son proporcionales a $\frac{d^m }{dx^m} P_n(x)$.

De forma convencional (por motivos de normalización), las soluciones $u(x)$ corresponden a los \textbf{polinomios asociadas de Legendre}, expresados como
\begin{equation} \label{eq:asociadas_legendre}
    P_n^m(x) = (1-x^2)^{m/2} \frac{d^m}{dx^m}P_n(x) \ .
\end{equation}

Observamos que cuando $m=0$, recuperamos los polinomios de Legendre, de modo que $P^0_n(x) \equiv P_n(x)$. Sustituyendo la fórmula de Rodrigues para $P_n(x)$ en \eqref{eq:asociadas_legendre}, obtenemos la \emph{fórmula de Rodrigues para los polinomios asociados de Legendre},
\begin{equation}
    P_n^m(x) = \frac{1}{2^n n!}(1-x^2)^{m/2} \frac{d^{n+m}}{dx^{n+m}}(x^2-1)^n \ ,
\end{equation}
que también es válida para $m<0$.

\subsection{Función generatriz}

\subsection{Propiedades}



\section{Armónicos Esféricos}
\chapter{Funciones de Bessel}

Cuando reolvemos la ecuación de Helmholtz en coordenadas cilíndricas, obtenemos una EDO para la coordenada radial de la forma
\begin{equation}
    \rho \frac{d}{d\rho}\left( \rho \frac{dP}{d\rho} \right) + (n^2\rho^2 - m^2)P = 0 \ .
\end{equation}

Haciendo el cambio de variable $x = n\rho$, $\frac{d}{dx} = \frac{1}{n} \frac{d}{d\rho}$, con lo que la ecuación tomará la forma
\begin{equation}
    \frac{x}{n} n \frac{d}{dx} \left( \frac{x}{n} n \frac{dy}{dx} \right) + (x^2 - m^2)y(x) = 0 \ ,
\end{equation}
o desarrollando más explícitamente la ecuación, obtenemos la \textbf{ecuación de Bessel}
\begin{equation}
    x^2 y''(x) + x y'(x) + (x^2 - m^2) y(x) = 0 \ .
\end{equation}

Si bien, como parte de la ecuación de Helmholtz, se ha impuesto que $m$ es un valor entero, esta no es una restricción propia de la EDO de Bessel, por lo que comúnmente se denota a esta constante como $\nu$, la que puede tomar \emph{valores reales no negativos}.

% Nuevamente, podemos hacer uso del método de Frobenius para resolver la EDO alrededor de $x=0$. Podríamos preguntarnos si esto es posible, ya que $x=0$ corresponde a un punto singular de la ecuación, ya que si consideramos una solución de la forma $y(x) = \sum_n a_n x^n$, en $x=0$, $y(0) = 0$
\chapter{Funciones de Green}

Hasta ahora, hemos visto maneras de resolver EDPs lineales y \emph{homogéneas}, es decir, que son igualables a cero, sin términos que no dependan de la función incógnita ni sus derivadas.

Sin embargo, muchas situaciones físicas no pueden ser descritas únicamente mediante ecuaciones homogéneas. ¿Cómo podemos resolverlas en este caso?

Una forma de hacerlo es mediante el \textbf{método de las funciones de Green}, gracias a las cuales podemos reducir una EDP lineal e inhomogénea a un problema abordable.

% En general, diremos que podemos hacer uso de las funciones de Green cuando tenemos un problema de la forma
% \begin{equation}
%     \mathcal{L}\Psi(\vec{x}) = f(\vec{x}) \ ,
% \end{equation}
% donde $f(\vec{x})$ corresponde a una función \emph{fuente} y $\mathcal{L}$ es un operador diferencial cualquiera. 
\begin{defi} \marginnote{Función de Green}
    Dado un operador diferencial $\mathcal{L}$ cualquiera, y una función \emph{fuente} $f(\x)$, tales que
    \begin{equation}
        \mathcal{L}\Psi(\vec{x}) = f(\vec{x}) \ .
    \end{equation}

    Una \textbf{función de Green} $G(\x, \x')$ para el operador $\mathcal{L}$ es aquella que satisface la ecuación
    \begin{equation}\label{eq:condicion_green}
        \mathcal{L}G(\vec{x}, \vec{x}') = \delta^{(n)}(\vec{x} - \vec{x}') \ ,
    \end{equation}
    donde $\delta^{(n)}$ es la delta de Dirac $n$-dimensional.
\end{defi}

\begin{propiedad} \marginnote{Condición de consistencia}
    \textbf{Condición de consistencia.} Integrando la definición de las funciones de Green \eqref{eq:condicion_green} sobre el volumen $V$ en el que se encuentre definida nuestro operador diferencial $\mathcal{L}$, la función de Green cumplirá que
    \begin{equation} \label{eq:condicion_consistencia}
        \int_V \mathcal{L} G(\x, \x') dV = \int_V \delta^{(n)}(\x - \x') dV = 1 \ , \quad \forall \ \x \in V \ .
    \end{equation}
\end{propiedad}
% En este caso, buscamos una \textbf{función de Green} $G(\vec{x}, \vec{x}')$ que satisfaga la ecuación
% \begin{equation}\label{eq:condicion_green}
%     \mathcal{L}G(\vec{x}, \vec{x}') = \delta^{(n)}(\vec{x} - \vec{x}') \ ,
% \end{equation}
% donde $\delta^{(n)}$ es la delta de Dirac $n$-dimensional.

\section{Motivación: Potencial electrostático}

En presencia de una carga (o distribución de cargas), el potencial electrostático satisface la \emph{ecuación de Poisson}
\begin{equation}
    \nabla^2 \phi = - \frac{\rho(x)}{\varepsilon_0} \ ,
\end{equation}
donde $\rho(x)$ es la distribución de cargas en la región considerada.

A partir de la ley de Coulomb, sabemos que podemos describir el potencial eléctrico de una distribución de cargas como
\begin{equation} \label{eq:potencial_coulomb}
    \phi(\vec{x}) = \frac{1}{4\pi \varepsilon_0} \int_V \frac{\rho(\vec{x}')}{|\vec{x} - \vec{x}'|} dV' \ ,
\end{equation}
donde hemos hecho la elección tradicional de que el potencial de referencia se anula en el infinito.

Observamos que, si definimos nuestra \emph{función de Green} como
\begin{equation}\label{eq:green_laplaciano} \marginnote{Función de Green para el Laplaciano}
    G(\vec{x}, \vec{x}') = - \frac{1}{4\pi} \frac{1}{|\vec{x} - \vec{x}'|} \ ,
\end{equation}
podemos reescribir \eqref{eq:potencial_coulomb} como
\begin{equation}
    \phi(x) = \int_V G(\vec{x}, \vec{x}') \left( - \frac{\rho(\vec{x}')}{\varepsilon_0} \right) dV' \ .
\end{equation}

Esta función de Green debe satisfacer la ecuación \eqref{eq:condicion_green}, que en este caso es dada por
\begin{equation}
    \nabla^2 G(\vec{x}, \vec{x}') = \delta^{(3)} (\vec{x} - \vec{x}') \ .
\end{equation}

De esta forma, observamos que
\begin{align}
    \nabla^2 \phi(\vec{x}) & = \nabla^2 \int_V G(\vec{x}, \vec{x}') \left( - \frac{\rho(\vec{x}')}{\varepsilon_0} \right) dV' \\
    & = \int_V [\nabla^2 G(\vec{x}, \vec{x}')] \left( - \frac{\rho(\vec{x}')}{\varepsilon_0} \right) dV' \\
    & = \int_V \delta^{(3)}(\vec{x} - \vec{x}') \left( - \frac{\rho(\vec{x}')}{\varepsilon_0} \right) dV'\\
    & = - \frac{\rho(\vec{x})}{\varepsilon_0} \ ,
\end{align}
recuperando la ecuación original.

\section{Encontrando soluciones mediante funciones de Green}

Consideremos una EDP \emph{lineal e inhomogénea} de la forma
\begin{equation}\label{eq:edp_general}
    \mathcal{L} \psi(\vec{x}) = \left(\nabla \cdot (p(\vec{x}) \nabla) + q(\vec{x})\right)\psi(\vec{x}) = f(\vec{x}) \ ,
\end{equation}
donde $p(\vec{x})$ y $q(\vec{x})$ son funciones conocidas.

Para resolver la EDP, encontramos primero la función de Green que satisface la expresión \eqref{eq:condicion_green}, junto a una solución \emph{particular} de la ecuación \eqref{eq:edp_general} en el punto $\vec{x}'$, podemos hallar una solución general del problema, la que será dada por
\begin{equation}\label{eq:solucion_green}
    \psi(\vec{x}) = \oint_{\partial V} p(\vec{x})[\psi(\vec{x}') \nabla' G(\vec{x}, \vec{x}') - G(\vec{x}, \vec{x}') \nabla' \psi(\vec{x}')] \cdot d\vec{S}' + \int_V G(\vec{x}, \vec{x}') f(\vec{x}') dV' \ ,
\end{equation}
donde $\nabla'$ indica que el operador actúa sobre las componentes de $\vec{x}'$. Aquí, $V$ denota al dominio en que la solución $\psi(\vec{x})$ es válida, y $\partial V$ es su frontera.

¿Cómo sabemos que esta ecuación es válida? Resolvamos la primera integral. Notemos que podemos utilizar el teorema de Gauss, de modo que
\begin{align}
    I & = \oint_{\partial V} p(\vec{x})[\psi(\vec{x}') \nabla' G(\vec{x}, \vec{x}') - G(\vec{x}, \vec{x}') \nabla' \psi(\vec{x}')] \cdot d\vec{S}' \label{eq:integral_green_superficie} \\
    & = \int_V \nabla' \cdot \left[ p(\vec{x}') \psi(\vec{x}') \nabla' G(\vec{x}, \vec{x}') - p(\vec{x}') G(\vec{x}, \vec{x}') \nabla' \psi(\vec{x}') \right] dV' \\
    & = \int_V \left[\psi(\vec{x}') \nabla' \cdot \left[ p(\vec{x}')  \nabla' G(\vec{x}, \vec{x}')\right] + \textcolor{blue}{\nabla \psi(\vec{x}') \cdot (p(\vec{x}') \nabla' G(\vec{x}, \vec{x}') )} \right. \nonumber \\
    & \qquad \quad \left. -  G(\vec{x}, \vec{x}') \nabla' \cdot \left[p(\vec{x}') \nabla' \psi(\vec{x}')\right] \textcolor{blue}{- \nabla \psi(\vec{x}') \cdot (p(\vec{x}') \nabla' G(\vec{x}, \vec{x}') )} \right] dV' \\
    & = \int_V \left[ \psi(\vec{x}') \left( \mathcal{L}' G(\vec{x}, \vec{x}') \right) - G(\vec{x}, \vec{x}') \left( \mathcal{L}' \psi(\vec{x}') \right) \right] dV' \\
    & = \int_V \left[ \psi(\vec{x}') \delta(\vec{x} - \vec{x}') - G(\vec{x}, \vec{x}') f(\vec{x}') \right] dV' \\
    & = \psi(\vec{x}) - \int_V G(\vec{x}, \vec{x}') f(\vec{x}') dV' \ ,
\end{align}
donde hemos definido un nuevo operador diferencial $\mathcal{L}'$ que actúa sobre las componentes primadas, tal que $\mathcal{L}' \psi(\vec{x}') = f(\vec{x}')$.

¿Cómo hallamos una solución en el punto $\vec{x}'$? Haciendo uso de las condiciones de borde del problema, que usualmente serán definidas en el contorno $\partial V$.

\subsection{Condiciones de borde de tipo Dirichlet}

Si las condiciones de borde son \emph{de tipo Dirichlet}, sabemos el valor de $\psi(\vec{x}')$ en el borde $\partial V$ de la región en que nos encontramos, por lo que podremos determinar el primer término de la integral sobre $\partial V$ en \eqref{eq:integral_green_superficie}, pero el segundo, que depende de la derivada de $\psi(\vec{x}')$, no estará determinado. Por ello, escogeremos \emph{de forma conveniente} una función de Green que se anule en los bordes, es decir,
\begin{equation}\label{eq:condicion_dirichlet_green}
    G(\vec{x}, \vec{x}') = 0 \ , \qquad \forall \ \vec{x}' \in \ \partial V \ .
\end{equation}

En este caso, la solución general tendrá la forma
\begin{equation}\label{eq:solucion_dirichlet}
    \psi(\vec{x}) = \int_V G(\vec{x}, \vec{x}') f(\vec{x}') dV + \oint_{\partial V} p(\vec{x}') \psi(\vec{x}') \frac{\partial G(\vec{x}, \vec{x'})}{\partial n'} dS' \ ,
\end{equation}
donde $\partial f/\partial n' = (\nabla' f) \cdot \hat{n}$ , con 
$\hat{n}$ un vector normal a la superficie.

Sin embargo, muchas veces puede ser difícil hallar directamente una función de Green que satisfaga a la vez \eqref{eq:condicion_dirichlet_green} y \eqref{eq:condicion_green}. Un ejemplo sencillo de esto es el operador laplaciano, cuando $\mathcal{L} = \nabla^2$. En este caso, es útil buscar una solución de la forma
\begin{equation}\label{eq:green_fundamental}
    G(\x, \x') = F(\x, \x') + H(\x, \x') \ ,
\end{equation}
donde $F(\x, \x')$ satisface la ecuación \eqref{eq:condicion_green} pero no necesariamente la condición de borde \eqref{eq:condicion_dirichlet_green}, mientras que $H(\x, \x')$ satisface la ecuación \emph{homogénea del problema} (es decir, donde no hay funciones fuente) en el interior de la región $V$, pero su valor en $\partial V$ es tal que, sumada a $F(\x, \x')$, resulta en que $G(\x,\x') = 0$ en el borde $\partial V$. Cuando seguimos este procedimiento, la función $F(\x,\x')$ es denominada \textbf{solución fundamental}. En efecto,
\begin{equation}
    \nabla^2 G(\x, \x') = \nabla^2 F(\x, \x') + \nabla^2 H(\x, \x') = \nabla^2 F(\x, \x') + 0 = \delta^{(3)}(\x - \x') \ .
\end{equation}

\begin{ejemplo}
    \textbf{(Riley, Sección 21.5.3)} Encuentre la solución fundamental a la ecuación de Poisson en tres dimensiones, sujeta a la condición de contorno $F(\x , \x') \to 0$ cuando $|\x| \to \infty$.

    \textbf{Solución.}  Dado que nuestro contorno se encuentra en infinito, el problema posee simetría esférica, y puede ser modelado como una esfera $S$ de radio $R$ \emph{muy grande} centrada en $\x'$. La condición de consistencia \eqref{eq:condicion_consistencia} establece
    \begin{equation*}
        \int_V \nabla^2 F(\x, \x') dV = \int_V \delta^{(3)}(\x-\x') dV = 1 \ ,
    \end{equation*}
    que a su vez, usando el teorema de Gauss, puede escribirse como
    \begin{equation}
        \int_S \nabla F(\x, \x') \cdot d\vec{S}' = 1 \ .
    \end{equation}

    Dada la simetría esférica del problema, podemos suponer que nuestra función dependa únicamente de la distancia entre el centro y el punto $\x$, de modo que $F(\x,\x') = F(|\x-\x') = F(r)$. Evaluando la integral de superficie, tenemos que
    \begin{equation*}
        4\pi r^2 \left. \frac{dF}{dr}\right|_{r=R} = 1 \ .
    \end{equation*}

    Integrando sobre $r$, podemos encontrar una expresión para $F$, de modo que
    \begin{equation*}
        F(r) = - \frac{1}{4\pi r} + C \ ,
    \end{equation*}
    que sumado a la condición de contorno del problema, deducimos que $C = 0$. Luego, la solución fundamental en tres dimensiones es dada por
    \begin{equation*}
        F(\x, \x') = - \frac{1}{4\pi |\x-\x'|} \ .
    \end{equation*}
\end{ejemplo}

\subsection{Condiciones de borde de tipo Neumann}


Si las condiciones de borde son \textbf{de tipo Neumann}, conocemos el valor de la derivada $\frac{\partial \psi}{\partial n}(\vec{x}')$ en el borde $\partial V$ de la región en que nos encontramos, por lo que podemos determinar el segundo término de la integral \eqref{eq:integral_green_superficie}, mas no así el primero. 

Por analogía al caso de las condiciones de Dirichlet, podríamos pensar en escoger una función de Green que satisfaga
\begin{equation} \label{eq:condicion_neumann_green}
    \frac{\partial G(\vec{x}, \vec{x}')}{\partial n'} = 0 \ , \qquad \forall \ \vec{x}' \in \ \partial V \ ,
\end{equation}
con lo que la solución general tendría la forma
% En este caso, la solución general tendrá la forma
\begin{equation}\label{eq:solucion_neumann}
    \psi(\x) = \int_V G(\x, \x') f(\x') dV - \oint_{\partial V} p(\x ') G(\x, \x') \frac{\partial \psi(\x')}{\partial n'} dS' \ .
\end{equation}

Lamentablemente, en general \emph{no es posible} encontrar una función de Green que satisfaga la condición \eqref{eq:condicion_neumann_green}, siendo nuevamente un ejemplo de esto el operador laplaciano $\nabla^2$, ya que las funciones de Green deben satisfacer la condición de consistencia \eqref{eq:condicion_consistencia}. Por ello, ya que descartamos la solución más sencilla posible ($\frac{\partial G(\vec{x}, \vec{x}')}{\partial n'} = 0$), escogemos la segunda más sencilla, igualar la derivada a una constante,
\begin{equation}
    \frac{\partial G(\vec{x}, \vec{x}')}{\partial n'} = C \ , \qquad \forall \vec{x}' \in \ \partial V \ .
\end{equation}
Para hallar el valor de esta constante, hacemos uso de la \emph{condición de consistencia}, de modo que el caso más sencillo será suponer que $C = 1/\text{Área}(\partial V)$.

De ser posible esta elección, la solución tendrá la forma
\begin{equation}\label{eq:solucion_neumann}
    \psi(\x) = \int_V G(\x, \x') f(\x') dV - \oint_{\partial V} p(\x ') G(\x, \x') \frac{\partial \psi(\x')}{\partial n'} dS' + \underbrace{\frac{1}{A} \oint_{\partial V} p(\x) \psi(\x') \ dS'}_{\langle \psi(\x') \rangle_{\partial V}} \ .
\end{equation}
Aquí, la notación $\langle \psi(\x') \rangle_{\partial V}$ representa una especie de \emph{valor promedio ponderado} sobre el contorno $\partial V$. Este promedio actuará como una constante que puede ser determinada libremente. Es más, si el contorno $\partial V$ de la región es el infinito, este término puede ser fijado a cero, pues no requeriremos de este término para que la solución tenga sentido físico.

% \begin{ejemplo}
%     \textbf{(Riley, Sección 21.5.4)} Resuelva la ecuación de Laplace en la región bidimensional $|\x| \leq a$ sujeta a la condición de contorno $\partial \Psi/\partial n = f(\phi)$ en $|\x| = a$. Considere que la condición de consistencia implica que $\int_0^{2\pi} f(\phi) d\phi = 0$.

%     \textbf{Solución.} 
% \end{ejemplo}

\begin{obs}{Observación}
    Es importante recordar que una misma EDP puede dar origen a diferentes funciones de Green, pues estas dependen también de las condiciones de contorno.
\end{obs}


\section{Simetría de las funciones de Green}

Cuando tenemos el caso particular en que $\mathcal{L} = \nabla^2$, la función de Green \eqref{eq:green_laplaciano} es simétrica bajo el intercambio de argumentos, es decir, $G(\x, \x') = G(\x', \x)$. Podemos buscar cuál es la condición general que permite que esto ocurra, escogiendo $\psi(\x) = \psi(\x'', \x)$, de modo que, según la expresión \eqref{eq:solucion_green},
\begin{align}
    \nonumber G(\x'', \x) & = \int_V G(\x, \x') \delta(\x'' - \x') dV' \\ 
    & \qquad + \oint_{\partial V} p(\x')\left[ G(\x'', \x') \nabla' G(\x, \x') - G(\x, \x') \nabla'G(\x'', \x') \right] \cdot d\vec{S}' \\
    & = G(\x, \x'') + \oint_{\partial V} p(\x')\left[ G(\x'', \x') \nabla' G(\x, \x') - G(\x, \x') \nabla'G(\x'', \x') \right] \cdot d\vec{S}' \ ,
\end{align}
con lo que vemos que la función de Green será simétrica siempre y cuando la integral de superficie se anule en la frontera, 
\begin{equation}\label{eq:condicion_simetria}
    \oint_{\partial V} p(\x')\left[ G(\x'', \x') \nabla' G(\x, \x') - G(\x, \x') \nabla'G(\x'', \x') \right] \cdot d\vec{S}' = 0 \ .
\end{equation}
% Esto ocurre, típicamente, para condiciones de tipo Dirichlet.
A continuación, listamos algunas situaciones en que una función de Green será simétrica.

\begin{propiedad}
    \textbf{Simetría de las funciones de Green.} Una función de Green para un operador diferencial $\mathcal{L}$ será simétrica, esto es, $G(\x, \x') = G(\x', \x)$, si 
    \begin{enumerate}
        \item El problema posee condiciones de tipo Dirichlet homogéneas.
        \item El problema posee condiciones de tipo Neumann homogéneas.
        \item El operador diferencial es \emph{hermítico}, esto es, $\mathcal{L} = \mathcal{L}^\dagger$.
        \item El operador diferencial es \emph{real}, ya que no poseerá parte imaginaria por conjugar.
        \item Cualquier otra condición que satisfaga la expresión \eqref{eq:condicion_simetria}.
    \end{enumerate}
\end{propiedad}

\section{Método de las imágenes}

Anteriormente mencionamos que podemos hallar la función de Green para un problema con condiciones de Dirichlet hallando la solución fundamental y una solución que produzca que $G(\x,\x') = 0$ en $\partial V$, como establecimos en \eqref{eq:green_fundamental}. Para ello, podemos hacer uso del \textbf{método de las imágenes}, en el cual hemos de considerar \emph{copias} de nuestra solución fundamental que representen \emph{fuentes} ubicadas en el exterior de nuestra región $V$. Este método es de gran utilidad en problemas de electrostática que presenten geometrías altamente simétricas, y será aplicado en este contexto en el curso de Electrodinámica I.
\begin{propo} \marginnote{Método de las imágenes}
    \textbf{Método de las imágenes.} Es posible hallar una función de Green para una ecuación inhomogénea al sumar a la solución fundamental $F(\x, \x')$ diferentes copias (o \emph{imágenes}) de la ella misma, ubicadas \emph{fuera} de la región $V$ en la que deseamos resolver la ecuación. Para ello, seguiremos el siguiente procedimiento,
    \begin{enumerate}
        \item Para una fuente singular $\delta(\x - \x')$ dentro de la región $V$, añadiremos fuentes imágenes \emph{fuera} de $V$, donde las posiciones $\x_n$ e intensidades $q_n$ de estas fuentes las podremos determinar a futuro, de modo que las fuentes externas se pueden expresar como
        \begin{equation}
            \sum_{n=1}^N q_n \delta(\x - \x_n) \ , \qquad \x_n \notin \ V \ .
        \end{equation}
    
        \item Ya que todas las imágenes se encuentran en el exterior de $V$, la solución fundamental que corresponde a cada una de las fuentes deberá satisfacer la ecuación de Laplace \emph{dentro} de $V$. Por ello, podemos suponer que cada imagen tiene asociada la misma \emph{solución fundamental} que la fuente en el interior de $V$, con lo que la función de Green tomará la forma
        \begin{equation}
            G(\x, \x') = F(\x, \x') + \sum_{n=1}^N q_n F(\x,\x_n) \ .
        \end{equation}
    
        \item Ajustamos las posiciones $\x_n$ e intensidades $q_n$ de las imágenes de modo que las condiciones de contorno se satisfagan en $S$. En el caso de las condiciones de Dirichlet, esto es exigir que $G(\x, \x') = 0$, para cualquier $\x' \in \partial V$.
    
        \item Por último, podemos usar la función de Green hallada para encontrar la solución, sujeta a las condiciones de Dirichlet, que satisfaga \eqref{eq:solucion_dirichlet}.
    \end{enumerate}
\end{propo}

En general, no es tarea sencilla encontrar las posiciones e intensidades correctas para cualquier problema, pero sí lo es para ciertos problemas con geometrías simples. Particularmente, este método es útil para problemas en los que los bordes sean \emph{líneas rectas} (problema bidimensional) o \emph{planos} (problema tridimensional), pues en esos casos simplemente supondremos que estos actúan como espejos, de modo que la fuente \emph{verdadera} se refleja simétricamente en este espejo.

\begin{ejemplo}
    \textbf{(Riley, Sección 21.5.3)} Resuelva la ecuación de Laplace en la región bidimensional $|\x| \leq a$ sujeta a la condición de contorno $u = h(\phi)$ en $|\x| = a$, dada una carga fuente ubicada en $\x_0$.

    \textbf{Solución.} Nuestro problema corresponde a un disco de radio $a$ sujeto a condiciones de Dirichlet. Como queremos que nuestra función de Green satisfaga que $G(\x, \x_0) = 0$ para $|\x| = a$, podemos suponer que nuestra \emph{carga imagen} se encuentra fuera del disco en una posición $\x_1 = (a^2/|\x_0|^2)\x_0$, de modo que, suponiendo que la intensidad de la carga imagen es $-1$, nuestra función de Green toma la forma (véase la siguiente sección para la deducción detallada de la función de Green para el Laplaciano en 2D)
    \begin{equation*}
        G(\x, \x_0) = \frac{1}{2\pi} \ln(|\x - \x_0|) - \frac{1}{2\pi} \ln(|\x - \x_1|) + C \ .
    \end{equation*}

    Del hecho que $G(\x, \x_0) = 0$ para $|\x| = a$, tenemos que
    \begin{align*}
        2\pi C & = \ln(|\x - \x_1|) - \ln(|\x - \x_0|) \\
        & = \ln\left( \frac{|\x - \x_1|}{|\x - \x_0|} \right) \\
        & = \ln \left( \frac{\sqrt{a^4/x_0^2 - 2 a^3 \cos(\phi - \phi_0)/x_0 + a^2}}{|\x - \x_0|} \right) \\
        & = \ln \left( \frac{\sqrt{(a^2/x_0^2)}\cancel{|\x - \x_0|}}{\cancel{|\x - \x_0|}} \right) \\
        & = - \ln \left( \frac{x_0}{a} \right)
        % \\
        % & = \ln\left( \sqrt{\frac{a^2}{x_0^2}} \right) \ ,
    \end{align*}
    donde $x_0 = |\x_0|$. Así,
    \begin{equation*}
        C = - \frac{1}{2\pi} \ln \frac{x_0}{a} \ ,
    \end{equation*}
    y nuestra función de Green toma la forma 
    \begin{equation*}
        G(\x, \x_0) = \frac{1}{2\pi} \left[ \ln|\x - \x_0| - \ln\left| \x - \frac{a^2}{x_0^2} \x_0 \right| - \ln \frac{x_0}{a} \right] \ .
    \end{equation*}

    Conociendo nuestra función de Green, podemos utilizar la ecuación \eqref{eq:solucion_dirichlet} para hallar una solución a nuestro problema. Como trabajamos con la ecuación de Laplace, $f(\x_0) = 0$, de modo que solo sobrevive la integral sobre el contorno de nuestra región en \eqref{eq:solucion_dirichlet}, con lo cual
    \begin{equation*}
        \psi(\x) = \oint_C \psi(\x') \frac{\partial G(\x, \x')}{\partial n'} \ d\ell' = \int_0^{2\pi} \psi(\x') \left. \frac{\partial G}{\partial \rho'}\right|_{\rho' = a} a \ d\phi' \ .
    \end{equation*}

    Para nuestra función de Green,
    \begin{align*}
        \frac{\partial G}{\partial \rho'} & = \hat{x}' \cdot \nabla' G(\x,\x') \\
        & = \hat{x}' \cdot \nabla' G(\x',\x) \\
        & = \frac{\x'}{2\pi x'} \cdot \left( \frac{\x' - \x}{|\x' - \x|^2} - \frac{\x' - (a^2/x^2)\x}{|\x' - (a^2/x^2)\x|^2} \right) \ .
    \end{align*}
    que en $\rho' = a$ satisface que
    \begin{align*}
        \left. \frac{\partial G}{\partial \rho'} \right|_{\rho' = a} & = \frac{\x'}{2\pi a} \cdot \left( \frac{\x' - \x}{|\x' - \x|^2} - \frac{\x' - (a^2/x^2)\x}{(a^2/x^2)|\x' - \x|^2} \right) \\
        & = \frac{\x'}{2\pi a} \cdot \frac{(a^2/x^2)\x' - (a^2/x^2) \x - \x' + (a^2/x^2)\x }{(a^2/x^2)|\x' - \x|^2} \\
        & = \frac{1}{2\pi a} \frac{a^2 - x^2}{|\x' - \x|^2} \\
        & = \frac{1}{2\pi a} \frac{a^2 - \rho^2}{a^2 + \rho^2 - 2a\rho \cos(\phi' - \phi)} \ .
    \end{align*}
    
    De esta manera, concluimos que una solución a nuestro problema puede ser hallada como
    \begin{equation*}
        u(\rho, \phi) = \frac{1}{2\pi} \int_0^{2\pi}  \frac{(a^2 - \rho^2) f(\phi') \ d\phi'}{a^2 + \rho^2 - 2a\rho \cos(\phi' - \phi)} \ .
    \end{equation*}
\end{ejemplo}

También es útil mencionar que es válido utilizar el método de las imágenes al trabajar con condiciones de Neumann, donde el procedimiento descrito anteriormente sigue siendo el mismo, pero utilizando las condiciones de Neumann para establecer nuestras imágenes.

\begin{ejemplo}
    \textbf{(Riley, Sección 21.5.4)} Resuelva la ecuación de Laplace en la región bidimensional $|\x| \leq a$ sujeta a la condición de contorno $\partial u/\partial n = f(\phi)$ en $|\x| = a$, con $\int_0^{2\pi} f(\phi) d\phi = 0$, como requiere la condición de consistencia.

    \textbf{Solución.} La geometría del problema es la misma que la del ejemplo anterior. Por ello, podemos suponer que la carga imagen se encuentra en la misma posición $\x_1 = (a^2/x_0^2) \x_0$, donde nuevamente $\x_0$ corresponde a la carga fuente dentro del círculo. Sin embargo, ahora no asumiremos la intensidad de la carga imagen, sino que la dejaremos como un parámetro $q$. De esta forma, la función de Green será de la forma
    \begin{equation*}
        G(\x, \x_0) = \frac{1}{2\pi} \left( \ln |\x - \x_0| + q \ln |\x - \x_1| + C \right) \ .
    \end{equation*}

    En este caso, la derivada radial (normal) de esta función es dada por
    \begin{equation*}
        \frac{\partial G}{\partial \rho} = \frac{\x}{|\x|} \cdot \nabla G(\x, \x_0) = \frac{\x}{2\pi |\x|} \cdot \left[ \frac{\x - \x_0}{|\x - \x_0|^2} + q \frac{\x - \x_1}{|\x - \x_1|^2} \right] \ ,
    \end{equation*}
    que evaluada en $\rho = a$, tenemos que
    \begin{equation*}
        \left. \frac{\partial G}{\partial \rho} \right|_{\rho = a} = \frac{1}{2\pi a} \frac{1}{|\x - \x_0|^2} \left[ |\x|^2 + qx_0^2 - (1+q) \x \cdot \x_0 \right] \ .
    \end{equation*} 

    Como se discutió anteriormente, en problemas con condiciones de tipo Neumann es conveniente escoger que esta derivada se iguale a alguna constante, típicamente el inverso del valor del ``área'' del contorno de nuestra región. Dado que estamos trabajando en una región bidimensional, esta ``área'' corresponde al perímetro del círculo, de modo que
    \begin{equation*}
        \left. \frac{\partial G}{\partial \rho} \right|_{\rho = a} = \frac{1}{2\pi a} \ ,
    \end{equation*}
    lo que se cumple escogiendo $q = 1$. Así, observamos que para un problema con la misma geometría, las condiciones de contorno afectarán al valor de la intensidad $q$ de la carga imagen.

    Podemos encontrar la solución a nuestro problema a partir de la ecuación \eqref{eq:solucion_neumann}, donde al tratarse de la ecuación de Laplace, $f(x) = 0$, con lo que la solución será de la forma
    \begin{equation*}
        \psi(\x) = \langle \psi(\x) \rangle_{C} - \oint_{\partial V} G(\x', \x) f(\x') \ d\ell' 
    \end{equation*}

    Así, la función de Green adecuada, al ser evaluada en $\rho' = a$, tomará el valor
    \begin{align*}
        G(\x',\x)|_{\rho' = a} & = \frac{1}{2\pi} \left[ \ln |\x' - \x| + \ln |\x' - (a^2/x^2)\x| + C \right] \\
        & = \frac{1}{2\pi} \left[ \ln\left( |\x' - \x| |\x - (a^2/x^2)\x| \right) + C \right] \\
        & = \frac{1}{2\pi} \left[ \ln\left( |\x' - \x|^2 \right) + \ln\frac{a}{x} C \right] \\
        & = \frac{1}{2\pi} \left[ \ln(a^2 + \rho^2 - 2 a \rho \cos(\phi' - \phi)) + \ln \frac{a}{\rho} + C \right] \ ,
    \end{align*}
    donde la constante $C$, es una constante arbitraria. Entonces,
    \begin{align*}
        \psi(\rho, \phi) & = \langle \psi(\x) \rangle_C - \frac{a}{2\pi} \int_0^{2\pi} f(\phi') \ln[a^2 + \rho^2 - 2a\rho \cos(\phi' - \phi)] \ d\phi' \\
        & \qquad - \frac{a}{2\pi} \left(\ln \frac{a}{\rho} + C \right) \int_0^{2\pi} f(\phi') \ d\phi' \\
        & = \langle \psi(\x) \rangle_C - \frac{a}{2\pi} \int_0^{2\pi} f(\phi') \ln[a^2 + \rho^2 - 2a\rho \cos(\phi' - \phi)] \ d\phi' \ ,
    \end{align*}
    donde hemos usado la condición dada en el enunciado, $\int_0^{2\pi} f(\phi) d\phi = 0$. 

\end{ejemplo}


\section{Algunas funciones de Green comunes}

Como mencionamos anteriormente, las funciones de Green no son únicas, sino que pueden construirse siempre a partir de una solución fundamental $F(\x,\x')$ que también es una función de Green, sumada a una función $H(\x,\x')$ que es solución al problema homogéneo asociado al operador $\mathcal{L}$,
\begin{equation}
    \mathcal{L} H(\x,\x') = 0 \ .
\end{equation} 

Por ello, listamos a continuación las soluciones fundamentales para el operador Laplaciano y para el Operador de Helmholtz, a las que podemos sumar soluciones a la ecuación homogénea de modo que la suma satisfaga las condiciones de borde del problema específico que podamos estar desarrollando.

\subsection{Ecuación de Laplace}

En este caso, nuestro operador diferencial corresponde a $p(x) = 1$ y $q(x) = 0$ en la definición \eqref{eq:edp_general}. Dado que este curso se centra en métodos útiles en problemas físicos, consideraremos siempre la función de Green que respete la \emph{homogeneidad e isotropía del espacio}, es decir, que la función de Green dependa únicamente de la \emph{diferencia} entre $\x$ y $\x'$ (homogeneidad), y en particular, dependa de la \emph{distancia} $|\x - \x'|$ (isotropía, o invariancia bajo rotaciones). Por ello, la función de Green en este caso corresponde a una función de una variable, que podemos llamar $r = |\x - \x'|$.

\subsubsection{Caso tridimensional}

En un sistema de coordenadas esféricas centradas en $\x'$, nuestra función de Green satisface
\begin{equation}
    \nabla^2 G(r) = \frac{1}{r^2} \frac{d}{dr} \left( r^2 \frac{dG}{dr} \right) = \delta^{(3)}(r) \ ,
\end{equation}
de modo que, si $r \neq 0$, 
\begin{equation}
    \frac{1}{r^2} \frac{d}{dr} \left( r^2 \frac{dG}{dr} \right) = 0 \ .
\end{equation}
Integrando esta expresión, hallamos que
\begin{equation}
    G(r) = \alpha + \frac{\beta}{r} \ ,
\end{equation}
donde $\alpha$ y $\beta$ son coeficientes a determinar. Podemos hacerlo utilizando el teorema de Gauss en la condición de consistencia \eqref{eq:condicion_consistencia}, de modo que
\begin{equation}
    \int_{\partial V} \nabla G \cdot d\vec{S} = \int_{\partial V} \frac{dG}{dr} r^2 d\Omega = 1 \ ,
\end{equation}
donde $\Omega$ corresponde a un ángulo sólido. Integrando, tenemos que
\begin{equation}
    \beta = - \frac{1}{4\pi} \ .
\end{equation}

Por otro lado, en la mayoría de los problemas es conveniente escoger una función de Green que se anule a grandes distancias ($\lim_{r\to \infty} G(r) = 0)$. Por ello, es común escoger $\alpha = 0$. Es más, podemos argumentar que una solución constante $H(r) = - \alpha$ corresponde a una solución al problema homogéneo, que podemos sumar a nuestra función de Green anterior, obteniendo así la solución fundamental para la ecuación de Laplace,
\begin{equation} \marginnote{Solución fundamental de la ecuación de Laplace en 3D}
    \boxed{G(\x - \x') = - \frac{1}{4\pi} \frac{1}{|\x - \x'|} \ .}
\end{equation}

\subsubsection{Caso bidimensional}

En un sistema de coordenadas polares centradas en $\vec{x}'$, nuestra función de Green satisface
\begin{equation}
    \nabla^2 G = \frac{1}{\rho} \frac{d}{d\rho} \left( \rho \frac{dG}{d\rho} \right) = \delta^{(2)}(\rho) \ .
\end{equation}
Integrando para $\rho \neq 0$, tenemos que
\begin{equation}
    G(\rho) = \alpha + \beta \ln \rho \ .
\end{equation}
Utilizando el teorema de Gauss (en dos dimensiones) y la condición de consistencia \eqref{eq:condicion_consistencia}, 
\begin{equation}
    \int_{\partial S} \nabla G \cdot d\vec{S} = \oint \frac{dG}{d\rho} \rho d\phi = 2\pi \beta = 1 \ , 
\end{equation}
por lo que nuevamente suponiendo $\alpha = 0$, tenemos que 
\begin{equation} \marginnote{Solución fundamental de la ecuación de Laplace en 2D}
    \boxed{G(|\x - \x'|) = \frac{1}{2\pi} \ln |\x - \x'| \ .}
\end{equation}

% \subsubsection{Caso unidimensional}

% En un sistema unidimensional con origen en $\x'$, nuestra función de Green satisface
% \begin{equation}
%     \nabla^2 G = \frac{d^2G}{dx^2} = \delta(x) \ .
% \end{equation}
% Integrando para $x \neq 0$, tenemos que
% \begin{equation}
%     G(x) = \alpha + \beta x \ ,
% \end{equation}
% de modo que aplicando la condición de consistencia, tenemos que
% \begin{equation}
%     \int \frac{dG}{dx} dx
% \end{equation}

\subsection{Ecuación de Helmholtz}

En este caso, nuestro operador diferencial corresponde a $p(x) = 1$ y $q(x) = k^2$ en \eqref{eq:edp_general}, de modo que $\mathcal{L} = \nabla^2 + k^2$. Nuevamente, supondremos homogenenidad e isotropía, de modo que buscaremos funciones de la forma $G(\x, \x') = G(|\x - \x'|)$.

\subsubsection{Caso tridimensional}

En coordenadas esféricas centradas en $\x'$, nuestra función de Green satisface
\begin{equation}
    (\nabla^2 + k^2) G(r) = \frac{1}{r^2} \frac{d}{dr}\left( r^2 \frac{dG}{dr} \right) + k^2 G(r) = \delta^{(3)}(r) \ ,
\end{equation}
de modo que al considerar $r \neq 0$, obtenemos la ecuación
\begin{equation}
    \frac{1}{r^2} \frac{d}{dr}\left( r^2 \frac{dG}{dr} \right) + k^2 G(r) = 0 \ .
\end{equation}

Bajo el cambio de variable $u(r) = r G(r)$, podemos reescribir la ecuación anterior como
\begin{equation}
    \frac{d^2 u}{dr^2} + k^2 u = 0 \ ,
\end{equation}
que corresponde a la ecuación de un oscilador armónico simple. Luego, las soluciones son de la forma
\begin{equation}\label{eq:solucion_green_Helmholtz}
    G(r) = \alpha \frac{e^{ikr}}{r} + \beta \frac{e^{-ikr}}{r} \ ,
\end{equation}
donde nuevamente, $\alpha$ y $\beta$ son coeficientes a determinar. Mediante el teorema de Gauss aplicado a la condición de consistencia \eqref{eq:condicion_consistencia}, sobre una esfera de rario $R$ centrada en $r=0$, de modo que
\begin{equation}
    \oint_{\partial V}  \nabla G \cdot \ d\vec{S} + k^2 \int_V G \ dV = \oint_{\partial V} \frac{dG}{dr} r^2 \ d\Omega + k^2 \int_V G r^2 \ dr \ d\Omega = 1 \ .
\end{equation}

La primera integral es directa, puesto que el integrando no depende del ángulo sólido $\Omega$. La segunda integral, por otro lado, debe ser resuelta con más cuidado. Tenemos que
\begin{align*}
    1 & = 4\pi R^2 \left. \frac{dG}{dr}\right|_R + 4\pi k^2 \int_0^R G r^2 \ dr \nonumber \\
    & = 4\pi R^2 \left[\frac{1}{R} (\alpha ik e^{ikR} - \beta ik e^{-ikR}) - \frac{1}{R^2} (\alpha e^{ikR} + \beta e^{-ikR}) \right] + 4\pi k^2 \int_0^R \left[ \alpha e^{ikr} + \beta e^{-ikr} \right] r \ dr \nonumber \\
    & = 4\pi \left[ \alpha (ikR-1)e^{ikR} - \beta (1+ikR)e^{-ikR} + \alpha (1-ikR)e^{ikR} + \beta (1+ikR) e^{-ikR} - (\alpha + \beta) \right] \nonumber \\
    & = -4\pi (\alpha + \beta) \ ,
\end{align*}
de modo que nuestros coeficientes deben satisfacer
\begin{equation}
    \alpha + \beta = - \frac{1}{4\pi} \ .
\end{equation}
Luego, podemos escribir la solución \eqref{eq:solucion_green_Helmholtz} en términos del coeficiente $\beta$ como
\begin{equation}
    G(r) = - \frac{1}{4\pi} \frac{e^{ikr}}{r} - 2i\beta \frac{\sin(kr)}{r} \ ,
\end{equation}
donde el segundo término corresponde a una solución de la ecuación de Helmholtz homogénea, pues es proporcional a la función esférica de Bessel $j_0(kr)$. Luego, considerando $\beta = 0$ por este motivo, es posible escoger la solución fundamental de la ecuación de Helmholtz como
\begin{equation} \marginnote{Solución fundamental de la ecuación de Helmholtz en 3D}
    \boxed{G(\x - \x') = - \frac{1}{4\pi} \frac{e^{ik|\x - \x'|}}{|\x - \x'|} \ .}
\end{equation}

\begin{obs}{Observación}
    Una elección igual de válida habría sido escribir la ecuación \eqref{eq:solucion_green_Helmholtz} en términos de $\alpha$, donde de igual forma obtendríamos un término proporcional a la función esférica de Bessel $j_0(kr)$. Luego, escogiendo $\alpha = 0$, obtendríamos la solución
    \begin{equation}
        \boxed{G_2(\x - \x') = - \frac{1}{4\pi} \frac{e^{-ik|\x - \x'|}}{|\x - \x'|} \ .}
    \end{equation}
    
    Dependerá de la fuente que estén consultando cuál de las dos soluciones se prefiere.
\end{obs}

\subsubsection{Caso bidimensional}

En un sistema de coordenadas polares centrado en $\x'$, nuestra función de Green satisface
\begin{equation}
    (\nabla^2 + k^2)G(\rho) = \frac{1}{\rho} \frac{d}{d\rho} \left( \rho \frac{dG}{d\rho} \right) + k^2 G(\rho) = \delta^{(2)}(\rho) \ ,
\end{equation}
que para $\rho \neq 0$ corresponde a la ecuación
\begin{equation}
    \frac{1}{\rho} \frac{d}{d\rho} \left( \rho \frac{dG}{d\rho} \right) + k^2 G = 0 \ .
\end{equation}

Bajo el cambio de variable $x = k\rho$, esta se reduce a la ecuación de Bessel de orden 0,
\begin{equation}
    x \frac{d}{dx}\left( x \frac{dG}{dx} \right)  + x^2 G(x) = 0 \ ,
\end{equation}
con soluciones de la forma
\begin{equation}\label{eq:solucion_green_Helmholtz_2d}
    G(\rho) = \alpha J_0(k\rho) + \beta Y_0(k\rho) = \tilde{\alpha} H_0^{(1)}(k\rho) + \tilde{\beta} H_0^{(2)}(k\rho) \ .
\end{equation}
Preferiremos utilizar la solución en términos de funciones de Hankel, pues es la elección habitual.

Integrando la ecuación original sobre un círculo $S$ de radio $R$ centrado en $\rho = 0$, obtenemos
\begin{equation}
    \oint_{\partial S} \nabla G \cdot d\vec{S} + k^2 \int_S G \ dS = \oint \left. \frac{dG}{d\rho}\right|_R R \ d\phi + 2\pi \int_0^R G \rho \ d\rho = 1 \ .
\end{equation}

Nuevamente, la primera integral se calcula directamente, ya que no existe dependencia de $\phi$, mientras que la segunda necesita un tratamiento más delicado. Recordemos que las funciones de Hankel, al ser funciones de Bessel, satisfacen las identidades $H_0^{\prime \, (1)}(x) = - H_1^{(1)}(x)$ y $x H_0^{(1)}(x) = [xH_1^{(1)}(x)]'$. Entonces, tenemos que
\begin{align*}
    1 & = 2\pi \left[ kR (\tilde{\alpha} H_0^{\prime \, (1)}(kR) + \tilde{\beta} H_0^{\prime \, (2)}(kR))  + \int_0^{kR} \left(\tilde{\alpha} H_0^{(1)}(x) + \tilde{\beta} H_0^{(2)}(x) \right) x \ dx \right] \\
    & = 2\pi \left[ - kR (\tilde{\alpha} H_1^{(1)}(kR) + \tilde{\beta} H_1^{(2)}(kR)) + \int_0^{kR} \left(\tilde{\alpha} \frac{d}{dx} \left( x H_1^{(1)}(x) \right) + \tilde{\beta} \frac{d}{dx} \left( x H_1^{(2)}(x) \right) \right) \ dx \right] \\
    & = 2\pi \left[ - kR (\tilde{\alpha} H_1^{(1)}(kR) + \tilde{\beta} H_1^{(2)}(kR)) + kR \left(\tilde{\alpha} H_1^{(1)}(kR) + \tilde{\beta} H_1^{(2)}(kR) \right) \right. \\
    & \left. \qquad - \lim_{x \to 0} \left( \tilde{\alpha} x H_1^{(1)}(x) - \tilde{\beta} x H_1^{(2)}(x) \right) \right] \\
    & = 2\pi \left[ \tilde{\alpha} \frac{2i}{\pi} - \tilde{\beta} \frac{2i}{\pi} \right] \\
    & = 4i (\tilde{\alpha} - \tilde{\beta}) \ ,
\end{align*}
de modo que nuestros coeficientes deberán satisfacer la condición
\begin{equation}
    \tilde{\alpha} - \tilde{\beta} = - \frac{i}{4} \ .
\end{equation}
Luego, la solución \eqref{eq:solucion_green_Helmholtz_2d} se puede escribir, en términos de $\tilde{\beta}$, como
\begin{equation}
    - \frac{i}{4} H_0^{(1)}(k\rho) + \tilde{\beta} \left( H_0^{(1)}(k\rho) + H_0^{(2)}(k\rho) \right) = - \frac{i}{4} H_0^{(1)}(k\rho) + 2 \tilde{\beta} J_0(k\rho) \ .
\end{equation}
Al igual que en el caso tridimensional, la solución proporcional a la función de Bessel de orden 0 corresponde a una solución del problema homogéneo, de modo que si $\tilde{\beta = 0}$, la solución fundamental de la ecuación de Helmholtz en 2D corresponde a
\begin{equation} \marginnote{Solución fundamental de la ecuación de Helmholtz en 2D}
    \boxed{G(\x - \x') = - \frac{i}{4} H_0^{(1)}(k|\x - \x'|) \ .}
\end{equation}
\begin{obs}{Observación}
    Al igual que en el caso anterior, la elección de $\tilde{\alpha} = 0$ conduce a una segunda solución posible, la que corresponde a 
    \begin{equation}
        \boxed{G_2(\x - \x') = \frac{i}{4} H_0^{(2)}(k|\x - \x'|) \ .}
    \end{equation}
\end{obs}
\chapter{Introducción a los Tensores Cartesianos}

Llegando al último capítulo del curso, nos desviamos un poco de la noción de que este curso se dedica a enseñar métodos matemáticos que pueden ser útiles en Física, para en su lugar introducir cantidades y conceptos con la misma utilidad.

Una de las nociones más importantes que tenemos en física clásica es el hecho de que \emph{los fenómenos físicos son los mismos, y no deben cambiar según el observador}, más allá de que las componentes de las cantidades que los describen puedan hacerlo. Particularmente, nos centraremos en las \emph{transformaciones ortogonales} de un sistema coordenado, referidas de forma más común como \textbf{rotaciones}. 

Por ejemplo, un vector que describe la posición de un objeto en función del tiempo puede ser diferente según el sistema de coordenadas en que se lo describa, pero el movimiento \emph{físico} seguirá siendo el mismo.

\section{Transformaciones ortogonales}

Antes de entrar más de lleno en la discusión, recordemos e introduzcamos algunas definiciones.

\begin{defi}
    Se denomina \textbf{delta de Kronocker} al elemento $\delta_{ij}$, definido en un espacio vectorial de $n$ dimensiones como
    \begin{equation}
        \delta_{ij} = \begin{dcases}
            1, \qquad i = j \\
            0, \qquad i \neq j
        \end{dcases} \ .
    \end{equation} 

    Este elemento puede representarse de forma matricial como la matriz identidad del espacio de dimensión $n$.
\end{defi}

\begin{defi}
    Sea un conjunto de vectores unitarios $\{ \hat{e}_i\}_{i=1}^n$ de un espacio $n$-dimensional. Diremos que este forma una \textbf{base ortonormal} si al realizar el producto escalar entre elementos del conjunto, se cumple la relación
    \begin{equation}
        \hat{e}_i\cdot \hat{e}_j = \delta_{ij} \ ,
    \end{equation}
    donde $\delta_{ij}$ es la delta de Kronecker.
\end{defi}

\begin{defi}
    Dado un sistema coordenado en un espacio de $n$ dimensiones, podemos definir el \textbf{vector posición} $\vec{x}$, que une el origen del sistema con punto con coordenadas $x_i$, con $i = 1, 2, \dots, n$ como
\begin{equation} \label{eq:vector-posicion}
    \vec{x} = \sum_{i=1}^n x_i \hat{e}_i \ ,
\end{equation}
donde las \textbf{componentes del vector} en la base $\{ \hat{e}_i\}_{i=1}^n$ puede expresarse como
\begin{equation}
    x_i = \vec{x} \cdot \hat{e}_i \ .
\end{equation}
\end{defi}

Más allá de que el vector posición es aquel que tiene un sentido \emph{físico} a partir del cual hacer las definiciones, podemos descomponer \emph{cualquier vector} en la base $\{ \hat{e}_i\}_{i=1}^n$ en términos de sus respectivas componentes, sin importar la cantidad que este pueda representar.

Una vez introducidas estas nociones, podemos definir un nuevo sistema coordenado, que llamaremos $x'_i$ y cuya base es $\{ \hat{e}_i'\}_{i=1}^n$, que corresponde a una \emph{rotación} del sistema $x_i$ definido anteriormente, como se ve en la figura X.

Respecto de esta nueva base, un vector $\vec{v}$ cualquiera puede ser descompuesto en sus componentes $v_i'$ \emph{en la base} $\{ \hat{e}_i'\}_{i=1}^n$,
\begin{equation}
    \vec{v} = \sum_{i=1}^n v_i' \hat{e}'_i \ .
\end{equation}

Dado que, si bien dan origen a sistemas de coordenadas diferentes, ambas bases se encuentran en el mismo espacio vectorial, ¿Cómo podemos relacionar ambas bases entre sí? Para ello, haremos uso de una \textbf{matriz de transformación}, definida como
\begin{equation}
    A = \begin{pmatrix}
        a_{11} & a_{12} & \dots & a_{1n} \\
        a_{21} & a_{22} & \dots & a_{2n} \\
        \vdots & \vdots & \ddots & \vdots \\
        a_{n1} & a_{n2} & \dots & a_{nn}
    \end{pmatrix} =
    \begin{pmatrix}
        \hat{e}_1 \cdot \hat{e}'_1 & \hat{e}_1 \cdot \hat{e}'_2 & \dots & \hat{e}_1 \cdot \hat{e}'_n \\
        \hat{e}_2 \cdot \hat{e}'_1 & \hat{e}_2 \cdot \hat{e}'_2 & \dots & \hat{e}_2 \cdot \hat{e}'_n \\
        \vdots & \vdots & \ddots & \vdots \\
        \hat{e}_n \cdot \hat{e}'_1 & \hat{e}_n \cdot \hat{e}'_2 & \dots & \hat{e}_n \cdot \hat{e}'_n
    \end{pmatrix} \ .
\end{equation}

De este modo, 
\begin{equation} \label{eq:transformacion-coordenadas}
    \hat{e}'_i = \sum_{j=1}^n a_{ij} \hat{e}_j \ .
\end{equation}
Sin embargo, esta definición describe una transformación entre dos bases cualesquiera. Lo que a nosotros nos interesa es trabajar con \emph{transformaciones ortogonales}.

\begin{defi}
    Una \textbf{transformación ortogonal} es aquella transformación de cambio de base que permite convertir una base ortonormal $\{\hat{e}_i\}_{i=1}^n$ en una nueva base ortonormal $\{\hat{e}'_i\}_{i=1}^n$.
\end{defi}

Para asegurarnos que la transformación \eqref{eq:transformacion-coordenadas} sea una transformación ortogonal, debe además satisfacer que
\begin{align}
    \delta_{ij} & = \hat{e}'_i \cdot \hat{e}'_j \\
    & = \left( \sum_{k=1}^n a_{ik} \hat{e}_k \right) \left( \sum_{l=1}^n a_{jl} \hat{e}_l \right) \\
    & = \sum_{k=1}^n \sum_{l=1}^n a_{ik} a_{jl} (\hat{e}_k \cdot \hat{e}_l) \\
    & = \sum_{k=1}^n \sum_{l=1}^n a_{ik} a_{jl} \delta_{kl} \\
    & = \sum_{k=1}^n a_{ik} a_{jk} \ , \label{eq:delta-transformacion-ortogonal}
\end{align}
o bien, matricialmente,
\begin{equation}\label{eq:condicion-matricial}
    A \cdot (A^T) = I \ ,
\end{equation}
que al calcular el determinante, observamos que
\begin{equation}
    \det(A)^2 = 1 \ , 
\end{equation}
de modo que una transformación ortogonal deberá satisfacer que $\det(A) = 1$, caso en que se denomina \emph{transformación propia}, o bien que $\det(A) = -1$, lo que se conoce como \emph{transformación impropia}.

En particular, de \eqref{eq:condicion-matricial} podemos observar que para una transformación ortogonal, $A^T = A^{-1}$, es decir, la transpuesta de la transformación coincide con su inversa, de modo que también se satisface que
\begin{equation}
    (A^T) \cdot A = I \ ,
\end{equation}
o en notación indicial,
\begin{equation}
    \sum_{k=1}^n a_{ki} a_{kj} =  \delta_{ij} \ .
\end{equation}

De este modo, conociendo las relaciones entre los vectores unitarios, podemos reescribir el vector $\vec{v}$ como
\begin{equation}
    \vec{v} = \sum_{i=1}^n v'_i \left( \sum_{j=1}^n a_{ij} \hat{e}_j \right) \ .
\end{equation}

\subsection{Convenio de suma de Einstein}

Antes de continuar la discusión, es útil introducir el  \textbf{convenio de suma de Einstein}, que establece que en toda expresión donde se repitan dos índices iguales, \emph{existe una suma implícita sobre todo el rango de variación del índice}. Es más, el índice de suma \emph{es una etiqueta arbitraria}, por lo que puede ser renombrada a conveniencia.

Por ejemplo, las expresiones \eqref{eq:vector-posicion} y \eqref{eq:delta-transformacion-ortogonal} pueden reescribirse como
\begin{align}
    \vec{v} & = v_i \hat{e}_i \, , \\
    \delta_{ij} & = a_{ik} a_{jk} \ .
\end{align}

Además, dado que existe una suma implícita, podemos aplicar la delta de Kronecker para reemplazar índices en una multiplicación, de modo que
\begin{equation}
    a_{ij} b_{jk} \delta_{ki} = a_{ij} b_{ji} = a_{kj} b_{jk} \ .
\end{equation}

\section{Covarianza y contravarianza}

En la sección anterior, vimos que podemos reescribir un vector $\vec{x} = x_i \hat{e}_i$ en términos de una segunda base ortonormal $\{\hat{e}'_i\}_{i=1}^n$ como
\begin{equation} \label{eq:transformacion-vector}
    \vec{x} = x'_i a_{ij} \hat{e}_j \ .
\end{equation}
Comparando esta expresión con $\x = x_i \hat{e}_i$, observamos que \emph{las componentes de un vector} transforman como
\begin{equation} \label{eq:transformacion-componentes}
    x'_j = a_{ji} x_i \ , 
\end{equation}
donde la inversión de los índices en las componentes de la matriz representa que estamos considerando la matriz transversa.

Nos gustaría poder encontrar una expresión explícita para dicha matriz. Para ello, podemos derivar la expresión \eqref{eq:transformacion-componentes} respecto a las coordenadas $x_i$, obteniendo
\begin{equation}
    a_{ji} = \frac{\partial x'_j}{\partial x_i} \ ,
\end{equation}
mientras que la transformación inversa satisface
\begin{equation}
    a_{ij} = \frac{\partial x_i}{\partial x'_j} \ .
\end{equation}

De este modo, podemos reescribir la ecuación \eqref{eq:transformacion-componentes}, con lo que \emph{las componentes de los vectores transforman, bajo transformaciones ortogonales, como}
\begin{equation}
    x'_i = \frac{\partial x'_i}{\partial x_j} A_j \ .
\end{equation}

Uno podría esperar que todos los vectores transformaran según esta regla. Sin embargo, veamos qué ocurre para el vector gradiente de un campo escalar, $\nabla\phi$, donde $(\nabla \phi)_j = (\partial \phi/\partial x_j )\vec{e}_j$. Tenemos, por regla de la cadena,
\begin{equation}
    (\nabla \phi)'_i = \frac{\partial \phi}{\partial x'_i} = \frac{\partial x_j}{\partial x'_i} \frac{\partial \phi}{\partial x_j} \ ,
\end{equation}
que es \emph{una ley de transformación diferente}. Sin embargo, ambas cantidades corresponden a vectores. ¿Cómo explicamos esta diferencia?

El hecho radica en que, en efecto, ambas cantidades son vectores \emph{en coordenadas cartesianas}, pero no necesariamente \emph{en cualquier sistema de coordenadas}. Por ello, es conveniente introducir las nociones de vectores \textbf{covariantes} y vectores \textbf{contravariantes}. En este curso esta distinción no es necesaria, pero quienes deseen trabajar en gravitación o en altas energías, deberán comenzar a tener en cuenta estas nociones.

\begin{defi}
    Un vector $\vec{v}$ es denominado \textbf{contravariante} cuando, al ser sometido a una transformación ortogonal, transforma según la regla
    \begin{equation} \label{eq:contravariante}
        v'_i = \frac{\partial x'_i}{\partial x_j} v_j \ ,
    \end{equation}
    y se denomina \textbf{covariante} cuando transforma según la regla
    \begin{equation} \label{eq:covariante}
        v'_i = \frac{\partial x_j}{\partial x_i'} v_j \ .
    \end{equation}

    Las componentes del vector posición siempre transforman como vectores contravariantes.
\end{defi}


Al trabajar en sistemas no cartesianos, es común representar los vectores contravariantes con superíndices en lugar de subíndices, de modo que las reglas \eqref{eq:contravariante} y \eqref{eq:covariante} se suelen escribir como
\begin{align*}
    v'^i & = \frac{\partial x'^{i}}{\partial x^j} v^j \ , \\
    v'_i & = \frac{\partial x^j}{\partial x'^{i}} v_j \ .
\end{align*}

Al utilizar esta convención, la suma se representa al tener \emph{índices cruzados}, es decir, índices repetidos tanto como superíndice y como subíndice.

\section{Tensores Cartesianos}

\section{Operaciones con tensores}

\section{Pseudovectores y pseudotensores}


% \input{./Cap-Transformada-de-Laplace.tex}

\appendix

\input{./Cap-Espacio-de-Funciones.tex}


\nocite{*} %Referencia todo, incluso lo que no está citado.
\printbibliography[title={Referencias}]

\end{document}
